\chapter{Results}

\section{Signal Yield}

0 events are observed in each signal region, in full agreement with the background estimate of fewer than 1 event per \ac{SR}. The result of each cut made in each \ac{SR} can be seen in \autoref{tab:data_cutflow_srmm}, \autoref{tab:data_cutflow_sree}, and \autoref{tab:data_cutflow_sremu} for SR-$\mu\mu$, SR-$ee$, and SR-$e\mu$, respectively. Also in agreement with the expectation, most powerful background discriminators are the \dpt cut for electrons and cosmic veto for muons. 

This analysis was designed to be as general as possible, so the results can be applied, with some care, to any new physics signature that results in 2 leptons with high \absdz and high \pt.  

\begin{table}[htb]
\begin{center}
%\resizebox{\columnwidth}{!}{
\begin{tabular}{l  c } 
Cut & data yield\\
\hline
Pass trigger and at least 2 baseline leptons & 65484 \\
2 leading leptons are electrons & 19419 \\ 
\pt $> 65$ GeV & 14061 \\
\absdz $> 3$ mm & 11589 \\
both electrons pass isolation & 9220\\
\dpt $\geq$ -0.5 & 7\\
\chiID $< $ 2 &  2\\
\nmiss $\leq 1$ & 0\\
\dRll $> 0.2$ &  0\\ 
\hline
\end{tabular}
\caption{Cutflow for SR-$ee$ for Run 2 data.}
\label{tab:data_cutflow_sree}
\end{center}
\end{table}

\begin{table}[htb]
\begin{center}
\begin{tabular}{l  c } 
Cut & data yield\\
\hline
Pass trigger and at least 2 baseline leptons & 65484\\
2 leading leptons are muons & 45845 \\ 
$\pt> 65$ GeV & 35607\\
\absdz $ > $ 3 mm & 35326\\
both muons pass isolation & 35204\\
muons pass cosmic veto & 2\\
\tavg$ <$ 30 & 2\\
\chiID $ < $ 2 & 2\\
\nmiss $\leq 1$ & 2\\
\nprecision $\geq 3$ & 0 \\
 \chiCB $ < $ 3& 0  \\
\nphi $> 0$ &0 \\
\dRll $ > 0.2$ & 0\\ 
\hline
\end{tabular}
\caption{Cutflow for SR-$\mu\mu$ for Run 2 data.}
\label{tab:data_cutflow_srmm}
\end{center}
\end{table}

\begin{table}[htb]
\small
\begin{center}
\begin{tabular}{l  c} 
Cut & data yield\\
\hline
Pass trigger and at least 2 baseline leptons & 6548\\
2 leading leptons are a muon and an electron & 1910 \\ 
$\pt > 65$ GeV  & 128\\
\absdz$ > $ 3 mm & 107\\
both leptons pass isolation & 98\\
muon pass cosmic veto & 86\\
muon \tavg$ <$ 30 & 82\\
muon \chiID$ < $ 2 & 72 \\
muon \nmiss $\leq 1$ & 68\\
muon \nprecision $\geq 3$ & 6 \\
muon \chiCB $ < $ 3 &  0\\
muon \nphi $> 0$ & 0\\
electron \dpt $ \geq$ -0.5  &0 \\
electron \chiID $ < $ 2 & 0 \\
electron \nmiss $\leq 1$ &0 \\
\dRll $ > 0.2$ &  0 \\ 
\hline
\end{tabular}
\caption{Cutflow for SR-$e\mu$ for Run 2 data.}
\label{tab:data_cutflow_sremu}
\end{center}
\end{table}

\section{Interpretation}

While these results are designed to be applicable to many \ac{BSM} physics models, we use them to set bounds on the possible parameter space of sleptons in \ac{GMSB} \ac{SUSY} models, discussed in \todo{theory ref}, now further restricted due to the lack of events in any \ac{SR} in this analysis. These limits are computed using HistFitter \cite{histfitter}, an \ac{ATLAS} framework that combines the observed number of events with uncertainties on background predictions and MC modeling and calculates both model dependent and model independent limits using the CL$_{\text{s}}$ technique \cite{CLs-1}. 

CL$_{\text{s}}$ is a useful tool for particle physics analyses, somewhere between a purely frequentist and purlely basian method of deriving a confidence interval, the CL$_{\text{s}}$ describes the confidence in a signal-only hypothesis. A CL$_{\text{s}}(m_{\tilde{\ell}}) \leq 5\%$ indicates that at any given point in the exclusion region, the probability of having falsely excluded a slepton of that mass is less than or equal to 5\%. The smaller the value of CL$_{\text{s}}$, the more lower the probability that a slepton with a given mass has been exluded.  

Model dependent limits are set in terms of the mass and lifetime of sleptons. Four different limits are set using the results of this analysis: each selectron, smuon, or stau \ac{NSLP}, or the mass degenerate case with all three as co-NSLPs. For the selectron and smuon cases, the results from SR-$ee$ and SR-$\mu\mu$, respectively, are used. While for the stau and co-NSLP cases, all three SRs are combined.

\todo{plots when finished}

This search provides almost an order of magnitude of exclusion over the previous result from OPAL and 90 \GeV. 


Additionally, model independent limits are set on generic new physics processes. These limits are based on the visible cross-section of new physics ($\langle \epsilon \sigma^{95}_{\text{obs}} \rangle$) and the observed ($S^{95}_{\text{obs}}$) and expected ($S^{95}_{\text{exp}}$) number of signal events. 

\todo{table when finished.}
\todo{what are toys}



\section{Future Prospects}
