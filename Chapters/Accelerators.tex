\chapter{The Large Hadron Collider}

The world's largest machine is required to study the universe's smallest particles. The \ac{LHC} is circular particle collider located on the French-Swiss border outside of Geneva, Switzerland. A series of accelerators culminate in a $27$ km ring in which beams of hadrons, either protons or ions of heavier elements, are collided. There are four major experiments around the \ac{LHC} ring: \ac{ATLAS} and \ac{CMS}, general purpose detectors designed independently and with different priorities in order to serve as a cross-check for each other; \ac{ALICE}, a tracking-focused detector designed to study collisions of heavy ions; and \ac{LHCb}, an asymmetrical detector designed to study \ac{CP} violation. The \ac{LHC} has been running in its current form since 2008

The dataset used in this analysis comes from proton-proton ($pp$) collisions during Run 2 of the \ac{LHC}, which spanned from 2015-2018 and had a center of mass energy of $\sqrt{s} = 13 \TeV$.

\section{Accelerators}


\todo{describe basics of beam: circle, bunches, IP}
\todo{describe magnets}
\todo{steps of injecting}

\subsection{Luminosity}

%USED pdg-review.pdf
If the primary goal of the collider is to study rare physics processes, whether difficult measurements of the \ac{SM} or heretofore unseen \ac{BSM} physics, enough data must be produced to be able to study them. The number of events of a given process in a given data set is given by

\begin{equation}
N_{\textrm{events}} = \mathcal{L}_{\textrm{int}} \times \sigma_{\textrm{process}}
\label{eq:nevents_lumi}
\end{equation}
where $\mathcal{L}$ is the \emph{integrated luminosity} of the dataset, and $\sigma_{\textrm{process}}$ is the cross section for the given process. More data ($\mathcal{L}_{\textrm{int}}$) is required to be able to see rare events (small $\sigma_{\textrm{process}}$), and many events are required in order to have the statistical power in able to make a discovery. The integrated luminosity is the integral of the instantaneous luminosity over data taking time

\begin{equation}
\mathcal{L}_{\textrm{int}} = \int \mathcal{L} dt
\end{equation}
and the \emph{instantaneous luminosity} is related to the parameters of the accelerator. For two identical bunches with $N_1$ and $N_2$ protons per bunch colliding with frequency $f$:

\begin{equation}
\mathcal{L} = \frac{N_1 N_2 f}{4\pi \sigma_x^* \sigma_y^*} \mathcal{F}
\end{equation}
where $\sigma_x^*$ and  $\sigma_y^*$ are the \ac{rms} of the beam width in the $x$ and $y$ directions, and $\mathcal{F}$ is a factor that takes in other geometric effects such as the crossing angle and bunch length, it is generally $\mathcal{O}(1)$. It can be rewritten more specifically for the \ac{LHC} as

\begin{equation}
\mathcal{L} = \frac{N_b^2 n_b f }{4\pi \sqrt{\epsilon_n \beta^*_x \beta^*_y}}\mathcal{F}
\end{equation}
where $N_b$ is the number of protons per bunch (assuming $N_1 = N_2$), $n_b$ is the number of bunches in the beam, $\epsilon_n$ is the \emph{emittance}, which describes the spread of the particles in the bunch, and $\beta^*$ is the value of the $\beta$-function at the \ac{IP}. The $\beta$-function describes the size of the beam as a function of location. A sampling of the values of these parameters in 2018 are shown in \autoref{tab:lumi-vals}.


%USED ATLAS-lumi-measurement.pdf
% https://indico.cern.ch/event/751857/contributions/3259373/attachments/1783143/2910577/belen-Evian2019.pdf
\begin{table}
\centering
\begin{tabular}{lc}
\hline
Paramter & value  \\
\hline
$N_b$ (protons per bunch)                                           & $1.1 \times 10^{11}$   \\
$n_b$ (bunches per beam)                                            & $2544$   \\
$\beta^*$ (beam size)                                               & $.3$ m   \\
$\epsilon_n$ (beam spread)                                          & $1.8-2.2 \mu$m-radians   \\
$\mathcal{L}_{\textrm{peak}}$ (peak instantaneous luminosity)       & $21 \times 10^{33} \textrm{cm}^{-2}\textrm{s}^{-1}$   \\
space between bunches                                               & $25$ ns   \\
\hline
\end{tabular}
\caption{Beam parameters for 2018 for standard running conditions used in the data collection for this analysis. Special runs take place where these parameters are changed.}
\label{tab:lumi-vals}
\end{table}

Several of these factors can be manipulated \emph{in situ}, allowing for luminosity increasing or decreasing, depending on the need. Decreasing $\beta^*$ increases the luminosity, so the beams are \emph{squeezed} with focusing quadrupole magnets at the \ac{IP}. Nonzero beam crossing angles and longer bunches decrease luminosity. In fact, during some runs of Run 2 of the \ac{LHC}, the beam angle was changed at the beginning in order to decrease the luminosity to a level more tolerable by the experiments, referred to as \emph{leveling}. 

%The instantaneous luminosity can be measured by measuring the components directly, or it can be inferred by measuring the number of events in a process with a well measured cross section, $\sigma_{\textrm{ref}}$ that produces $N_{\textrm{ref}}$ events. By counting $N_{\textrm{exp}}$, the actual number of events produced, one can infer $\sigma_{\textrm{exp}}$. The difference between $\sigma_{\textrm{ref}}$ and $\sigma_{\textrm{exp}}$ gives information about the instantaneous luminosity from an equation similar to \autoref{eq:nevents_lumi}. 


%USED LHC-vandermeer.pdf
%USED ALTAS-lumi-measurement.pdf
In \ac{ATLAS}, luminosity is measured in two steps. First, the rate of $pp$ collisions ($\mu_{\textrm{vis}}$) is measured using detectors close to the beam pipe. This is done both online, so that adjustments to the data collection scheme can be done on the fly, as well as offline, to determine the amount of data collected. Second, the $pp$ collision rate is translated into a luminosity using a \emph{van der Meer} scan. During the scan, the beam is widened and, initially, the beams are separated in both $x$ and $y$. The beams are moved incrementally closer to each other by known amounts such that the separation between the beams is always known as the number of collisions increases. The luminosity per bunch can be expressed as

\begin{equation}
\mathcal{L}_b = \frac{\mu_{\textrm{vis}}}{\sigma_{\textrm{vis}}} f
\end{equation}
where $\mu_{\textrm{vis}}$ is measured by the luminosity detectors. The luminosity, $\mathcal{L}_b$, is known during the van der Meer scan, and so the inelastic cross section, $\sigma_{\textrm{vis}}$ can be determined and used to calculate $\mathcal{L}_b$ from $\mu_{\textrm{vis}}$. Both $\mu_{\textrm{vis}}$ and $\sigma_{\textrm{vis}}$ contain detection efficiency effects. 

%USED luminosity public results
In Run 2, the \ac{LHC} produced an integrated luminosity of $156 \textrm{fb}^{-1}$, \ac{ATLAS} recorded $147 \textrm{fb}^{-1}$, with $139 \textrm{fb}^{-1}$ available to be used for physics analyses.



\subsection{Pileup}

While high instantaneous luminosity enables the fast accumulation of data required to study rare physical processes, it also creates a dense environment in which those processes occur. The number of concurrent $pp$ collisions is called \emph{pileup}; it is substantial because the instantaneous luminosity is greater than the $pp$ inelastic scattering cross section. The average number of interactions per bunch crossing is not a static number, but the average is $33$ for all of Run 2, as seen in \autoref{fig:pileup}

\begin{figure}[htbp]
\centering
\includegraphics[width=.8\textwidth]{figures/Detector/lhc-mu.pdf}
\caption{Average number of interactions per bunch crossing during Run 2. }
\label{fig:pileup}
\end{figure}


