\chapter{Introduction}

Particle physics seeks to understand the fundamental structure of the universe by defining the minimal set of particles and interactions required to describe all physical phenomena. The \acf{SM} is the best attempt at such a description. The \ac{SM} has undergone decades of rigorous testing and can explain almost all phenomena we see in experiments. Yet it is known that the \ac{SM} is missing explanations for crucial physical phenomena like quantum description of gravity or a dark matter candidate. As a result, many \acf{BSM} theories have been developed and tested, hoping to extend and complete the picture the \ac{SM} gives. So far, no evidence for any of these theories has been seen.  

The \acf{LHC} at \acf{CERN}, a 27 km particle collider outside of Geneva, Switzerland, is the largest particle physics experiment in the world, and provides an extremely effective environment to test the \ac{SM} and a wide variety of \ac{BSM} theories. This thesis uses data from the \acf{ATLAS} experiment, one of the four largest experiments along the \ac{LHC} ring.

Beams of protons circulate and collide in the \ac{LHC}, and if two protons collide with sufficiently high energy, massive particles can be created; $\sqrt{s} = 8 \TeV$ collisions enabled the 2012 discovery of the Higgs boson with mass of 125 \GeV. The \ac{LHC} provides 60 million collisions per second, enabling physicists to search for new and rare physical phenomena. Unfortunately, after 8 years of data taking no evidence of \ac{BSM} physics has been found. Data taking is scheduled to resume in 2022 with only a moderate increase in collision energy and about a factor 2 more data. 

This is a call to expand the suite of \ac{BSM} searches by re-examining the assumptions made in searches that have been performed so far. What could we have missed in our search for new physics at the TeV scale? The \ac{LHC} detectors are designed to look the decays of short-lived, heavy particles with the assumption that the decay products will be \emph{prompt} and trace back to the collision point. This misses a large range of intermediate lifetimes of possible \ac{BSM} particles that decay inside of the detector material.  These signatures are challenging, but not impossible, to identify with \ac{ATLAS} as they result in \emph{displaced} \ac{SM} particles that do not point back to the collision point. Many \ac{SM} particles are long lived, like muons or neutrons, and many \ac{BSM} theories predict particles with lifetimes that result in displaced decays. This thesis presents a search for one such signature.

This thesis presents a search for two displaced \ac{SM} leptons, either electrons or muons, that are not connected by a displaced vertex \cite{conf}. Due to the displacement and lack of vertex, a \ac{BSM} particle decaying to this signature would be vetoed by all other analyses at the \ac{LHC}, even those targetting \acp{LLP}. This search has unique sensitivity to a specific \acf{GMSB} \acf{SUSY} model where the \acf{LSP} is the superpartner to the graviton, the gravitino, and the \acf{NLSP} is a slepton (\slep), the superpartner to a lepton. The \slep is long lived because it must decay to the gravitino through the very weak gravitational coupling. The last time this model was explored was in the OPAL, ALEPH, DELPHI, and L3 experiments at \acf{LEP} \cite{opal}, where masses up to about 90 GeV were probed for the full range of lifetimes, immediately decaying to detector stable and all possible signatures in between. This search probes almost an order of magnitude of mass phase space in a limited lifetime phase space, and in a significantly more challenging environment than \ac{LEP}.

Since this is the first search for displaced leptons at the \ac{LHC} particle selection algorithms and robust data-driven background estimates needed to be developed. This search for displaced leptons uses $139~ \ifb$ of data collected by \ac{ATLAS} during Run 2 of the \ac{LHC}. Major backgrounds come from fakes of the reconstruction algorithms and muons from cosmic rays. Less than 1 background events are predicted and zero events are seen, and so limits on the mass and lifetime of \slep. 

This thesis is organized into three main sections: first the search is motivated theoretically, then the experimental setup is described, and finally the search strategy and its results are presented.

\autoref{chap:theory} provides theoretical motivation for searches for \ac{SUSY} and in particular long lived and \ac{GMSB} \ac{SUSY}.

\autoref{chap:LHC} describes the \ac{LHC} and its design and operation.

\autoref{chap:ATLAS} describes the \ac{ATLAS} subdetectors and how they are used to measure particles.

\autoref{chap:eventreco} details particle reconstruction algorithms and the modifications made for this analysis.

\autoref{chap:datamc} describes the \ac{ATLAS} data and \acf{MC} simulation of the signal model.

\autoref{chap:llps} provides context for \ac{LLP} decays and searches.

\autoref{chap:context} provides experimental context for the signature and model.

\autoref{chap:eventselection} describes the analysis strategy, lepton selection requirements and final event selection criteria.

\autoref{chap:backgrounds} details the backgrounds to this signature and the algorithms for estimating them.

\autoref{chap:systematics} describes the uncertainties applied to signal \ac{MC} in order to use the data to make a statement about the \ac{SUSY} model.

\autoref{chap:results} presents the unblinded results.

\autoref{chap:interp} discusses the statistical interpretation of the analysis.  

\autoref{chap:conclusions} provides reflections on the analysis as well as possible improvements for future \ac{ATLAS} analyses for displaced leptons.


