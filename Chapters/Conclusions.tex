\chapter{Conclusions and Future Improvements}
\label{chap:conclusions}

The lifetime is an underexplored region of phase space for many natural \ac{BSM} theories and this thesis describes a search that covers a large gap in signature space at the \ac{LHC}. In $139~ \ifb$ of $\sqrt{s} = 13~\TeV$ \ac{LHC} data collected by the \ac{ATLAS} detector, no signs of \ac{BSM} physics are seen. Three orthogonal signal regions are defined and less than 1 background event is predicted in each. 0 events are seen and limits are set on the possible mass and lifetime of \slep in \ac{GMSB} \ac{SUSY} models. For a lifetime of 0.1 ns, selectron NLSP, smuon NSLP, stau NSLP, and co-NLSP scenarios are excluded for slepton masses up to 720~\GeV, 720~\GeV, 370~\GeV, and 830~\GeV, respectively, exceeding the LEP co-NSLP limit~\cite{LEP-comb} by almost an order of magnitude. Co-NSLP events are also excluded up to 10 ns for masses below 330~\GeV.

A minimal, though substantial, set of optimizations were done to obtain this result, but future analyses using the same dataset could improve upon this result in several ways. First, further optimization could be done of the electron reconstruction algorithm in order to boost efficiency at high \absdz as well as reduce the systematic uncertainty due to the variation in the lepton displacement selection efficiency. Also, additional signal regions could be defined using a \ac{MET} trigger to increase sensitivity. This signal model has real \ac{MET} due to the gravitinos (and neutrinos in the \stau decays). More importantly, muons are not included in the \ac{MET} calculation at the trigger stage, so an \ac{MET} trigger is an effective displaced muon trigger. A bolder improvement would be to reconstruct all recorded events with \ac{LRT}, instead of the filtered 10\% that is currently used. This would enable more creativity in signal region design and more centralized displaced lepton identification algorithms. 

Additional optimizations should be performed to prepare for an analysis in Run 3. In particular, this search suffered from trigger limitations. A simple improvement to the data collection scheme would be to introduce an EM-only + MS-only trigger. This would likely have a low enough fake rate that it could be added in relatively easily and \pt and $\eta$ requirements on the single EM-only and MS-only signatures could be relaxed. A more exciting improvement would be to implement \ac{LRT} in the trigger and design trigger-level displaced lepton identification algorithms. This would likely have a low enough rate to reduce the \pt requirements substantially. 

This result should also be interpreted in conjunction with other searches, including prompt searches and those for disappearing tracks and stable massive particles, to probe the full possible lifetime space of sleptons and make a more lifetime-inclusive statement about \ac{GMSB} \ac{SUSY} at the \ac{LHC}. Furthermore, the minimal event-level requirements make this result model-independent and applicable to any \ac{BSM} decay resulting in displaced leptons.
