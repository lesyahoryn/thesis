\chapter{Conclusions}
\label{chap:conclusions}

The lifetime is an underexplored region of phase space for many natural \ac{BSM} theories and this thesis describes a search that covers a large gap in signature space at the \ac{LHC}. In $139 \ifb$ of $\sqrt{s} = 13~\TeV$ \ac{LHC} data collected by the \ac{ATLAS} detector, no signs of \ac{BSM} physics are seen. Three orthogonal signal regions are defined and less than 1 background event is predicted in each. 0 events are seen and limits are set on the possible mass and lifetime of sleptons in \ac{GMSB} \ac{SUSY} models. For a lifetime of 0.1 ns, selectron NLSP, smuon NSLP, stau NSLP, and co-NLSP scenarios are excluded for slepton masses up to 720~\GeV, 720~\GeV, 370~\GeV, and 830~\GeV, respectively, exceeding the OPAL co-NSLP limit~\cite{Abbiendi:2005gc} by almost an order of magnitude. Co-NSLP events are also excluded up to 10 ns for masses below 330~\GeV.

A minimal, though substantial, set of optimizations were done to enable this result, but future analyses could improve upon this result in several ways. First, further optimization could be done of the electron reconstruction algorithm in order to boost efficiency at high \absdz as well as reduce the systematic uncertainty due to the variation in the lepton displacement selection efficiency. 

An additional signal region could be added using a \ac{MET} trigger could increase sensitivity. This signal model has real \ac{MET} due to the gravitinos, but also muons are not included in the \ac{HLT} \ac{MET} calculation, so an \ac{MET} trigger is an effective displaced muon trigger. This is particularly potentially effective for low mass staus, which really suffer from high single lepton trigger \pt thresholds. 

A bolder improvement would be to reconstruct all recorded events with \ac{LRT}, instead of the filtered 10\% that is currently used. This would enable more creativity in signal region design.

This result should also be interpreted in conjunction with other searches, including prompt searches and those for stable massive particles, to probe the full possible lifetime space of sleptons and make a more lifetime-inclusive statement about \ac{GMSB} \ac{SUSY} at the \ac{LHC}. 
