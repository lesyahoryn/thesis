\chapter{Signal Systematics}

In order to use this analysis to make a statement about a potential \ac{BSM} model, the extent to which the signal \ac{MC} correctly simulates the real physical environment must be evaluated. Differences between data and \ac{MC} are studied. Where possible, the \ac{MC} is corrected to better represent the data using \emph{scale-factors}, and in other cases systematic uncertainties are applied to the final result. 

There are many standard systematic uncertainties derived by \ac{ATLAS}, for example the measurement of the energy of jets or the sagitta of muons, that do not have a large impact on this analysis with nonstandard physics objects. A major source of systematic uncertainty on the signal \ac{MC} in this analysis comes from the modeling of the selection of displaced leptons, for which there is no comparable data. As a result, this is evaluated in several steps: first the trigger, reconstruction, and selection efficiencies are compared for prompt leptons in the same physics process in data and \ac{MC}; then the tracking efficiency is compared between signal muons and cosmic muons, a different physical phenomenon that also provides high \pt, high \absdz tracks; finally, the lepton reconstruction efficiency is studied with respect to displacement and a conservative additional uncertainty to account for any missed effects in data. Other event-level systematic uncertainties are applied to account for mismodeling of pileup and theory assumptions made during \ac{MC} generation. 

\todo{table when finalized}

\section{Displaced Lepton Reconstruction}

\section{Pileup Modeling}

\section{Theory Uncertainties}

