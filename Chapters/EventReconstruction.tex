\chapter{Event Reconstruction}
\label{ch:EventReconstruction}

Event reconstruction is the process by which detector signals are turned into objects that can be used for physics analysis. This is a complex process that requires a great deal of focused effort by the \ac{ATLAS} collaboration. First, digital signals from the detector are collected into tracks and clusters, then they are combined to form first-stage physics objects. Then, a identification steps is performed, where quality requirements are placed on the first-stage objects to classify them into particles like electrons, muons, and jets that can be used in physics analyses. 

These algorithms are centrally developed by the collaboration and designed to reconstruct and identify prompt objects ($|d_{0}| < 10| \textrm{mm}$). This section describes this process for objects which are relevant to this analysis, as well as the changes to these algorithms that we have implemented to be able to study displaced objects. Other objects, such as jets, taus, and missing transverse energy, are also reconstructed in this analysis, though the final event selection remains agnostic to their existence or quality, but does perform a overlap removal process to ensure that the same particle is not accidentally reconstructed as two different objects. 

Reconstruction of tracks, including modifications to reconstruct tracks with high impact parameter, is described in \autoref{sec:trackreco}. Electron and muon reconstruction, as well as their modifications, are described in \autoref{sec:elecreco} and \autoref{sec:muonreco}, respectively. The reconstruction of jets, photons, and $\tau$ leptons is not discussed here. All of these objects are reconstructed in this analysis, though no selection is made on them. 


%-----------------------------
% Track Reconstruction
%-----------------------------
\section{Track Reconstruction}
\label{sec:trackreco}

Track reconstruction is the process by which \ac{ID} data is converted into particle trajectories. This is a complicated process, due to both the density of each event in the detector (in Run 2, there were an average of 40 collisons per bunch crossing, all of which produce hadronic sprays of particles), as well the helical trajectory (in $\phi$) the particles take due to the solenoidal magnetic field. Tracks are described by five parameters with respect to the beamspot position: \dzero (the transverse point of closest approach, or transverse impact parameter), \zzero (the longitudinal point of closest approach, or longitudinal impact parameter), $\phi$ (the azimuthal angle of the track momentum), $\theta$ (the polar angle of the track momentum), and $q/p$ (the ratio of the track's charge to the magnitude of its momentum).


%From ATLAS-tracking-algo.pdf
First, information from the \ac{ID} are converted into three-dimensional \emph{space-points}. In the Pixel detector, this is simply one cluster, while for the \ac{SCT}, it is taken from both sides of a strip layer.

%From ATLAS-LRT.pdf
Tracking in ATLAS is performed in two step. During the first step, called \emph{inside-out}, tracks are seeded from the silicon layers. In the Pixel and \ac{SCT} detectors, track seeds are formed from sets of three space-points, each from a separate layer. If the seed passes an assortment of selection criteria, including on the \pt and \dzero, track candidates are built using a combinatorial Kalman filter. These requirements serve to cut down on the number of tiemes the computationally expensive Kalman filter must be implemented. Multiple track candidates can be built from the same seed.

\editnote{I'd like to learn what a Kalman filter is}

%From ATLAS-tracking-algo.pdf
Since all possible track combinations are created in the previous stage, an \emph{abiguity solving} step is now required. Tracks are scored based on a variety of critera, the number of holes (detector elements the track intersected with, but do not contain a cluster), $\chi^{2}$ (to prioritize tracks with a better fit), and \pt (to prioritize tracks with a higher \pt). A further requriement that no more than two tracks may share the same cluster to reduce the number of duplicate tracks; however, a cluster may be removed from a track to stay within this limit and the track is then re-scored. 

Next, the track candidates are extended into the outer \ac{TRT} using a classical track extrapolation, then the track is scored using a method similar to the ambiguity solver. If this extension is successful, the track is labeled as having a \emph{\ac{TRT} extension}, if not, the track can still be retained, particularly at large $|\eta|$. However, if the score after TRT extension is worse than the silicon-only score, the track is rejected. This completes step one.

\editnote{I'd like to learn how classical track extrapolation works}


Step two takes an \emph{outside-in} approach, where track segments are reconstructed in the \ac{TRT}, seeded by depsoits in the \ac{EM} calorimeter. These segments are then extended inward to the silicon detectors, where any clusters not used in step one can be associated to the track. The extension inward is not required, as \ac{TRT}-only tracks are used for reconstructing converted photons. 


\subsection{Large Radius Tracking}

\ac{LRT} is required to reconstuct tracks with a impact parameters larger than what is allowed by the seed requirements in the \ac{ST} algorithm. These requirements are implemented during seed selection and extrapolation and are designed around the maximum displacement of a b hadron decay (a few mm) and no more.

\ac{LRT} is an optional third step of the tracking algorithm described above. It uses the same \emph{inside-out} tracking algorithm, but relaxes various requirements that allow for a much more inclusive track collection. The major changes are summarized in \ref{tab:LRT}. These cuts are applied during both the seeding and extension steps. Additionally, \ac{LRT} uses a sequential Kalman filter as opposed to the combintorial Kalman filter uses in \ac{ST}.

\editnote{What is sequential vs combinatorial Kalman filter}

\ac{LRT} is required for this analysis, but cannot be applied to all events in the Run 2 dataset. The full event reconstruction with \ac{LRT} takes about 2.5 times longer than with \ac{ST}. So events are filtered based on the triggers that selected them, such that this algorithm is only run on about 10\% of the dataset.


\begin{table}
\centering
\begin{tabular}{lcc}
\hline
Cut & \ac{ST} & \ac{LRT}  \\
\hline
$d_{0}^{\textrm{max}}$ (mm)   & 10   & 300 \\
$z_{0}^{\textrm{max}}$ (mm)   & 250   & 1500 \\
$ |\eta^{\textrm{max}}|$        & 2.7   & 5 \\
Max shared silicon modules    & 1     & 2 \\
Min unshared silicon clusters   & 6     & 5 \\
Min number of silicon hits   & 7     & 7 \\
\hline
\end{tabular}
\caption{Most important cuts that differ between \ac{ST} and \ac{LRT}}
\label{tab:LRT}
\end{table}

\editnote{Add in some efficiency plots}

%-----------------------------
% Muon Reconstruction
%-----------------------------
\section{Muons}
%USED atlas-muon-reco.pdf
\label{sec:muonreco}
\subsection{Standard Reconstruction and Identification}

Muons are reconstructed by combining a \ac{MS} track with an \ac{ID} track. Then, at the identification stage, quality requirements are imposed on the combined tracks to reduce improve the purity of the muon collection. For this analysis, the muon reconstruction remains unchanged, while changes are made at the identification stage.

\subsubsection{Muon Track Reconstruction}
To reconstruct \ac{MS} tracks, a Hough transform is used to search for hits in each \ac{MDT} chamber to find hits following a trajectory on the $\eta$ plane of the detector. These hits are fit to a straight line within each chamber to form \emph{segments}. Co-located \ac{RPC} and \ac{TGC} hits are used to measure the $\phi$ coordinate. 

Hits from segments in various layers are fit to form track candidates. This fitting is first seeded from segments in the middle layers of the \ac{MS} where more trigger hits are available, then extrapolated inward and outward. A next pass is done using inner and outer station segments as seeds. The extrapolation relies on relative positional and angular information, as well as the fit quality and hit multiplicity of the segments. Two segments are required to make a track, except in regions with limited detector coverage, where one high quality segment is sufficient. After all extrapolation, an overlap removal procedure is performed, allowing for a segment to be shared between at most two tracks. A $\chi^{2}$ test is performed, where outliers hits can be removed from the track and additional hits consistent with the track candidate's trajectory can be added.

\subsubsection{Combined Muons}
Next, the \ac{MS} tracks are combined with \ac{ID} tracks to form \ac{CB} muons, a track fit over the two tracks. \ac{CB} muon tracks are generally seeded from the \ac{MS}, then extrapolated inward and matched to an \ac{ID} track, but the inverse is also allowed. Hits from the \ac{MS} may be added or removed to improve the fit between the two tracks.


\editnote{what are "trigger hits"}

\subsubsection{Muon Identification}

This analysis uses the default muon working point for ATLAS analyses, called a \emph{medium} muon. This working point places a requirement on the number of \ac{ID} and \ac{MS} hits that comprise the track to ensure a robust momentum measurement. At least 1 pixel hit, at least 5 \ac{SCT} hits, and at least 10\% of the \ac{TRT} hits associated to the object are included in the final fit are required. It also requires there to be fewer than 3 holes in the silicon tracking layers, where holes are defined as a missing hit from a sensor traversed by the track. Furthermore, \ac{MS} track must have at least 3 hits in at least 2 \ac{MDT} layers. In the crack region $|\eta| < 0.1$, \ac{MS} tracks with at least three hits in only one \ac{MDT} layer are allowed provided there are no holes in the track. Finally, a loose requirement is placed on the consistency between the \ac{MS} and \ac{ID} tracks. Namely, the \emph{q/p significance}, the difference between the charge and momentum ratio in the \ac{ID} and \ac{MS} divided by their uncertainties summed in quadrature, is required to be less than 7. 

\editnote{insert plot about muon performance}


\subsection{Modifications}

For this analysis, muons are reconstructed after \ac{LRT} is performed and the reconstruction and identification efficiency is quite high. Furthermore, we remove the requirement that the \ac{ID} track has at least one pixel hit, further improving the efficiency at high $d_{0}$. The effect of these improvements is show in fig %TODO


This modification of the muon identification increases the fake rate of muons, so again we impose quality requirements that are independent of displacement. Primarily, we require the muon to have at least two \ac{MS} layers with at least three precision hits, that the muon have at least one $\phi$ measurement (otherwise the \ac{MS} $\phi$ measurement is taken as the center of the \ac{MDT}, with an uncertainity of 0.2) and require the $\chi^{2}_{CB}/N_{DoF} < 3$. The $\chi^{2}_{CB}$ requirement is, in effect, a requirement on the consistency of the $p_{T}$ of the two tracks. 


\editnote{plots of quality vars}

\editnote{make muon reco, ID, signal efficiences with higher stats -- grab from INT when available}



%-----------------------------
% Electron Reconstruction
%-----------------------------
\section{Electrons}
%USED atlas-electron-reco.pdf atlas-electron-reco-sliding-window.pdf
\label{sec:elecreco}

Electrons are reconstructed using clusters from the \ac{EM} calorimeter as well as tracks from the \ac{ID}. Electron reconstruction brings more complication and ambiguity than muon reconstruction because of the presence of photons, converted photons, and bremsstrahlung radiation from electrons moving through material. These factors make the identification and accurate measurement of electrons quite challenging. More than in muon reconstruction, it relies on displacement-based quality information, posing a problem for this search. We modify these requirements and use \ac{LRT} tracks, but have a resulting lower selection efficiency. 

\subsection{Standard Reconstruction and Identification}

\editnote{schematic from page 10 of atlas-electron-reco}

\subsubsection{Cluster Reconstruction}

First, clusters are formed from $\eta \times \phi$ towers of size $\Delta \eta \times \Delta \phi = 0.025 \times 0.025$, which is the granularity of the second layer of the \ac{EM} calorimeter where about 80\% of the energy in a shower is deposited. In each region, the energy deposited in all layers of the calorimeter is summed and are used as input to a seeding algorithm to form clusters. 

A \emph{sliding window} algorithm is used to form clusters. In this algorithm, an $3 \times 5$ tower window is moved across each tower, the energy is summed inside of this window and if the sum is a local maximum and is above a threshold of $\et > 2.5 \GeV$, this window is considered a cluster. A duplicate removal process is then performed for nearby, similar energy clusters, keeping only the cluster with the largest $E_{T}$. Inefficiency in the cluster reconstruction step is negligible compared to the uncertainty in the next two steps. The efficiency of this step is 65\% at $\et = 4.5 \GeV$ and $> 99\%$ above $\et = 15 \GeV$.

\subsubsection{Tracking}
Since electrons are so light, they can lose a significant amount of their energy due to bremsstrahlung radiation as they traverse the \ac{ID}, thus resulting in a track seed that cannot be extended to the requisite number of silicon layers using the processes described in \ref{sec:trackreco}. Thus, a second pass at tracking, allowing for 30\% energy loss due to bremsstrahlung radiation at each detector layer is performed in the vicinity of a good quality \ac{EM} cluster. Here, the track candidate $p_{T}$ is lowered to $400 \MeV$ (compare to $1 \GeV$), but still uses the hypothesis that the track like that from a pion. If this fit still fails, a third pass is performed using the assumption that the track is like that from an electron, allowing for an additional degree of freedom in the $\chi^2$ calculation that accounts for additional radiation. In all, this gives 80\% reconstruction efficiency at $\et = 1 \GeV$ and 98\% above $\et = 10 \GeV$. 

This loosened track fitting requirements allow for increased efficiency, but the resulting tracks do not correctly account for the energy loss of the electron to the material. An additional tracking pass, using an optimized \ac{GSF} is used to correct for this. 

This procedure is performed on tracks with $>4$ silicon hits and roughly match to an \ac{EM} cluster. In general, this matching is done based on the angles of the track and the cluster, which creates a problem for tracks with high $d_{0}$. This needed to be altered in order to efficiently reconstruct electrons with high \dzero. 


The \ac{GSF} procedure is based on a series of Kalman filters running in parallel. Six Gaussian filters are used to describe energy loss in material, and up to twelve are used to model the track parameters. This procedure also accounts for the increase in curvature caused by the decrease in momentum due to energy loss in material, improving the calculation of track parameters. An example of this is show in the figure below. The reconstruction efficiency for this step is around 98\% for electrons with $\et > 30 \GeV$. 

\editnote{add qxd0 figure from p15 of atlas-electron-reco}


\subsubsection{Combined Reconstruction}
To put it all together, \ac{GSF} tracks are then matched to seed calorimeter clusters then the final cluster size is determined. Tracks and clusters are matched with stricter requirements, requiring that their $\phi$ measurement is within $+0.05$ or $-0.10$. It is possible that several tracks might match to the same cluster, but a primary track is assigned based on its proximity and quality. If the track can be associated to a vertex, it is classified as a potential photon conversion, not an electron. Electrons are further distinguised from photon conversions using their $E/p, p_{T}$, number of pixel hits, and relationship to a secondary vertex. 

Final clusters are formed by looking in a window around the seed cluster. Offline calibration of the reconstructed cluster is required. The reconstructed electron's energy measurment is taken from its full cluster, and directional information taken from its track. At $\et > 15 \GeV$, there is a 99\% efficiency to reconstruct an electron (provided it has at least one pixel hit and at least seven total silicon hits on track). 

\editnote{add figure from p 12 of atlas-electron-reco}

\subsubsection{Identification}

Electrons are identified using a likelihood that further distinguishes them from photons and converted photons. Many factors enter the electron identification likelihood, listed in table \editnote{TABLE}. The likelihood is more flexible than a simple cut-based method, allowing for electrons to fail one selection criterion. It also allows for the use of discriminating critera which have relatively similar shapes in signal and background.


\subsection{Modifications}
To be able to reconstruct electrons with high impact parameter, several changes needed to be made to the reconstruction and identification algorithms. 

First, the reconstruction algorithm needed to be changed to allow tracks with a high $d_{0}$ to be extrapolated to clusters, and remove a requirement on pixel hits, and instead only require a total number of silicon hits. The reconstruction is then run on the track collection including \ac{LRT} tracks. 

At the identification stage, we remove variables concerned with $d_{0}$ from the likelihood consideration, but do not retrain the likelihood itself. We also remove the cut on the number of silicon hits on top of that made at the reconstruction stage. 

After these modifications, we introduce many fake electrons, primarily resulting from a fake \ac{LRT} track being associated to an \ac{EM} cluster from a photon. The most powerful discriminator is the consistency in the $p_{T}$ as measured by the track and the cluster, defined as $\Delta p_{T} = p_{T}^{track}/p_{T}^{e}$. Furthermore, we require the primary track to be good quality, with $\chi^{2} < 2$ and at maximum one missing hit on track after the innermost hit. 

\editnote{plots of quality vars}

\editnote{make electron reco, ID, signal efficiences with higher stats -- grab from INT}




\section{Isolation}

In this analysis, both muons and electrons are required to be isolated to reduce background from heavy flavor decays. That is, they are not expected to be surrounded by much other activity from the hard scatter in either the Inner Detector or Calorimeters. Isolation is generally measured in terms as the scalar sum of energy in a radius $\Delta R = \sqrt{(\Delta \eta)^2 + (\Delta \phi)^2}$ divided by the lepton's $p_{T}$. Track based isolations are called $p_{T}^{\textrm{varcone}}$ and calorimeter based isolations called $E_{T}^{\textrm{topocone}}$. 

For example, the track-based isolation called $p_{T}^{\textrm{varcone30}}$ is the scalar sum of the transverse momenta of tracks with $p_{T} > 1 \GeV$ in a cone of $\Delta R = \textrm{min}(10 \GeV/p_{T}^{\ell}, 0.3)$, then a value is selected to determine what should be considered ``isolated''. Whereas the calorimeter-based isolation just counts the energy in a specified \dR, not as a funtion of the energy of the particle. Isolation working points are then centrally defined as a combination of cuts on the track-based and calorimeter-based isolation. 

These definitions are much simpler for muons than for electrons. Electrons are very likley to emit bremsstrahlung radiation as they traverse the inner detector and those photons can then convert back into electrons. The tracks from these secondary particles are considered part of the electron's \pT. Furthermore, of course the electron leaves its energy deposit in the \ac{EM} calorimeter and this energy must be subtracted from the $E_{T}^{\textrm{topocone}}$ calculation.


In this analysis we use the following working points
\editnote{grab isolation requirements from INT when finalized}



