\chapter{Event Selection}
\label{chap:event-selection}

From the entire 139 \ifb~ collected by \ac{ATLAS}, this analysis aims to select events with long lived sleptons that, should they exist, have cross sections many orders of magnitude smaller producing only a handful of events. The event selection in this search is designed such that there are no backgrounds from Standard Model processes.

Since this analysis has not been done before in the conditions of the \ac{LHC}, extensive optimization of the signal regions and lepton selection criteria needed to be done. A cut in the event selection should decrease or eliminate some background, but have little to no impact on the signal. For example, quality criteria are designed for the displaced leptons that create the target signature that ensure the leptons are well measured in the detector. Leptons from slepton decays will pass these quality criteria, but fake leptons from reconstruction failures generally will not.

This analysis uses two definitions of leptons: \emph{baseline} and \emph{signal}. Baseline leptons are required to pass the reconstruction and identification criteria described in \autoref{sec:el_reco_mods} and \autoref{sec:mu_reco_mods} and have $\pt > 50 \GeV$ and $\absdz > 2$ mm. Signal muons are required to further pass bespoke quality requirements, an isolation cut, and have $\pt > 65 \GeV$ and $\absdz > 3$ mm. Displacement-independent quality variables, described in the rest of this chapter, were defined specifically for this analysis as many standard quality criteria place requirements on the \absdz, \absz, or the number of hits in the Pixel detector, all of which would limit the signal efficiency.

\section{Event Requirements}

Events must pass a trigger in order to be recorded by the \ac{ATLAS} detector. Three different triggers are used in this analysis and the data separated into three orthogonal regions based on the topology of the event, described in \autoref{tab:triggers}. The trigger is required to pass in the appropriate region for the event to be selected.

Each event is required to pass a standard set of \ac{ATLAS} event quality preselection criteria. Specifically, these include detector error flags which reject events with noise bursts or data corruption, or events in periods where any sub-detector was operating suboptimally in some way. Events are required to have at least one \ac{PV} with $|z| < 200$ mm; if there are multiple \ac{PV}s, the one with the highest sum of track \pt is identified as the \ac{PV} and the rest are considered pileup vertices. 

From events that pass the previous two requirements, three signal regions are defined, SR-$ee$ with two signal electrons, SR-$\mu\mu$ with two signal muons, and SR-$e\mu$ with one signal electron and one signal muon. The leptons are required to be well separated with $\dR_{\ell\ell} > 0.2$. This eliminates background from lepton pairs that could be created from an interaction with the material of the detector. Finally, events are required to have zero cosmic tagged muons. The cosmic tag and associated background will be discussed in \autoref{sec:cosmics}. Other signal, control, and validation regions are defined with different numbers of leptons and cosmic tags are used for the background estimates and their validation. The signal regions are optimzed and backgrounds estimated while keeping the \ac{SR}s blinded, so control (where backgrounds are estimated) and validation regions (where additional studies are done) are defined. A full list of all signal, control, and validation regions can be seen in \autoref{tab:regions}. 


\section{Electrons}


\section{Muons}

\begin{table}[htb]
\small
\begin{center}
\begin{tabular}{l|c}
\multicolumn{2}{c}{Muon Selections}\\
\hline
\pt & $> 65~\gev$ \\
\absdz & $> 3$ mm \\
$|\eta|$ & $< 2.5$ \\
Isolation & \texttt{FCTight} \\
\nprecision & $\geq$ 3 \\
\chiCB & $< 3$ \\
\nphi  & $> 0$ \\
\chiID & $< 2$ \\
\nmiss & $\leq$ 1 \\
$|\tavg|$ & $< 30$ \\
Pass Cosmic Veto & True \\
\hline
\end{tabular}
\caption{Overview of muon signal selections.}
\label{tab:muon_sel}
\end{center}
\end{table}


\begin{table}[htb]
\small
\begin{center}
\begin{tabular}{l|c}
\multicolumn{2}{c}{Electron Selections}\\
\hline
\pt & $> 65~\gev$ \\
\absdz & $> 3$ mm \\
$|\eta|$ & $< 2.47$ \\
Isolation & \texttt{FCTight} \\
\dpt & $\geq$ -0.5 \\
\chiID & $< 2$ \\
\nmiss & $\leq$ 1 \\ 
\hline
\end{tabular}
\caption{Overview of electron signal selections.}
\label{tab:electron_sel}
\end{center}
\end{table}


\section{Signal and Control Regions}

\begin{sidewaystable}[htb]
\small
\begin{center}
\begin{tabular}{|l|l|c|c|c|}
\hline
Purpose & Name & \# of Leptons & \# of Cos.Tags & Additional Requirements\\
\hline\hline
%\multicolumn{5}{|c|}{Signal Regions} \\
%\hline
\multirow{3}{*}{Signal Regions} & SR-$ee$ 	& $\geq$ 2 $e$ 						& 0  & \\
								& SR-$\mu\mu$ & $\geq$ 2 $\mu$ 					& 0  & \\
								& SR-$e\mu$ 	& $\geq$ 1 $e$, $\geq$ 1 $\mu$  & 0  & \\
\hline\hline
\multicolumn{5}{|c|}{Control Regions} \\
\hline
\multirow{2}{*}{Fake Estimation} 	&CR-$ee$-fake		& $\geq$ 2 $e$					& 0		& $\geq$ 2 loosened electrons, not in SR-$ee$ (Section~\ref{sec:ee_fakes}) \\
									&CR-$e\mu$-fake		& $\geq$ 1 $e$, $\geq$ 1 $\mu$ 	& 0		& $\geq$ 2 loosened leptons, not in SR-$e\mu$ (Section~\ref{sec:em_fakes}) \\
\hline
Heavy Flavor Estimation 			&CR-$\mu\mu$-hf		& $\geq$ 2 $\mu$				& 0 	& $\geq$ 1 anti-isolated $\ell$, loosened \pt and \dz (Section~\ref{sec:fake_mus}) \\
\hline
\multirow{2}{*}{Cosmic Estimation} 	&CR-$M_{\mathrm{full}}$  & $\geq 1 \mu$		& $\geq 1$ 	    &  includes muons failing \nphi and \nprecision cuts (Section~\ref{sec:cos_est}) \\
									&CR-$\mu\mu$-topbad	& $\geq$ 2 $\mu$				& 0  	& one signal and one loosened muon (Section~\ref{sec:cos_est})\\
\hline\hline
\multicolumn{5}{|c|}{Validation Regions} \\
\hline
\multirow{4}{*}{Cut Evaluation}		&VR-$M_{\mathrm{narrow}}$& $\geq 1 \mu$		& $\geq 1$ 	& using narrow cosmic tag \\ 
									&VR-$e$					& 1 $e$, 0 $\mu$	& 0 		& electron is baseline  \\
									&VR-$\mu$				& 0 $e$, 1 $\mu$	& 0 		& muon is baseline  \\   
% evaluating cleaning cuts
\hline
\multirow{4}{*}{Fake Validation}	&VR-$ee$-fake			& $\geq$ 2 $e$					& 0 & inverted \dpt selection (Section~\ref{sec:ee_fakes}) \\
									&VR-$ee$-fake-hf		& $\geq$ 2 $e$					& 0 & $\geq 1$ anti-isolated $\ell$, loosened quality (Section~\ref{sec:ee_fakes}) \\
									&VR-$e\mu$-fake			& $\geq$ 1 $e$, $\geq$ 1 $\mu$ 	& 0 & 1 $e$ fails \dpt, 1 $\mu$ fails \chiCB (Section~\ref{sec:em_fakes}) \\ 
									&VR-$e\mu$-fake-hf		& $\geq$ 1 $e$, $\geq$ 1 $\mu$	& 0 & 1 $e$ fails \dpt, no isolation req., loosened quality (Section~\ref{sec:em_fakes}) \\
% Validation of fake estimates, including HF contribution
\hline
\multirow{3}{*}{Cosmic Validation}	&VR-$\mu M_{\mathrm{full}}$  & $\geq$ 2 $\mu$			& 1 	& \\		
									&VR-$\mu$-narrow 			& $\geq$ 1 $\mu$			& 0* 	& using narrow cosmic tag \\
									&VR-$\mu M_{\mathrm{narrow}}$& $\geq$ 2 $\mu$			& 1* 	& using narrow cosmic tag \\
% Validation of cosmics

\hline

\hline
\end{tabular}
\caption{Summary of signal, control and validation regions used in the analysis. All regions are defined exclusively by their reconstructed leptons. In the table, all lepton requirements are made on signal leptons, unless otherwise noted. Lepton requirements are imposed on the two baseline leptons with the leading \pt in the event. Except where noted, additional leptons are permitted, but are not considered when determining the region an event falls into. In each region, the appropriate trigger selection from Table~\ref{tab:triggers} is made. For more details on lepton selection criteria, see section \ref{sec:objs}. In region names, a capital $M$ denotes a cosmic-tagged muon, and for requirements on numbers of cosmics, all leptons in the event are considered. An * on the number of cosmic tags denotes the number of narrow tags, not number of full tags as used in the SR.}
\label{tab:regions}
\end{center}
\end{sidewaystable}

\section{Acceptance and Efficiency}

Acceptance is defined as the fraction of events that could enter the \ac{SR} based on their kinematics and efficiency is fraction of the accepted events that get correctly identified. The acceptance and efficiency in this analysis are both rather low. The exponential nature of the \pt and \absdz distributions of the slepton decays, many daughter leptons have low \pt and low \absdz and do not pass the signal kinematic selections and are not accepted. This particularly effects low mass, low lifetime sleptons; the \pt cut has an even stronger impact on leptons from stau decays, which must decay through a standard model $\tau$. Conversely, the degradation of the \ac{LRT} efficiency at high \absdz means that leptons with high \absdz are often not reconstructed. This particularly effects sleptons with long lifetimes. Additionally, the \pt and \absdz and $\eta$ cuts required to pass one of the triggers used in this analysis to pass the \ac{LRT} filters further reduces the acceptance and efficiency.

\todo{truth plot of pt and d0}

The acceptance is highest for lifetimes around 0.1 ns, around 30\% for slepton production and only 0.5\% or stau production. The efficiency is higher, around 50\% for lifetimes of order 0.1 ns for both slepton and stau production. The values for a range of possible mass and lifetime points can be see in 

\todo{acceptance and efficiency plots, sec 6.1 of INT}






