%-------------------------------------------------------------------------------
% This file provides template SUSY group object descriptions and cuts.
% \pdfinclusioncopyfonts=1
% This command may be needed in order to get \ell in PDF plots to appear. Found in
% https://tex.stackexchange.com/questions/322010/pdflatex-glyph-undefined-symbols-disappear-from-included-pdf
%-------------------------------------------------------------------------------
% Specify where ATLAS LaTeX style files can be found.
\newcommand*{\ATLASLATEXPATH}{../../latex/}
% Use this variant if the files are in a central location, e.g. $HOME/texmf.
% \newcommand*{\ATLASLATEXPATH}{}
%-------------------------------------------------------------------------------
\documentclass[NOTE, atlasdraft=true, texlive=2016, USenglish]{\ATLASLATEXPATH atlasdoc}
% The language of the document must be set: usually UKenglish or USenglish.
% british and american also work!
% Commonly used options:
%  atlasdraft=true|false This document is an ATLAS draft.
%  texlive=YYYY          Specify TeX Live version (2016 is default).
%  txfonts=true|false    Use txfonts rather than the default newtx
%  paper=a4|letter       Set paper size to A4 (default) or letter.

%-------------------------------------------------------------------------------
% Extra packages:
\usepackage{\ATLASLATEXPATH atlaspackage}
% Commonly used options:
%  subfigure|subfig|subcaption  to use one of these packages for figures in figures.
%-------------------------------------------------------------------------------
\usepackage{multirow}

% Useful macros
\usepackage[jetetmiss]{\ATLASLATEXPATH atlasphysics}
% See doc/atlas_physics.pdf for a list of the defined symbols.
% Default options are:
%   true:  journal, misc, particle, unit, xref
%   false: BSM, heppparticle, hepprocess, hion, jetetmiss, math, process,
%          other, snippets, texmf
% See the package for details on the options.

% Package for creating list of authors and contributors to the analysis.
\usepackage{\ATLASLATEXPATH atlascontribute}

% Add you own definitions here (file atlas-document-defs.sty).
% \usepackage{atlas-document-defs}

% Paths for figures - do not forget the / at the end of the directory name.
\graphicspath{{\ATLASLATEXPATH ../logos/}{figures/}}

%-------------------------------------------------------------------------------
% Generic document information
%-------------------------------------------------------------------------------

\AtlasTitle{SUSY group text snippets for INT notes}
\AtlasVersion{0.1}
\author{ATLAS SUSY Group}
\AtlasRefCode{SUSY-2018-XX}
\AtlasAbstract{%
  This note contains text snippets and tables that should be included in supporting notes
  from the SUSY group.

  The templates are in American English.
  If wanted, some adaption to British English could be made. 

  This document was generated using version \ATPackageVersion\ of the ATLAS \LaTeX\ package.

  \emph{2019-02-04: This file is a work in progress (WIP) and will probably be updated.
  Backwards incompatible changes may be made as the examples develop.}
}
% Author and title for the PDF file
\hypersetup{pdftitle={ATLAS SUSY supporting note},pdfauthor={ATLAS SUSY group}}

%-------------------------------------------------------------------------------
% Main document
%-------------------------------------------------------------------------------
\begin{document}

\maketitle

\tableofcontents

%-------------------------------------------------------------------------------
% The executive summary template is provided by the group
\section{Executive Summary}

This section, ideally less than two pages, should be placed at the beginning of the supporting note.
It should give a high-level overview of the analysis including (but not limited to):
\begin{itemize}
\item physics target and the general characteristics of the signal;
\item analysis strategy;
\item general characteristics of the control, validation, and signal regions;
\item a background estimation strategy overview;
\item highlights of major or most important points of the analysis;
\item a table or list of all critical tasks and who is responsible for each.
\end{itemize}

This section should include explicit pointers to the items required for the PAR and FAR\@.  For reference, those items are listed below. Please feel free to modify these lists into references to document sections.

For a PAR (ed board request), the SUSY group looks for:
\begin{itemize}
\item A definition of the target scenarios
\item A brief run-down of the signal grids, in particular pointing to any production problems or places where the production has not yet begun
\item A discussion of any non-standard object definitions in the analysis, and any on-going development that might affect the object definition
\item A discussion of the derivations: whether any reprocessing is needed, and whether the required samples have been requested
\item Data/MC comparisons in some inclusive regions (demonstrating the technical ability to use the data from all periods in the team's framework)
\item Any signal region definitions that are available, with some preliminary optimization in place and some description of the optimization procedure
\item Plans for further optimization (e.g.\ optimization for different model space regions, or use of an MVA or multi-bin fit)
\item An outline of the plans to get from here to a paper (plan of work, noting if person power is insufficient for any of the areas), as a part of this summary
\item An explicit list of differences in object definitions from the recommendations of the \href{https://gitlab.cern.ch/atlas-phys-susy-wg/Combinations/readme/wikis/home}{Combination Team}, with justification.
\item The location of the analysis code in GIT and demonstration that you have set up containers (\href{https://recast-docs.web.cern.ch/recast-docs/building_images_on_ci/}{described here}).
\end{itemize}

If you are using SUSYTools, please include the current configuration file in the \href{http://gitlab.cern.ch/atlas-phys-susy-wg/AnalysisSUSYToolsConfigurations}{archive}.

For a FAR, the SUSY group looks for:
\begin{itemize}
\item Definitions of SR, CR, VR, including expected yields with the targeted luminosity for all backgrounds that will be estimated with transfer factors or pure MC\@.
    Include signal contamination in CR and VR\@.
\item Cutflow for the background and for representative signal points.
\item Outline of background estimation strategies, including validation and closure tests for data-driven estimations. Statistical uncertainties for transfer-factor (TF) estimated and data-driven estimated backgrounds.
\item Comparison of data and MC with the relevant dataset in CR and VR\@. Strategy for mitigation of mis-modelling wherever needed with proof of feasibility.
\item Estimate of the detector level systematic uncertainties through propagation of CP recommendations.
\item A clear statement on how all others systematic uncertainties will be evaluated, including theory uncertainties on both backgrounds and signal: all the procedures need to be defined before unblinding.
\item A clear plan for ``discovery regions'' as well as the statistical treatment of the signal regions.
\item Background only fit with pull plot for the nuisance parameters.
\item Estimated exclusion, including depth of exclusion within the normal exclusion contour, with Asimov fit.
\item To-do list for achieving final result and possible bottlenecks (can the rest of the SUSY WG help with anything?)
\item For analyses using machine learning methods, additional diagnostics are required, which are explained \href{https://twiki.cern.ch/twiki/bin/view/AtlasProtected/SusyMachineLearning}{here}.
\item The location in GIT where your code is using containers and the first implementation of workflows towards recast (\href{https://recast-docs.web.cern.ch/recast-docs/workflowauthoring/intro/}{described here}).
\end{itemize}

The background forum recommends the following diagnostic plots:
\begin{itemize}
\item Standard occupancy maps and plots that can reveal detector or non-collision background issues --- plot for CRs, VRs and SRs of MET phi, the leading jet and/or leading lepton eta vs.\ phi 2D map, and the leading jet and/or leading lepton phi distributions
\item Selection efficiencies as a function of mu (VR, SR, and CR): to check how dependent the analysis is on pileup (primarily for MC)
\item Run number and data period dependencies: plot lumi-normalized yields in CRs, VRs and SRs as a function of the run number and data period (data only).
    This is to check for potential temporary issues in the data present only for certain runs, and to reveal potential chunks of data not processed by mistake.
    You should normalize the per-run yield using the lumi as reported from an independent source (not the in-file metadata!),
    e.g.\ simply use \href{https://svnweb.cern.ch/trac/atlasoff/browser/PhysicsAnalysis/SUSYPhys/SUSYTools/trunk/scripts/ilumi2histo.py}{this script in SUSYTools} to build the luminosity-vs-run histogram from your \texttt{iLumiCalc} file.
\item In particular, for Full Run 2 dataset analyses, a plot of data from 2018 vs period, specifically to compare the efficiencies for the period with two dead tile modules to the periods without.
\item Check for missing data: Compare the total number of processed data events and compare to the reference numbers for the combination of GRL and derivation you're running over in \href{https://docs.google.com/spreadsheets/d/1LMioo0nvALkKgoCKRW_ihQThHyVwcOQVRJe4aXiENUs/edit#gid=424865170}{this spreadsheet}. If your total does not match this reference number, feel free to contact BG forum conveners and derivation contacts for help with debugging.
\item Check for duplicated events: several bugs in MC and DAOD production have caused duplicated events to appear in the derivations in the past. This can potentially happen in both MC and data. Please check that there are no duplicated events in your CRs, VRs or SRs. If you notice duplicated events in your derivations, please get in touch with the Background Forum conveners immediately.
\item Comparisons of data and MC in the CRs and VRs, as well as MC in the SRs, for 2015+2016, 2017, and 2018 separately, for key distributions and yields.
\item Debug stream yields in SR and CR\@. The full name of the debug stream is \texttt{debugrec\_hlt}, and derivation datasets can be found for both 2015 and 2016 data with a query like \texttt{rucio ls --short --filter type=container data*\_13TeV.00*.debugrec\_hlt*DAOD\_SUSY1*p2709*}.
\item Pileup reweighting check: plot the nvtx distribution before and after pileup reweighting (data Vs MC). Purpose: check that the pileup reweighting works as intended.
\end{itemize}

Again, if you are using SUSYTools, please include the current configuration file in the \href{http://gitlab.cern.ch/atlas-phys-susy-wg/AnalysisSUSYToolsConfigurations}{archive}.  Please also take care that you have looked into the \href{https://twiki.cern.ch/twiki/bin/viewauth/Atlas/DataPreparationCheckListForPhysicsAnalysis}{items recommended by DataPrep}.

%-------------------------------------------------------------------------------

Please feel free to also include a change log with major updates either before or after the executive summary.

%-------------------------------------------------------------------------------
% The executive summary template is provided by the group
\section{Introduction}
\label{sec:introduction}
%-------------------------------------------------------------------------------

Place a short introduction here.  It is useful to introduce your analysis target signals, place them in context,
and describe any previous analyses that this analysis follows on (particularly if significant pieces are in common).

Please note that in an internal note there is no need for a description of the ATLAS detector,
unless the search uses some unusual or less well-known features of the detector of which reviewers will need a reminder.

%-------------------------------------------------------------------------------
\section{Signal Models}
\label{sec:signals}
%-------------------------------------------------------------------------------

Place the description of your target signal models here, potentially including a description of the ``signal grids'' that you will ultimately use for interpretataion.

%-------------------------------------------------------------------------------
\section{Data and Simulated Event Samples}
\label{sec:samples}
%-------------------------------------------------------------------------------

Place the description of your dataset (including GRL) and Monte Carlo simulated samples here.
Please place any long tables (e.g. lists of used datasets) into appendices.

%-------------------------------------------------------------------------------
\section{Object definition}
\label{sec:objects}
%-------------------------------------------------------------------------------

Please describe your object definition here.
% Object definition recommendations are here: https://twiki.cern.ch/twiki/bin/view/AtlasProtected/SusyObjectDefinitionsr2113TeV

%-------------------------------------------------------------------------------
\section{Event selection}
\label{sec:selection}
%-------------------------------------------------------------------------------

Place the description of your event selection here.  This can include signal region optimization, control region selection, and pre-selection.

%-------------------------------------------------------------------------------
\section{Background Estimation}
\label{sec:backgrounds}
%-------------------------------------------------------------------------------

Place the description of your background estimation here.

%-------------------------------------------------------------------------------
\section{Systematic Uncertainties}
\label{sec:systematics}
%-------------------------------------------------------------------------------

Place the description of your systematic uncertainties here.
% Useful recommendations are maintained by the SUSY group: https://twiki.cern.ch/twiki/bin/view/AtlasProtected/SUSYSpecificRecommendations
% and by the PMG: https://twiki.cern.ch/twiki/bin/view/AtlasProtected/PhysicsModellingGroup#Systematic_uncertainties_and_rew

%-------------------------------------------------------------------------------
\section{Results}
\label{sec:result}
%-------------------------------------------------------------------------------

Place your results here.

% The required information is available here: https://twiki.cern.ch/twiki/bin/viewauth/AtlasProtected/SUSYResultsPresentation
% All figures and tables should appear before the summary and conclusion.
% The package placeins provides the macro \FloatBarrier to achieve this.
% \FloatBarrier


%-------------------------------------------------------------------------------
\section{Conclusion}
\label{sec:conclusion}
%-------------------------------------------------------------------------------

Place your conclusion here.


%-------------------------------------------------------------------------------
% If you use biblatex and either biber or bibtex to process the bibliography
% just say \printbibliography here
\printbibliography
% If you want to use the traditional BibTeX you need to use the syntax below.
% \bibliographystyle{obsolete/bst/atlasBibStyleWithTitle}
% \bibliography{atlas-document,bib/ATLAS,bib/CMS,bib/ConfNotes,bib/PubNotes}
%-------------------------------------------------------------------------------

%-------------------------------------------------------------------------------
% Print the list of contributors to the analysis
% The argument gives the fraction of the text width used for the names
%-------------------------------------------------------------------------------
\clearpage
The supporting notes for the analysis should also contain a list of contributors.
This information should usually be included in \texttt{mydocument-metadata.tex}.
The list should be printed either here or before the Table of Contents.
\PrintAtlasContribute{0.30}


%-------------------------------------------------------------------------------
\clearpage
\appendix
\part*{Appendices}
\addcontentsline{toc}{part}{Appendices}
%-------------------------------------------------------------------------------

In an ATLAS note, use the appendices to include all the technical details of your work
that are relevant for the ATLAS Collaboration only (e.g.\ dataset details, software release used).
This information should be printed after the Bibliography.

\end{document}
