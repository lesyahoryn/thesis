% ---- ETD Document Class and Useful Packages ---- %
\documentclass{ucetd}
%$\usepackage{subfigure}
\usepackage{epsfig}
\usepackage{amsfonts}
\usepackage{amsmath}
\usepackage{amssymb}
\usepackage{amsthm}
\usepackage[printonlyused]{acronym}
\usepackage{atlaslatex/latex/atlasmisc}
\usepackage{atlaslatex/latex/atlasparticle}
\usepackage{atlaslatex/latex/atlasunit}
\usepackage{xspace}
\usepackage{xcolor}
\usepackage{rotating}
\usepackage{placeins}
%\usepackage{caption}
\usepackage{subcaption}
\usepackage[hidelinks]{hyperref}
\usepackage{multirow}
\usepackage{setspace}
\usepackage{url}
\usepackage[backend=bibtex,sorting=none]{biblatex}
\usepackage[nottoc,numbib]{tocbibind}
\setcounter{tocdepth}{3}
\renewcommand{\etdChapterHeadFormat}[1]{\uppercase{#1}}
\renewcommand{\chapterautorefname}{Chapter}
\newcommand{\todo}[2][TODO]{{\color{red}\em[{\textbf\em {#1}: }{#2}]}}
\newcommand{\question}[2][Question]{{\color{blue}\em[{\textbf\em {#1}: }{#2}]}}
\newcommand{\comebackto}[2][come back to this?]{{\color{magenta}\em[{\textbf\em {#1}: }{#2}]}}
\usepackage{etoolbox}
\patchcmd{\part}{\thispagestyle{plain}}{\thispagestyle{empty}}{}{}
\AtBeginEnvironment{thebibliography}{\linespread{1}\selectfont}


%% Use these commands to set biographic information for the title page:
%for uchicago
\title{A Search for Displaced Leptons in the ATLAS Detector}
\author{Lesya Horyn}
%for uchicago
\department{Physics}
\division{Physical Sciences}
\degree{Doctor of Philosophy}
\date{December 2020}


%% Use these commands to set a dedication and epigraph text
\dedication{To my father, who taught me to look to the stars, and to my mother, who taught me to look in a book.}
\epigraph{Epigraph Text}

\addbibresource{Chapters/horyn_thesis_chicago.bib} 


\begin{document}
%% Basic setup commands
% If you don't want a title page comment out the next line and uncomment the line after it:
\maketitle
%\omittitle

% These lines can be commented out to disable the copyright/dedication/epigraph pages
\makecopyright

\clearpage
\begin{center}
    \thispagestyle{empty}
    \vspace*{\fill}
    For my father, who taught me to look to the stars, 

    and for my mother, who taught me to look in a book.
    \vspace*{\fill}
\end{center}
\clearpage



%% Make the various tables of contents

\tableofcontents
\listoffigures
\listoftables
%----------------------------------------------------------------------------------------
%	Acronyms
%----------------------------------------------------------------------------------------

%\refstepcounter{dummy}
%\addcontentsline{toc}{chapter}{Acronyms} % Uncomment if you would like the acronyms to appear in the table of contents
%\pdfbookmark[1]{Acronyms}{acronyms} % Bookmark name visible in a PDF viewer

%\markboth{\spacedlowsmallcaps{Acronyms}}{\spacedlowsmallcaps{Acronyms}}

\chapter*{Acronyms}

\begin{acronym}[UML]
% detector acronyms
\acro{IBL}{Insertable B-Layer}
\acro{MS}{Muon Spectrometer}
\acro{ID}{Inner Detector}
\acro{SCT}{Silicon Microstrip Tracker}
\acro{TRT}{Transition Radiation Tracker}
\acro{ToT}{Time Over Threshold}
\acro{LAr}{Liquid Argon Calorimeter}
\acro{FCAL}{Forward Calorimeter}
\acro{MDT}{Monitored Drift Tube}
\acro{CSC}{Cathode-Strip Chamber}
\acro{RPC}{Resistive Plate Chamber}
\acro{TGC}{Thin Gap Chamber}
\acro{L1}{Level One}
\acro{HLT}{High Level Trigger}
\acro{L1Calo}{L1 Calorimeter Trigger}
\acro{L1Muon}{L1 Muon Trigger}
\acro{CTP}{Central Trigger Processor}
\acro{TTC}{Trigger Timing and Control}
\acro{ROB}{Read Out Board}
\acro{RoI}{Region of Interest}

%lhc acronyms
\acro{CERN}{European Center for Nuclear Research}
\acro{LHC}{Large Hadron Collider}
\acro{LEP}{Large Electron-Positron Collider}
\acro{LINAC}{Linear Accelerator}
\acro{PSB}{Proton Synchrotron Booster}
\acro{PS}{Proton Synchrotron}
\acro{SPS}{Super Proton Synchrotron}
\acro{ATLAS}{A Toroidal LHC Apparatus}
\acro{CMS}{Compact Muon Solenoid}
\acro{ALICE}{A Large Ion Collider Experiment}
\acro{LHCb}{Large Hadron Collider beauty}
\acro{RF}{Radiofrequency}
\acro{PSB}{Proton Synchrotron Booster}
\acro{PS}{Proton Synchrotron}
\acro{RMS}{root mean square}
\acro{HF}{Heavy Flavor}

% reco acronyms
\acro{OR}{Overlap Removal}
\acro{PV}{Primary Vertex}
\acro{PVs}[PVs]{Primary Vertices}
\acro{EM}{Electromagnetic}
\acro{CB}{Combined}
\acro{LRT}{Large Radius Tracking}
\acro{ST}{Standard Tracking}
\acro{GSF}{Gaussian-sum Filter}
\acro{IP}{Interaction Point}


% theory acronyms
\acro{MC}{Monte Carlo simulation}
\acro{SM}{Standard Model}
\acro{BSM}{Beyond the Standard Model}
\acro{QCD}{Quantum Chromodynamics}
\acro{PDF}{Parton Distribution Function}
\acro{DM}{Dark Matter}
\acro{LO}{Leading Order}
\acro{NLO}{Next to Leading Order}
\acro{NLO+NLL}{Next-to-Leading-Logarithmic Accuracy}
\acro{SUSY}{Supersymmetry}
\acro{MSSM}{Minimal Supersymmetric Standard Model}
\acro{LSP}{Lightest Supersymmetric Particle}
\acro{NLSP}{Next Lightest Supersymmetric Particle}
\acro{CP}{charge parity}
\acro{GMSB}{Gauge Mediated Supersymmetry Breaking}
\acro{RPV}{R-parity violating}


% analysis acronyms
\acro{LLP}{Long Lived Particle}
\acro{DV}{Displaced Vertex}
\acro{MET}{Missing Transverse Energy}

\acro{AOD}{Analysis Object Data}
\acro{dAOD}{derived AOD}
\acro{SR}{Signal Region}
\acro{VR}{Validation Region}
\acro{CR}{Control Region}
\acro{FS}{Flavor Symmetric}
\acro{CL}{Confidence Level}
\acro{HL-LHC}{High Luminosity Large Hadron Collider}
\end{acronym} 
\addcontentsline{toc}{chapter}{List of Acronyms}


\acknowledgments
I think it's safe to say that nothing about this PhD has gone the way I expected. I didn't expect to write this dissertation during a pandemic, and I didn't expect to feel so content at the end of this time. I came to the University of Chicago with a lot of self doubt and not a lot of certainty, but I leave feeling like a scientist. For that feeling, I have to thank everyone who has guided and supported me over the last six years.

I've had the immense privilege of having some incredible mentors.
Young-Kee, thank you for pushing me and always making sure I had the support I needed. I've learned so much from your perspective on the world and on physics. 
Tova, I can only hope that I can be as great of a postdoc to someone as you have been to me. Working with you has really shaped my identity as a scientist. Thank you for always treating me with respect. I will really miss working with you.
Max, I truly wouldn't have made it through my first year without you. You've been such a generous mentor, completely independent of how closely we were working together at the time. Thank you for always taking the time to answer all my questions. 

I've also learned that working with good people makes all the difference. The displaced leptons analysis was a small and scrappy team who all taught me a lot. Thank you to Elodie, for pushing this through until the very end, to Xiaohe for being such a wonderful undergrad to work with, and to Kate for being the most helpful at the most important time. 
Though not the focus of this thesis, I spent four years working on FTK with a truly incredible team. Thanks to Mel, Patrick, and Rui, for your guidance and opinions. And to Todd, my favorite physics younger brother, somehow we made it through together and I'm really grateful for your friendship. 
Liza, I've learned so much from you about problem solving, care, thoughtful work, and of course, team building. Also to Tyler, Stany, Bri, Jon, and the rest of the team that made a stressful project a lot of fun.

I've been extremely fortunate to have friends that always made me feel like I was at home, no matter where in the world I was. I'll always cherish my first year in Chicago, I was very lucky to have a wonderful cohort full of smart, caring people, especially Gautam, Mark, and Dani. But most especially Lipi, the best long distance friend and spontaneous roommate turned family. 
Thank you to my CERN lunch crew turned pandemic survival team, Max, Tova, Kate, Larry, Liza, Karri, Ann, Emma, and Chris. I'm so grateful for your friendship and have learned so much from your many many opinions.
A very important thank you to Julia and Audrey, how could any of us have survived this alone. 

And of course thank you to Simone. You've brought me immeasurable joy and comfort. Thank you for never wavering in your support, for always being patient with me, and for all of the adventures. I can't wait to see what's next.

Most importantly, thank you to my family. Without your support this never would have been possible. Sofia, thank you for always being on my side. Mama and Tato, there are no words to express how grateful I am. You taught me to think for myself and to ask questions, and you always made sure that the world was open before me.  For all of my family, these last few years have been hard in many ways, but there's no feeling quite like going home and being surrounded by your love.
 




\abstract
A search for long-lived particles decaying into displaced electrons and/or muons with large impact parameters is presented. This signature provides unique sensitivity to the production of theoretical supersymmetric lepton-partners, sleptons, with lifetimes between 0.01 and 10 ns. This search is done for the first time at the Large Hadron Collider (LHC), and covers a long-standing gap in coverage of possible new physics signatures. The search is performed using 139 fb$^{-1}$ of $\sqrt{s} = 13$ TeV proton-proton collision data collected by the ATLAS detector at the LHC between 2015 and 2018. Special reconstruction and identification algorithms are used to select leptons with large impact parameters. Backgrounds are estimated from data. Results are consistent with background, and so limits on slepton masses and lifetimes in this model are calculated at 95\% CL. For 0.1 ns lifetimes selectron, smuon and stau masses up to 720 GeV, 680 GeV, and 340 GeV are excluded, respectively, drastically improving on the previous best limits from LEP. 


\mainmatter
% Main body of text follows


%----------------------------------------------------------------------------------------
%	THESIS CONTENT - CHAPTERS
%----------------------------------------------------------------------------------------

%----------------------------------------------------------------------------------------
\part{Introduction and Motivation}
\chapter{Introduction}

Particle physics seeks to understand the fundamental structure of the universe by defining the minimal set of particles and interactions required to describe all physical phenomena. The \acf{SM} is the best attempt at such a description. The \ac{SM} has undergone decades of rigorous testing and can explain almost all phenomena we see in experiments. Yet it is known that the \ac{SM} is missing explanations for crucial physical phenomena like quantum description of gravity or a dark matter candidate. As a result, many \acf{BSM} theories have been developed and tested, hoping to extend and complete the picture the \ac{SM} gives. So far, no evidence for any of these theories has been seen.  

The \acf{LHC} at \acf{CERN}, a 27 km particle collider outside of Geneva, Switzerland, is the largest particle physics experiment in the world, and provides an extremely effective environment to test the \ac{SM} and a wide variety of \ac{BSM} theories. This thesis uses data from the \acf{ATLAS} experiment, one of the four largest experiments along the \ac{LHC} ring.

Beams of protons circulate and collide in the \ac{LHC}, and if two protons collide with sufficiently high energy, massive particles can be created; $\sqrt{s} = 8 \TeV$ collisions enabled the 2012 discovery of the Higgs boson with mass of 125 \GeV. The \ac{LHC} provides 60 million collisions per second, enabling physicists to search for new and rare physical phenomena. Unfortunately, after 8 years of data taking no evidence of \ac{BSM} physics has been found. Data taking is scheduled to resume in 2022 with only a moderate increase in collision energy and about a factor 2 more data. 

This is a call to expand the suite of \ac{BSM} searches by re-examining the assumptions made in searches that have been performed so far. What could we have missed in our search for new physics at the TeV scale? The \ac{LHC} detectors are designed to look the decays of short-lived, heavy particles with the assumption that the decay products will be \emph{prompt} and trace back to the collision point. This misses a large range of intermediate lifetimes of possible \ac{BSM} particles that decay inside of the detector material.  These signatures are challenging, but not impossible, to identify with \ac{ATLAS} as they result in \emph{displaced} \ac{SM} particles that do not point back to the collision point. Many \ac{SM} particles are long lived, like muons or neutrons, and many \ac{BSM} theories predict particles with lifetimes that result in displaced decays. This thesis presents a search for one such signature.

This thesis presents a search for two displaced \ac{SM} leptons, either electrons or muons, that are not connected by a displaced vertex. Due to the displacement and lack of vertex, a \ac{BSM} particle decaying to this signature would be vetoed by all other analyses at the \ac{LHC}, even those targetting \acp{LLP}. This search has unique sensitivity to a specific \acf{GMSB} \acf{SUSY} model where the \acf{LSP} is the superpartner to the graviton, the gravitino, and the \acf{NLSP} is a slepton (\slep), the superpartner to a lepton. The \slep is long lived because it must decay to the gravitino through the very weak gravitational coupling. The last time this model was explored was in the OPAL experiment at \acf{LEP} \cite{opal}, where masses up to about 90 GeV were probed for the full range of lifetimes, immediately decaying to detector stable and all possible signatures in between. This search probes almost an order of magnitude of mass phase space in a limited lifetime phase space, and in a significantly more challenging environment than \ac{LEP}.

Since this is the first search for displaced leptons at the \ac{LHC} particle selection algorithms and robust data-driven background estimates needed to be developed. This search for displaced leptons uses $136 \ifb$ of data collected by \ac{ATLAS} during Run 2 of the \ac{LHC}. Major backgrounds come from fakes of the reconstruction algorithms and muons from cosmic rays. Less than 1 background events are predicted and zero events are seen, and so limits on the mass and lifetime of \slep. 

This thesis is organized into three main sections: first the search is motivated theoretically, then the experimental setup is described, and finally the search strategy and its results are presented.

\autoref{chap:theory} provides theoretical motivation for searches for \ac{SUSY} and in particular long lived and \ac{GMSB} \ac{SUSY}.

\autoref{chap:LHC} describes the \ac{LHC} and its design and operation.

\autoref{chap:ATLAS} describes the \ac{ATLAS} subdetectors and how they are used to measure particles.

\autoref{chap:eventreco} details particle reconstruction algorithms and the modifications made for this analysis.

\autoref{chap:datamc} describes the \ac{ATLAS} data and \acf{MC} simulation of the signal model.

\autoref{chap:llps} provides context for \ac{LLP} decays and searches.

\autoref{chap:context} provides experimental context for the signature and model.

\autoref{chap:eventselection} describes the analysis strategy, lepton selection requirements and final event selection criteria.

\autoref{chap:backgrounds} details the backgrounds to this signature and the algorithms for estimating them.

\autoref{chap:systematics} describes the uncertainties applied to signal \ac{MC} in order to use the data to make a statement about the \ac{SUSY} model.

\autoref{chap:results} presents the unblinded results as well as a description of the statistical interpretation. 

\autoref{chap:conclusions} provides reflections on the analysis as well as possible improvements for future \ac{ATLAS} analyses for displaced leptons.



\chapter{Theory}

\section{The Standard Model}
\section{Open Questions}
\section{Supersymmetry}


\cleardoublepage 

%----------------------------------------------------------------------------------------
\part{The Experiment}

\chapter{The Large Hadron Collider}

The world's largest machine is required to study the universe's smallest particles. The \ac{LHC} is circular particle collider located on the French-Swiss border outside of Geneva, Switzerland. A series of accelerators culminate in a $27$-km ring in which beams of hadrons, either protons or heavier ions, are collided. The \ac{LHC}, the most powerful hadron accelerator ever built, has been running in its current form since 2008. It is built in the tunnel previously used by \ac{LEP}, the most powerful lepton accelerator ever built, which was used to precisely measure the mass of the W and Z bosons. Collider experiments are based around the idea that $E=mc^2$: the more energetic the reaction, the more massive the particle that can be produced by it, allowing scientists to study physical phenomena not accessible in other experiments. 

There are four major experiments around the \ac{LHC} ring: \ac{ATLAS} and \ac{CMS}, general purpose detectors designed independently in order to serve as a cross-check for each other; \ac{ALICE}, a tracking-focused detector designed to study collisions of heavy ions; and \ac{LHCb}, an asymmetrical detector designed to study \ac{CP} violation. 

The dataset used in this analysis comes from proton-proton ($pp$) collisions during Run 2 of the \ac{LHC}, which spanned from 2015-2018 and had a center of mass energy of $\sqrt{s} = 13 \TeV$.

\section{Structure of the LHC}

\subsection{Acceleration}
%USED https://home.cern/science/engineering/accelerating-radiofrequency-cavities
The basic units of an accelerator are \ac{RF} cavities, used to accelerate particles, and magnets, used to focus and bend the beam. A \ac{RF} cavity is a metallic structure with an electric field that oscillates at a specified frequency. Positively charged particles are repelled from the positive field and pulled by the positive field; in the \ac{LHC}, this accelerates protons from $450 \GeV$ to $6.5 \GeV$. Once the particle reaches the required energy, it will not be further accelerated. If the proton does not have the requisite energy and arrives early or late with respect to the oscillation, it will be accelerated or decelerated by the cavity. This procedure creates a beam made of \emph{bunches} of protons, the size and spacing of which is dictated by the oscillation frequency of the accelerating cavities (at the \ac{LHC} proton bunches are $25$ ns apart)

%USED https://home.cern/science/engineering/pulling-together-superconducting-electromagnets
%USED LHC-magnets.pdf
Strong electromagnets are used to focus and bend the beam into its circular path. Superconducting \footnote{If the magnets were not superconducting, the \ac{LHC} would need to be an order of magnitude larger, $127 \textrm{km}$ in order to reach the same energy}, $14$ m long, $8.4$ T dipole magnets are required to bend the beam into its circular shape. Additional sextupole, octupole and decapole magnets accompany the dipole magnets to  correct for edge effects at their extremities. Quadrupole magnets are used to \emph{focus} the beams to squeeze them horizontally and vertically at the collision points. Careful control of the beam is extremely important for the safety of the \ac{LHC} and the detectors. 


\subsection{A Circular Proton-Proton Collider}

The history of particle physics is rich with different kinds of accelerators: linear and circular; colliding electrons, protons, and anti-protons. The \ac{LHC} is a circular, proton-proton collider -- a deliberate choice due to its physics goals. These considerations are an important part of the ongoing conversation about how the next generation of particle physics experiments should be designed. 

%USED https://home.cern/science/accelerators
First, accelerators can be linear or circular. In a linear accelerator, a particle is propelled from one end of the beam pipe to the other, increasing its energy as it goes, and eventually collides with a target or beam from another linear accelerator (or is used to feed another accelerator as in the \ac{LHC}). This means that each particle can only interact with the accelerating field once and then the particles must interact or be dumped. However, in a circular accelerator, particles make many revolutions around the acceleration path, so greater energies can be achieved in the same accelerating distance. In a circular collider, two beams of particles revolve in opposite directions and are focused at specified \ac{IP}. Particles that do not collide the first time the beams cross can be recycled and be made to collide again so collisions can happen continuously for hours. Circular electron colliders suffer from large energy loss due to synchrotron radiation due to the small mass of the electron, so the energies that can be achieved are limited (the \ac{LHC} collides protons at $\sqrt{s} = 13 \TeV$ while \ac{LEP} collided electrons and positrons at $\sqrt{s} = 209 \GeV$ in the same tunnel). In all, linear accelerators are simpler to create, but are less efficient accelerators than their circular counterparts. 

Secondly, the \ac{LHC} collides protons. Electrons are fundamental particles which create a well understood and clean collision. A collision occurs when an electron and positron annihilate. The energy produced is set by the energy of the beams and there is no ambiguity in the production mechanism of the physical process observed in the detector. Protons, however, are not fundamental particles and when two protons collide, it is their constituent quarks and gluons that interact. This means that not all of the energy of the proton is involved in the primary interaction, nor is the production mechanism known. The fraction of the proton's energy carried by the quarks and gluons is not a known quantity, it must be measured as a \ac{PDF}, which gives the probability of finding a given constituent with a given momentum. Uncertainties in \ac{PDF} measurements add to the uncertainties of the final result. This can be seen as an advantage or disadvantage: electron-positron collisions happen at a fixed energy, so they are very powerful for performing precision measurements, for example of the W and Z boson masses, where one would like many clean events where W or Z bosons were produced. In a proton-proton machine, however, a range of energy is available at each collision, making them ideal discovery environments, where one would like to look for a new particle at a wide range of masses, like in the case of the \ac{LHC} searching for the Higgs boson and \ac{BSM} phenomena. 

Additionally, the other quarks and gluons in the proton can interact as well, creating a messier collision environment compared to the electron-positron collision. Furthermore, in order to collide protons, the two beams must be separated in order for both beams to be accelerated. This is not true if one collides proton and anti-protons -- since they are oppositely charged, the same field can be used to accelerate protons in one direction and anti-protons in the other. Proton-anti-proton collisions feature more production mechanisms, but anti-protons are much harder to make and keep than protons. 







\subsection{Injection Chain}


\todo{steps of injecting}

\subsection{Using the LHC for Particle Physics}

%USED pdg-review.pdf
If the primary goal of the collider is to study rare physics processes, whether difficult measurements of the \ac{SM} or heretofore unseen \ac{BSM} physics, enough data must be produced to be able to study them. The number of events of a given process in a given data set is given by

\begin{equation}
N_{\textrm{events}} = \mathcal{L}_{\textrm{int}} \times \sigma_{\textrm{process}}
\label{eq:nevents_lumi}
\end{equation}
where $\mathcal{L}$ is the \emph{integrated luminosity} of the dataset, and $\sigma_{\textrm{process}}$ is the cross section for the given process. More data ($\mathcal{L}_{\textrm{int}}$) is required to be able to see rare events (small $\sigma_{\textrm{process}}$), and many events are required in order to have the statistical power in able to make a discovery. The integrated luminosity is the integral of the instantaneous luminosity over data taking time

\begin{equation}
\mathcal{L}_{\textrm{int}} = \int \mathcal{L} dt
\end{equation}
and the \emph{instantaneous luminosity} is related to the parameters of the accelerator. For two identical bunches with $N_1$ and $N_2$ protons per bunch colliding with frequency $f$:

\begin{equation}
\mathcal{L} = \frac{N_1 N_2 f}{4\pi \sigma_x^* \sigma_y^*} \mathcal{F}
\end{equation}
where $\sigma_x^*$ and  $\sigma_y^*$ are the \ac{rms} of the beam width in the $x$ and $y$ directions, and $\mathcal{F}$ is a factor that takes in other geometric effects such as the crossing angle and bunch length, it is generally $\mathcal{O}(1)$. It can be rewritten more specifically for the \ac{LHC} as

\begin{equation}
\mathcal{L} = \frac{N_b^2 n_b f }{4\pi \sqrt{\epsilon_n \beta^*_x \beta^*_y}}\mathcal{F}
\end{equation}
where $N_b$ is the number of protons per bunch (assuming $N_1 = N_2$), $n_b$ is the number of bunches in the beam, $\epsilon_n$ is the \emph{emittance}, which describes the spread of the particles in the bunch, and $\beta^*$ is the value of the $\beta$-function at the \ac{IP}. The $\beta$-function describes the size of the beam as a function of location. A sampling of the values of these parameters in 2018 are shown in \autoref{tab:lumi-vals}.


%USED ATLAS-lumi-measurement.pdf
% https://indico.cern.ch/event/751857/contributions/3259373/attachments/1783143/2910577/belen-Evian2019.pdf
\begin{table}
\centering
\begin{tabular}{lc}
\hline
Paramter & value  \\
\hline
$N_b$ (protons per bunch)                                           & $1.1 \times 10^{11}$   \\
$n_b$ (bunches per beam)                                            & $2544$   \\
$\beta^*$ (beam size)                                               & $.3$ m   \\
$\epsilon_n$ (beam spread)                                          & $1.8-2.2 \mu$m-radians   \\
$\mathcal{L}_{\textrm{peak}}$ (peak instantaneous luminosity)       & $21 \times 10^{33} \textrm{cm}^{-2}\textrm{s}^{-1}$   \\
space between bunches                                               & $25$ ns   \\
\hline
\end{tabular}
\caption{Beam parameters for 2018 for standard running conditions used in the data collection for this analysis. Special runs take place where these parameters are changed.}
\label{tab:lumi-vals}
\end{table}

Several of these factors can be manipulated \emph{in situ}, allowing for luminosity increasing or decreasing, depending on the need. Decreasing $\beta^*$ increases the luminosity, so the beams are \emph{squeezed} with focusing quadrupole magnets at the \ac{IP}. Nonzero beam crossing angles and longer bunches decrease luminosity. In fact, during some runs of Run 2 of the \ac{LHC}, the beam angle was changed at the beginning in order to decrease the luminosity to a level more tolerable by the experiments, referred to as \emph{leveling}. 

%The instantaneous luminosity can be measured by measuring the components directly, or it can be inferred by measuring the number of events in a process with a well measured cross section, $\sigma_{\textrm{ref}}$ that produces $N_{\textrm{ref}}$ events. By counting $N_{\textrm{exp}}$, the actual number of events produced, one can infer $\sigma_{\textrm{exp}}$. The difference between $\sigma_{\textrm{ref}}$ and $\sigma_{\textrm{exp}}$ gives information about the instantaneous luminosity from an equation similar to \autoref{eq:nevents_lumi}. 


%USED LHC-vandermeer.pdf
%USED ALTAS-lumi-measurement.pdf
In \ac{ATLAS}, luminosity is measured in two steps. First, the rate of $pp$ collisions ($\mu_{\textrm{vis}}$) is measured using detectors close to the beam pipe. This is done both online, so that adjustments to the data collection scheme can be done on the fly, as well as offline, to determine the amount of data collected. Second, the $pp$ collision rate is translated into a luminosity using a \emph{van der Meer} scan. During the scan, the beam is widened and, initially, the beams are separated in both $x$ and $y$. The beams are moved incrementally closer to each other by known amounts such that the separation between the beams is always known as the number of collisions increases. The luminosity per bunch can be expressed as

\begin{equation}
\mathcal{L}_b = \frac{\mu_{\textrm{vis}}}{\sigma_{\textrm{vis}}} f
\end{equation}
where $\mu_{\textrm{vis}}$ is measured by the luminosity detectors. The luminosity, $\mathcal{L}_b$, is known during the van der Meer scan, and so the inelastic cross section, $\sigma_{\textrm{vis}}$ can be determined and used to calculate $\mathcal{L}_b$ from $\mu_{\textrm{vis}}$. Both $\mu_{\textrm{vis}}$ and $\sigma_{\textrm{vis}}$ contain detection efficiency effects. 

%USED luminosity public results
In Run 2, the \ac{LHC} produced an integrated luminosity of $156 \textrm{fb}^{-1}$, \ac{ATLAS} recorded $147 \textrm{fb}^{-1}$, with $139 \textrm{fb}^{-1}$ available to be used for physics analyses.



\subsection{Pileup}

While high instantaneous luminosity enables the fast accumulation of data required to study rare physical processes, it also creates a dense environment in which those processes occur. The number of concurrent $pp$ collisions is called \emph{pileup}; it is substantial because the instantaneous luminosity is greater than the $pp$ inelastic scattering cross section. The average number of interactions per bunch crossing is not a static number, but the average is $33$ for all of Run 2, as seen in \autoref{fig:pileup_plot}

\begin{figure}[htbp]
\centering
\includegraphics[width=.8\textwidth]{figures/Detector/lhc-mu.pdf}
\caption{Average number of interactions per bunch crossing during Run 2. }
\label{fig:pileup_plot}
\end{figure}

In practice, pileup collisions are generally lower energy, \ac{QCD}-only interactions, producing sprays of low energy activity in the \ac{ID} and calorimeters, which can create fake tracks, clusters, or add energy to non-pileup detector signatures.  Pileup can be mitigated by reconstructing all of the vertices in the event and removing detector signatures associated to non-primary vertices. An example of an event with 25 simultaneous collisions can be see in \autoref{fig:pileup_eventdisplay}. The tracks from the target event are the yellow lines, however there are many other tracks to confuse the situation.


\begin{figure}[htbp]
\centering
\includegraphics[width=.8\textwidth]{figures/Detector/lhc-pileup-eventdisplay.png}
\caption{An event display with 25 simultaneous interactions. The primary interaction is in yellow, but the pileup-dense environment makes it hard to see that event among all of the other activity.}
\label{fig:pileup_eventdisplay}
\end{figure}





\chapter{The ATLAS Detector}

The \ac{ATLAS} detector is a cylindrical general purpose particle detector designed to measure the products of $\sqrt{s} = 14 \TeV$ proton-proton collisions at the \ac{LHC}. It consists of three major sub-detectors: closest to the beamline is the the \ac{ID}, which measures the trajectories of charge particles, followed by the Calorimeters, which measure the energies of electromagnetic and hadronically interacting particles, and finally the \ac{MS} which measures the trajectories of muons. The \ac{ID} is surrounded by a super conducting solenoidal magnet that provides a uniform $2\textrm{T}$ magnetic field, enabling measurement of particles' charge and momentum, and a toroidal magnet surrounds \ac{MS}, allowing for charge and momentum measurements of muons. In general, each subdetector consists of a barrel detector parallel to the beampipe and end-cap detectors perpendicular to the beampipe.

A schematic of the \ac{ATLAS} detector is shown in \autoref{fig:atlas-schematic}.


%https://atlas.cern/discover/detector
\begin{figure}[htbp]
\centering
\includegraphics[width=.8\textwidth]{figures/Detector/atlas-schematic.jpg}
\caption{A diagram of the \ac{ATLAS} detector. The dimensions, subdetectors, and magnet systems are labeled. }
\label{fig:atlas-schematic}
\end{figure}

\section{Coordinate System}
\ac{ATLAS} uses a Cartesian right-handed coordinate system, with the origin defined as the $pp$ collision point. The $z$-axis points along the beampipe, where $+z$ points counter-clockwise. The transverse plane, the $y$-axis and $x$-axis, points upward and toward the center of the \ac{LHC} ring, respectively. The detector is built with with symmetry across the origin in in $z$, as well as with rotational symmetry in the transverse plane. The $+z$ side of the detector is referred to as the A-side, and $-z$ as the C-side.

Cylindrical coordinates provide a comfortable description of the \ac{ATLAS} detector, where $\phi$ measures the angle in the $x-y$ plane around the beampipe, and $\theta$ the angle from the $z$ axis. $\phi$ is positive for positive $y$. 

A given particle's momentum in $z$ is not known, but its transverse momentum is known to be $0$, so it is advantageous to define spatial variables independent of $z$ momentum. Thus, instead of $\theta$, $\eta = - \textrm{ln}(\textrm{tan}\frac{\theta}{2})$ is used to describe angle from the $z$ axis. Particles perpendicular to the $z$ axis have $\eta = 0$, while those parallel to the beamline have $\eta \rightarrow \infty$. 

Angular distances between objects is described using $\Delta R = \sqrt{\Delta \eta ^2 + \Delta \phi ^2}$ and the radial distance from the origin in the $x-y$ plane is denoted $R$. 

A particle's momentum will generally be described in terms of its \pT, its momentum in the transverse direction. A particle's $3$-vector is described by $(\pt, \eta, \phi)$, which are all invariant under boosts in $z$ assuming the particle can be considered massless (which is true in the case of particles in \ac{ATLAS}).





%USED ATLAS-overview.pdf
\section{Inner Detector}
The Inner Detector measures the trajectories of charged particles resulting from \ac{LHC} collisions. The \ac{ID} covers the region with $|\eta| < 2.5$, measuring approximately $1000$ particles per bunch crossing. In order to achieve the momentum and vertex resolution required to achieve \ac{ATLAS}'s physics goals three subdetectors are used: the Pixel detector, the \ac{SCT}, and the \ac{TRT}. The Pixel and \ac{SCT} detectors are used for high granularity precision tracking and the \ac{TRT} is used to distinguish electrons from converted photons. All of this is immersed in a $2T$ magnetic field, curving charged particles in proportion to its momentum.

\todo{describe conversions somewhere... Maybe in particle descriptions in theory?}


The \pt resolution of the \ac{ID} scales with track \pt. Higher \pt tracks are less curved, so the measurement resolution is worse. In the \ac{ATLAS} \ac{ID}, the \pt resolution  $0.05\% \times \pt$ with a $1\%$ constant term. The constant term describes measurement uncertainties that do not scale with momentum or energy, such as material imperfections, non-uniform detector response, or other constant measurement issues and is added in quadrature ($\oplus$) with the stochastic term.


A schematic of the \ac{ID} can be seen in \autoref{fig:atlas-id} and a detailed distribution of the various subdetectors is shown in \autoref{fig:atlas-id-layers}. 


%https://cds.cern.ch/images/CERN-GE-0803014-01/file?size=medium
\begin{figure}[htbp]
\centering
\includegraphics[width=.8\textwidth]{figures/Detector/atlas-ID.jpg}
\caption{A diagram of the \ac{ATLAS} \ac{ID} with the major subsystems labeled. The Pixel and \ac{SCT} are of particular importance for this analysis.}
\label{fig:atlas-id}
\end{figure}

%https://www.researchgate.net/publication/325643426/figure/fig8/AS:669532737769482@1536640452588/Segment-of-the-ATLAS-inner-detector-showing-the-tracker-layers-The-silicon-strip.ppm
\begin{figure}[htbp]
\centering
\includegraphics[width=.8\textwidth]{figures/Detector/atlas-id-layers.png}
\caption{A schematic of the \ac{ATLAS} \ac{ID} shown in the $r-z$ plane. }
\label{fig:atlas-id-layers}
\end{figure}

\subsection{The Pixel Detector}
\subsubsection{Hit Reconstruction}
\subsection{The Silicon Microstrip Tracker}
\subsubsection{Hit Reconstruction}
\subsection{The Transition Radiation Tracker}

\subsection{Solenoid Magnet}

The central solenoid surrounds the \ac{ID} and provides a uniform $2\textrm{T}$ field that bends the trajectories of charged particles. The transverse momentum of the particle can be inferred from its radius of curvature, $R$, in the $x-y$ plane using the equation $p_{T} = qBR$, where $q$ is the charge of the particle and $B$ the magnetic field in the $z$ direction.

However, the placement of the solenoid between the \ac{ID} and calorimeters necessitates careful design choices so that all of a given particle's energy is still measured by the calorimeters. The solenoid only contributes about $0.66$ radiation lengths \footnote{Radiation lengths measure the mean distance in which an electron loses all but $\frac{1}{e}$ of its energy.}. In order to achieve this, the solenoid and \ac{EM} calorimeter share a vacuum vessel, eliminating the need for two vacuum walls. It is made of Al-stabilised NbTi superconductor which allows a high electric field to be achieved ($7.730 \textrm{kA}$) while optimizing the thickness of the coil. The solenoid has an axial length of $5.8 \textrm{m}$ and radial thickness of $100 \textrm{cm}$ and it operates at a temperature of $4.5 \textrm{K}$. 


\section{Calorimeters}

The \ac{ATLAS} calorimeters measure the energy of electromagnetic and hadronic particles. The calorimeter system is composed of \ac{EM}, hadronic, and forward calorimeters. While they use a variety of different technologies to measure energies, they are both sampling calorimeters composed of alternating active and absorbing layers. Particles shower in absorbing layers and the showers are measured in the active layers, but the actual energy of each particle is not measured because some is lost to the absorbing layers. Unlike the tracker, the energy resolution of a calorimeter increases with increasing energy due to the increased signal generated. The size of each calorimeter is set by its radiation length or nuclear interaction length such that the calorimeter absorbs all of given particle's energy by the far end of the calorimeter and only muons and neutrinos should escape the calorimeters layers. A schematic of the \ac{ATLAS} calorimeters can be seen in \autoref{fig:atlas-calos}


%https://arxiv.org/pdf/1603.02934.pdf
\begin{figure}[htbp]
\centering
\includegraphics[width=.8\textwidth]{figures/Detector/atlas-calorimeters.jpg}
\caption{A diagram of the \ac{ATLAS} calorimeters with the major subsystems labeled. The \ac{EM} calorimeter is of particular importance for this analysis.}
\label{fig:atlas-calos}
\end{figure}


\subsection{Electromagnetic Calorimeter}

The \ac{LAr} \ac{EM} calorimeter is the innermost calorimeter and gives excellent energy and position resolution. It is composed of barrel ($|\eta| < 1.5$) and end-cap ($1.4 < |\eta| < 3.2$) components. Both components use a lead absorber with liquid argon active material. The layers of the calorimeter have an accordion shape (shown in \autoref{fig:atlas-lar}), which allows for the multiple absorbing layers without any gaps between them, as well as complete $\phi$ symmetry. The first layer has finer segmentation in $\eta$ to allow for more precise angular measurements of photons (which do not produce an \ac{ID} track). The thickness of the absorbing plates varies as a function of $\eta$ to optimize energy resolution. An active liquid argon presampler is placed before the accordion layers in the region with $|\eta| < 1.8$ to correct for energy loss upstream of the calorimeter. It is about 22 radiation lengths wide and gives energy resolution of $10\%/\sqrt{E} \oplus 0.7\%$

%https://www.researchgate.net/profile/Denis_Oliveira_Damazio/publication/229849719/figure/fig2/AS:667635272413188@1536188061121/Calorimeter-cells-for-different-layers-left-Note-the-very-fine-segmentation-in-the_Q320.jpg
\begin{figure}[htbp]
\centering
\includegraphics[width=.6\textwidth]{figures/Detector/lar.jpg}
\caption{A diagram of the \ac{ATLAS} \ac{EM} \ac{LAr} calorimeter. It has a unique shape in order to provide precision position and energy resolution.}
\label{fig:atlas-lar}
\end{figure}


\subsection{Hadronic Calorimeter}

The hadronic calorimeter surrounds the \ac{EM} calorimeter also within $|\eta| < 3.2$. The barrel region ($|\eta| < 1.7$) is made of steel absorbers with active material of scintillating tiles. Here, the calorimeter is about $2$m long in the radial direction and covers about $8$ interaction lengths. The Hadronic End-cap Calorimeters cover the region $1.5 < |\eta| < 3.2$ with a copper absorber and liquid argon active material.


The hadronic calorimeter has an energy resolution of  $50\%/\sqrt{E} \oplus 3\%$. 

\subsection{Forward Calorimeter}
The \ac{FCAL} is measures  both \ac{EM} and hadronic energy and extends coverage to $3.1 < |\eta| < 4.9$. The detector uses liquid argon as its active material with copper (for \ac{EM} activity) and tungsten (for hadronic activity) absorbers. It is about $10$ interaction lengths deep and also serves to add some shielding to the \ac{MS}. The energy resolution of the \ac{FCAL} is $100\%/\sqrt{E} \oplus 10\%$.




\section{Muon Spectrometer}
The Muon Spectrometer is the outermost subdetector designed to measure muons, which are too massive to be stopped by the \ac{LAr}. The \ac{MS} relies on a toroidal magnet system which enables high precision tracking in the three precision layers. Together, these give a constant momentum resolution of $10\%$ at $\pT > 1 \TeV$ \footnote{At high \pt, the \ac{MS} performance is independent of the \ac{ID} resolution.} and separate chambers, with timing resolution of $1.5-4 \textrm{ns}$, are used for triggering.  

\ac{MDT}s are used for precision tracking in the $\eta$ coordinate in the range $|\eta| < 2.7$, except in the inner most layer where the \ac{MDT}s extend only to $|\eta| < 2.0$ and \ac{CSC}s cover the region $2.0|\eta| < 2.7$. Triggering and $\phi$ measurements are provided by \ac{RPC}s in the range $|\eta| < 1.05$ and \ac{TGC}s in $1.05 |\eta| < 2.7$. A schematic of the \ac{MS} can be seen in \autoref{fig:atlas-ms}.  




%$https://cds.cern.ch/record/1631701/files/MuonSpectrometer_profile.png
\begin{figure}[htbp]
\centering
\includegraphics[width=.8\textwidth]{figures/Detector/atlas-ms.png}
\caption{A diagram of the \ac{ATLAS} \ac{MS} in the $r-z$ plane. The barrel \ac{MDT} are shown in green and the \ac{RPC} shown in black. In the end-caps, the \ac{MDT} are shown in blue and the \ac{TGC} shown in purple. }
\label{fig:atlas-ms}
\end{figure}




\subsection{Toroid Magnets}

The toroid magnet system, composed of a barrel and two end-caps, provides a toroidal magnetic field of $0.5 \textrm{T}$ and $1 \textrm{T}$ for the barrel ($|\eta| < 1.4$) and end-cap ($1.6 <|\eta| < 2.7$) regions of the \ac{MS} (in the transition region ($1.5 < |\eta| < 1.6$) muons are bent by a combination of the two fields). The toroid is much larger than the solenoid, $25.3 \textrm{m}$ long and $10 \textrm{m}$ in radial width, but also operates at a temperature of $4.5 \textrm{K}$. All three toroid magnets are made of Al-stabilised Nb/Ti/Cu conductor. They have an air-core structure, which gives them a strong bending power over a large volume while minimizing additional material scattering. 




\section{Particles in ATLAS}
The previously described subdetectors are used in combination to identify particles in \ac{ATLAS}. Charged particles interact with the \ac{ID} resulting in hits to be reconstructed into tracks. A track that points to a calorimeter cluster indicates the kind of charged particle that made the track and a cluster without an associated track indicates a neutral particle. The calorimeters are designed such that they absorb all of the energy of a particle and \ac{EM} particles do not enter the hadronic calorimeter, and hadronic particles do not enter the \ac{MS}. Muons do not interact with the calorimeters, but do leave a track in both the \ac{ID} and \ac{MS}. The only \ac{SM} particle that escapes the detector entirely is a neutrino. An undetected particle, like an \ac{SM} $\nu$ or some \ac{BSM} particle, could be seen as an imbalance in transverse momentum. The transverse momenta of all particles should sum to zero in order to conserve momentum, so any non-zero sum indicates an undetected particle.


\begin{figure}[htbp]
\centering
\includegraphics[width=.8\textwidth]{figures/Detector/particle-doodle.png}
\caption{A schematic of the signatures of Standard Model particles in the \ac{ATLAS} detector that illustrates how the subdetectors are used together to identify particles. Dashed lines indicate a particle trajectory that leaves no detector signature. Figure not drawn to scale nor does it represent a real physical process.}
\label{fig:particle-doodles}
\end{figure}





%$\chapter{Data Acquisition}

\section{Overview}
\section{General Challenges}
\section{Challenges for Long Lived Particles}
\section{The Fast TracKer}
\section{Fast TracKer Applications for Long Lived Particles and Future Prospects}

\chapter{Event Reconstruction}
\label{ch:EventReconstruction}

Event reconstruction is the process by which detector signals are converted into objects that can be used for physics analysis. This is a complex process that requires a great deal of focused effort by the \ac{ATLAS} collaboration. First, digital signals from the detector are collected into tracks and clusters, then they are combined to form reconstruction-level physics objects. Then, a identification steps is performed, where quality requirements are placed on the reconstruction-level objects to identify them as signatures of physical particles, like electrons and muons, that can be used in physics analyses. 

These algorithms are centrally developed by the collaboration and are generally designed to reconstruct and identify prompt objects ($|d_{0}| < 10 \textrm{mm}$). This section describes this process for objects which are relevant to this analysis, as well as the changes to these algorithms that have been implemented to be able to study objects with a wider range in \dzero.  

Reconstruction of tracks, including modifications to reconstruct tracks with high impact parameter, is described in \autoref{sec:trackreco}. Electron and muon reconstruction, as well as their modifications, are described in \autoref{sec:elecreco} and \autoref{sec:muonreco}, respectively. The reconstruction of jets, photons, and $\tau$ leptons is not discussed here. All of these objects are reconstructed in this analysis, though no selection is made on them. 


%-----------------------------
% Track Reconstruction
%-----------------------------
\section{Track Reconstruction}
\label{sec:trackreco}

Track reconstruction is the process by which \ac{ID} data is converted into particle trajectories. This is a complicated process, due to both the density of each event in the detector (in Run 2, there were an average of 40 collisons per bunch crossing, all of which produce hadronic sprays of particles), as well the helical trajectory (in $\phi$) the particles take due to the solenoidal magnetic field. Tracks are described by five parameters with respect to the beamspot position: \dzero (the transverse point of closest approach, or transverse impact parameter), \zzero (the longitudinal point of closest approach, or longitudinal impact parameter), $\phi$ (the azimuthal angle of the track momentum), $\theta$ (the polar angle of the track momentum), and $q/p$ (the ratio of the track's charge to the magnitude of its momentum).


%From ATLAS-tracking-algo.pdf
First, clusters from the \ac{ID} are converted into three-dimensional \emph{space-points}. In the Pixel detector, a space-point is simply one cluster, while in the \ac{SCT}, it is taken from both sides of a strip layer.

%From ATLAS-LRT.pdf
Tracking in ATLAS is performed in two steps. During the first step, called \emph{inside-out}, tracks are seeded from the silicon layers. In the Pixel and \ac{SCT} detectors, track seeds are formed from sets of three space-points, each from a separate silicon layer. If the seed passes an assortment of selection criteria, including on the \pt and \dzero, track candidates are built using a combinatorial Kalman filter (further described in \autoref{sec:kalman}). Seed requirements serve to cut down on the number of times the computationally expensive Kalman filter must be run. Multiple track candidates can be built from the same seed.

%From ATLAS-tracking-algo.pdf
Since all possible track combinations are created in the previous stage, an \emph{ambiguity solving} step is now required. Tracks are scored based on a variety of criteria, the number of holes (detector elements the track could have intersected with, but do not contain a cluster), $\chi^{2}$ (to prioritize tracks with a better fit), and \pt (to prioritize tracks with a higher \pt). A further requirement that no more than two tracks may share the same cluster reduces the number of duplicate tracks; however, a cluster may be removed from a track to stay within this limit and the track is then re-scored. 

Next, the track candidates are extended into the  \ac{TRT} using a classical track extrapolation technique, then the track is scored using a method similar to the ambiguity solver. If this extension is successful, the track is labeled as having a \emph{\ac{TRT} extension}, though the track can still be retained if this extensin fails, particularly at large $|\eta|$. However, if the score after TRT extension is worse than the silicon-only score, the track is rejected. This completes the \emph{inside-out} step.

\comebackto{How does classical extrapolation work?}


Step two takes an \emph{outside-in} approach, where track segments are reconstructed in the \ac{TRT}, seeded by deposits in the \ac{EM} calorimeter. These segments are then extended inward to the silicon detectors, where any clusters not used in step one can be associated to the track. The extension inward is not required, as \ac{TRT}-only tracks are used for reconstructing and identifying converted photons. 


\subsection{Large Radius Tracking}

\ac{LRT} is required to reconstruct tracks with impact parameters larger than what is allowed by the \ac{ST} algorithm. These requirements are designed for a maximum displacement a few mm, to enable the identification of heavy flavor hadron decays.

An optional third step of the tracking chain, it uses the same \emph{inside-out} tracking algorithm, but relaxes various requirements that allow for a much more inclusive track collection. The major changes are summarized in \autoref{tab:LRT}. These cuts are applied during both the seeding and extension steps. Additionally, \ac{LRT} uses a sequential Kalman filter as opposed to the combinatorial Kalman filter used in \ac{ST}.

\ac{LRT} is required for this analysis, but cannot be applied to all events in the Run 2 dataset. The full event reconstruction with \ac{LRT} takes about 2.5 times longer than with \ac{ST}, so events are filtered based on the triggers that selected them, such that this algorithm is only run on about 10\% of the dataset.


\begin{table}
\centering
\begin{tabular}{lcc}
\hline
Cut & \ac{ST} & \ac{LRT}  \\
\hline
$d_{0}^{\textrm{max}}$ (mm)   & 10   & 300 \\
$z_{0}^{\textrm{max}}$ (mm)   & 250   & 1500 \\
$ |\eta^{\textrm{max}}|$        & 2.7   & 5 \\
Max shared silicon modules    & 1     & 2 \\
Min unshared silicon clusters   & 6     & 5 \\
Min number of silicon hits   & 7     & 7 \\
\hline
\end{tabular}
\caption{Most important cuts that differ between \ac{ST} and \ac{LRT}}
\label{tab:LRT}
\end{table}

\begin{figure}[htbp]
\centering
\includegraphics[width=.5\textwidth]{figures/EventReconstruction/ST-d0-eff.png}
\includegraphics[width=.41\textwidth]{figures/EventReconstruction/LRT-d0-eff.png}
\caption{Number of tracks reconstructed with respect to \dzero in \ac{ST} (left) and \ac{LRT}. Note the difference in x-axis range.}
\label{fig:trking_d0_eff}
\end{figure}

\begin{figure}[htbp]
\centering
\includegraphics[width=.45\textwidth]{figures/EventReconstruction/ST-d0-res.png}
\includegraphics[width=.5\textwidth]{figures/EventReconstruction/LRT-d0-res.png}
\caption{\dzero resolution as a function of \pt in \ac{ST} (left) and \ac{LRT}. This analysis uses high \pt leptons with high \pt tracks, and thus with very good \dzero resolution}
\label{fig:trking_d0_res}
\end{figure}

%mostly used this: https://pdfs.semanticscholar.org/a591/7c383e4b07d64116b57c9c33e82138a08d12.pdf (p19+)
\subsection{Tracking Algorithms}

\subsubsection{\label{sec:kalman} Kalman Filter}

Kalman filters are widely used across LHC experiments for track fitting. It is a recursive algorithm that allows the user to efficiently extrapolate from a track seed. In the sequential, or extended, Kalman filter, uses a linear approximation to the track path. It makes a prediction for the next layer of detector material based on the seed, and then looks for a hit in that region. At each step, it updates the linearization to improve the measurement for the next step, where each measurement is weighted by its certainty. \ac{LRT} uses this approach.

In \ac{ST}, a combinatorial Kalman filter is used. This method employs several Kalman filters running in parallel and allows for the assumption that the measurement in the next layer is not necessarily assumed to be part of the track that formed the seed (as is often the case in the dense environment of the \ac{ATLAS} \ac{ID}). At each successive layer in track finding, several branches are extended if several measurements can be found in the same layer. The update of the linearization is done independently for each branch, and branches are created for missing measurements to account for detector inefficiencies. The branch is extended to the next layer if a measurement is found, and the process is repeated. Branches with no measurement for several layers are removed, and at the end, the branch with the best quality is selected. This process allows for parallelization of a complex combinatorial process. 

The Kalman filter works well for tracking in environments such as \ac{ATLAS} because even though it is computationally intensive, it generally gives the best precision.

\subsubsection{\label{sec:hough} Hough Transform}

%https://indico.cern.ch/event/602049/contributions/2429704/attachments/1440531/2217489/Piucci_05_04_2017.pdf
%hough-transform.pdf

A Hough transform is a pattern recognition algorithm that is useful for identifying a particular signature that can be described in a known parametric form. It performs well in noisy environments and is tolerant of holes in the signature. In relies on a transformation from physical to parameter space. Each point in physical space maps to a line in parameter space, and the intersection of the parameter space lines gives the values of constants of the parametric form. 

For example, if the pattern being searched for is a line described by $y = mx + b$, a Hough transform maps from $x-y$ space to $m-b$ space. Each point $(x_i, y_i)$ in the original image maps to a line in parameter space. If all $(x_i, y_i)$ are plotted together, the intersections of the lines in parameter space at $(m_i, b_i)$ define lines in physical space with parameters $m_i$ and $b_i$. 

The Hough transform is a \emph{pattern finding} algorithm, not a tracking algorithm. Is not ideal for full \ac{ID} tracking, because its complexity grows exponentially with the number of dimensions and the form of the pattern must be known \emph{a priori}.

\comebackto{How are seeds formed? Hough transform?}


\subsection{Primary Vertex Reconstruction}

After all of the tracks are reconstructed, they must be correctly assigned to a \ac{PV}. A \ac{PV} is the point in space where the $pp$ collision occurred. Generally, there are many \ac{PV}s per event: one is the hard-scatter, high-energy event of interest, and the others are pileup. In the Run 2 dataset, there are an average of 33 \ac{PV}s per event.

\ac{PV}s are reconstructed using an n iterative vertexing procedure. First, good quality tracks tracks, combined with the beamspot measurement, are used to find an optimal vertex position. Then each track is assigned a weight based on its compatibility with that position, and the vertex position is then recomputed using the weights of the tracks. After the final vertex position is determined, tracks very incompatible with the vertex are removed from it and can be used to create another vertex. Any vertex with at least two tracks are considered \ac{PV}s.



%-----------------------------
% Muon Reconstruction
%-----------------------------
\section{Muons}
%USED atlas-muon-reco.pdf
\label{sec:muonreco}
\subsection{Standard Reconstruction and Identification}

Muons are reconstructed by combining a \ac{MS} track with an \ac{ID} track. Then, at the identification stage, quality requirements are imposed on the combined tracks to improve the purity of the muon collection. For this analysis, the muon reconstruction algorithm remains unchanged (though \ac{LRT} tracks are used), while changes are made at the identification stage.

\subsubsection{Muon Track Reconstruction}
To reconstruct \ac{MS} tracks, a Hough transform (described in \autoref{sec:hough}) is used to search for hits in each \ac{MDT} chamber to find hits following a trajectory on the $\eta$ plane of the detector. These hits are fit to a straight line within each chamber to form \emph{segments}. Co-located \ac{RPC} and \ac{TGC} hits are used to measure the $\phi$ coordinate. 

Hits from segments in various layers are fit to form track candidates. This fitting is first seeded from segments in the middle layers of the \ac{MS} where more trigger hits are available, then extrapolated inward and outward. A next pass is done using inner and outer station segments as seeds. The extrapolation relies on relative positional and angular information, as well as the fit quality and hit multiplicity of the segments. Two segments are required to make a track, except in regions with limited detector coverage, where one high quality segment is sufficient. After all extrapolation, an overlap removal procedure is performed, allowing for a segment to be shared between at most two tracks. A $\chi^{2}$ test is performed, where outliers hits can be removed from the track and additional hits consistent with the track candidate's trajectory can be added.

\comebackto{what are "trigger hits"}


\subsubsection{Combined Muons}
Next, the \ac{MS} tracks are combined with \ac{ID} tracks to form \ac{CB} muons, a track fit over the two tracks. \ac{CB} muon tracks are generally seeded from the \ac{MS}, then extrapolated inward and matched to an \ac{ID} track, but the inverse is also allowed. Hits from the \ac{MS} may be added or removed to improve the fit between the two tracks.



\subsubsection{Muon Identification}

This analysis uses the default muon working point for ATLAS analyses, called a \emph{medium} muon. This working point places a requirement on the number of \ac{ID} and \ac{MS} hits that comprise the track to ensure a robust momentum measurement. At least 1 pixel hit, at least 5 \ac{SCT} hits are required and at least 10\% of the \ac{TRT} hits associated to the object must be included in the final fit. There must also be fewer than 3 holes in the silicon tracking layers. Furthermore, the \ac{MS} track must have at least 3 hits in at least 2 \ac{MDT} layers. In the crack region $|\eta| < 0.1$, \ac{MS} tracks with at least three hits in only one \ac{MDT} layer are allowed provided there are no holes in the track. Finally, a loose requirement is placed on the consistency between the \ac{MS} and \ac{ID} tracks. Namely, the \emph{q/p significance}, the difference between the charge and momentum ratio in the \ac{ID} and \ac{MS} divided by their uncertainties summed in quadrature, is required to be less than 7. 


\begin{figure}[htbp]
\centering
\includegraphics[width=.43\textwidth]{figures/EventReconstruction/muon-reco-eta.png}
\includegraphics[width=.52\textwidth]{figures/EventReconstruction/muon-reco-pt.png}
\caption{Medium muon identification efficiency with standard tracking and criteria. Left is efficiency vs $\eta$ and right is efficiency as a function of $p_{T}$. Medium muons are reconstructed very efficiently, except for around $|\eta| \approx 0$, where the \ac{MS} is missing detector coverage.}
\label{fig:std_muon_eff}
\end{figure}


\subsection{Modifications}

For this analysis, muons are reconstructed after \ac{LRT} is performed and the reconstruction and identification efficiency is quite high. Furthermore, at the identification stage, we remove the requirement that the \ac{ID} track has at least one pixel hit, further improving the efficiency at high $d_{0}$. The effect of these improvements is show in \autoref{fig:cust_muon_eff}


This modification of the muon identification increases the fake rate of muons, so again we impose quality requirements that are independent of displacement. We require the muon to have at least two \ac{MS} layers with at least three precision hits, that the muon have at least one $\phi$ measurement (otherwise the \ac{MS} $\phi$ measurement is taken as the center of the \ac{MDT}, with an uncertainty of 0.2) and require the $\chi^{2}_{CB}/N_{DoF} < 3$. The $\chi^{2}_{CB}$ requirement is, in effect, a requirement on the consistency between the the \ac{ID} and \ac{MS} tracks. These will be further discussed in \autoref{sec:mu_qual_req}


\begin{figure}[htbp]
\centering
\includegraphics[width=.48\textwidth]{figures/EventReconstruction/wp_m_d0_all_wip.pdf}
\includegraphics[width=.48\textwidth]{figures/EventReconstruction/wp_m_pt_all_wip.pdf}
\caption{Muon identification efficiency with modified criteria. Left is efficiency vs $d_{0}$ and right is efficiency as a function of $p_{T}$. The red denotes reconstruction with Standard Tracking (ST), and the blue with Large Radius Tracking (LRT), and the filled in circles use the modified identification working point. \todo{Needs update when available}}
\label{fig:cust_muon_eff}
\end{figure}



%-----------------------------
% Electron Reconstruction
%-----------------------------
\section{Electrons}
%USED atlas-electron-reco.pdf atlas-electron-reco-sliding-window.pdf
\label{sec:elecreco}

Electrons are reconstructed using clusters from the \ac{EM} calorimeter as well as tracks from the \ac{ID}. Electron reconstruction brings more complication and ambiguity than muon reconstruction because of the presence of photons, converted photons, and bremsstrahlung radiation from electrons moving through material. These factors make the identification and accurate measurement of electrons quite challenging. More than in muon reconstruction, it relies on displacement-based quality information, posing a problem for this search. We modify these requirements and use \ac{LRT} tracks, but have a resulting lower selection efficiency. 

\subsection{Standard Reconstruction and Identification}


\subsubsection{Cluster Reconstruction}

First, clusters are formed from $\eta \times \phi$ \emph{towers} of size $\Delta \eta \times \Delta \phi = 0.025 \times 0.025$ (roughly granularity of the second layer of the \ac{EM} calorimeter, where about 80\% of the energy in a shower is deposited). In each region, the energy deposited in all layers of the calorimeter is summed and are used as input to a seeding algorithm to form clusters. 

A \emph{sliding window} algorithm is used to form clusters. In this algorithm, an $3 \times 5$ window is slid across each tower, the energy is summed inside of this window. If the sum is a local maximum and is above a threshold of $\et > 2.5 \GeV$, this window is considered a \emph{cluster}. A duplicate removal process is then performed for nearby clusters with similar energy measurements, keeping only the cluster with the largest $E_{T}$. Inefficiency in the cluster reconstruction step is negligible compared to the uncertainty in the next two steps. The efficiency of this step is 65\% at $\et = 4.5 \GeV$ and $> 99\%$ above $\et = 15 \GeV$.

\begin{figure}[htbp]
\centering
\includegraphics[width=.8\textwidth]{figures/EventReconstruction/electron-reco-sketch.png}
\caption{A sketch of an electron's path through a slice of the \ac{ATLAS} detector.}
\label{fig:elec_reco_sketch}
\end{figure}

\subsubsection{Tracking}
Since electrons are so light, they can lose a significant amount of their energy due to bremsstrahlung radiation as they traverse the \ac{ID}, thus resulting in a track seed that cannot be extended to the requisite number of silicon layers using the processes described in \autoref{sec:trackreco}. Thus, a second pass at tracking, allowing for 30\% energy loss due to bremsstrahlung radiation at each detector layer is performed in the vicinity of a good quality \ac{EM} cluster. Here, the track candidate $p_{T}$ is lowered to $400 \MeV$ (compare to $1 \GeV$), but still uses the standard hypothesis that the track is like that from a pion. If this fit still fails, a third pass is performed using the assumption that the track is like that from an electron, allowing for an additional degree of freedom in the $\chi^2$ calculation that accounts for additional radiation. In all, this gives 98\% reconstruction efficiency for electrons with $\et > 10 \GeV$. 

This loosened track fitting requirements allow for increased efficiency, but the resulting tracks do not correctly account for the energy loss of the electron to the material. An additional tracking pass, using an optimized \ac{GSF} is used to correct for this. 

This procedure is performed on tracks with at least 5 silicon hits and roughly match to an \ac{EM} cluster. The \ac{GSF} procedure is based on a combinatorial Kalman filter, resulting in track parameters weighted by Gaussian function describing material-induced energy losses. This procedure also accounts for the increase in curvature caused by the decrease in momentum due to energy loss in material, improving the calculation of track parameters. An example of this is show in the figure below. The reconstruction efficiency for this step is around 98\% for electrons with $\et > 30 \GeV$. 

\begin{figure}[htbp]
\centering
\includegraphics[width=.7\textwidth]{figures/EventReconstruction/elec-qxd0.png}
\caption{The impact on the $q \times \dzero /\sigma(\dzero)$ measurement by the addition of \ac{GSF} tracking. This is sampled from prompt electrons, so the narrower distribution centered around zero indicates significant improvement.}
\label{fig:elec_gsf}
\end{figure}


\subsubsection{Combined Reconstruction}
To put it all together, \ac{GSF} tracks are then matched to seed calorimeter clusters then the final cluster size is determined. Tracks and clusters are matched with stricter requirements, requiring that their $\phi$ measurement is within $+0.05$ or $-0.10$. It is possible that several tracks might match to the same cluster, but a primary track is assigned based on its proximity and quality. If the track can be associated to a vertex, it is classified as a potential photon conversion, not an electron. Electrons are further distinguished from photon conversions using their $E/p, p_{T}$, and number of pixel hits. 

Final clusters are formed by looking in a window around the seed cluster. The reconstructed electron's energy measurement is taken from its full cluster, and directional information taken from its track. At $\et > 15 \GeV$, there is a 99\% efficiency to reconstruct an electron (provided it has at least one pixel hit and at least seven total silicon hits on track). 

\begin{figure}[htbp]
\centering
\includegraphics[width=.4\textwidth]{figures/EventReconstruction/elec-reco.png}
\includegraphics[width=.5\textwidth]{figures/EventReconstruction/elec-reco-steps.png}
\caption{Electron reconstruction efficiency as a function of true transverse energy, \et. For the various electron working points (left) and for each step in the reconstruction process (right).}
\label{fig:elec_gsf}
\end{figure}

\subsubsection{Identification}

Electrons are identified using a likelihood that further distinguishes them from photons, light flavor jets, and leptonic heavy flavor decays. The likelihood is more flexible than a simple cut-based method, allowing for electrons to fail one selection criterion. It also allows for the use of discriminating criteria which have relatively similar shapes in signal and background. A some cuts are made at the identification step, including on the number of pixel and silicon hits, as well as on the shower width. Many other factors not cut on but are included in the likelihood including the track quality, consistency between the electron's track and its cluster, as well energy ratios in the various layers of the \ac{EM} calorimeter, as well as the \ac{EM} calorimeter compared to the hadronic calorimeter.


\subsection{Modifications}
To be able to reconstruct electrons with high impact parameter, several changes needed to be made to the reconstruction and identification algorithms. 

Similar to the muon case, the reconstruction is then run on the track collection including \ac{LRT} tracks. At the identification stage, we remove variables concerned with $d_{0}$ from the likelihood consideration, but do not retrain the likelihood itself. We also remove the cut on the number of silicon hits on top of that made at the reconstruction stage. 

After these modifications, we introduce many fake electrons, primarily resulting from a fake \ac{LRT} track being associated to an \ac{EM} cluster from a photon. The most powerful discriminator is the consistency in the $p_{T}$ as measured by the track and the cluster, defined as $\Delta p_{T} = p_{T}^{track}/p_{T}^{e}$. Furthermore, we require the primary track to be good quality, with $\chi^{2} < 2$ and at maximum one missing hit on track after the innermost hit. These will be further discussed in \autoref{sec:elec_qual_req}

\begin{figure}[htbp]
\centering
\includegraphics[width=.48\textwidth]{figures/EventReconstruction/wp_e_d0_all_wip.pdf}
\includegraphics[width=.48\textwidth]{figures/EventReconstruction/wp_e_pt_all_wip.pdf}
\caption{Muon identification efficiency with modified criteria. Left is efficiency vs $d_{0}$ and right is efficiency as a function of $p_{T}$. The red denotes reconstruction with Standard Tracking (ST), and the blue with Large Radius Tracking (LRT), and the filled in circles use the modified identification working point. \todo{Needs update when available}}
\label{fig:cust_elec_eff}
\end{figure}




\section{Isolation}

In this analysis, both muons and electrons are required to be isolated to reduce background from heavy flavor decays. That is, they are not expected to be surrounded by much other activity from the hard scatter in either the Inner Detector or Calorimeters. Isolation is generally measured in terms as the scalar sum of energy in a radius $\Delta R = \sqrt{(\Delta \eta)^2 + (\Delta \phi)^2}$ around the lepton. Track based isolations are called $p_{T}^{\textrm{varcone}}$ and calorimeter based isolations called $E_{T}^{\textrm{topocone}}$. 

For example, the track-based isolation called $p_{T}^{\textrm{varcone30}}$ is the scalar sum of the transverse momenta of tracks with $p_{T} > 1 \GeV$ in a cone of $\Delta R = \textrm{min}(10 \GeV/p_{T}^{\ell}, 0.3)$, then a value is selected to determine what should be considered ``isolated''. Whereas the calorimeter-based isolation just counts the energy in a specified \dR, not as a function of the \pt of the particle. Isolation working points are then centrally defined as a combination of cuts on the track-based and calorimeter-based isolation. 

These definitions are much simpler for muons than for electrons. Electrons are very likely to emit bremsstrahlung radiation as they traverse the inner detector and those photons can then convert back into electrons. The tracks from these secondary particles are considered part of the electron's \pT. Furthermore, the electron leaves its own energy deposit in the \ac{EM} calorimeter and this energy must be subtracted from the $E_{T}^{\textrm{topocone}}$ calculation.


In this analysis we use the following working points \todo{grab isolation requirements from INT when finalized}




\chapter{Data and Monte Carlo Samples}

This chapter describes the data and simulated Monte Carlo samples used together to perform this analysis. ATLAS data from Run 2 of the LHC undergoes a special processing in order to make use of the special tracking and object reconstruction required to identify the targetted model. Monte Carlo samples of both the bechmark model and Standard Model backgrounds are generated. All backgrounds are estimated from data, but Monte Carlo is used extensively to optimize signal selection.

\section{Data}

This analysis makes use of $139~\ifb$ of $\sqrt{s}=14\TeV$ $pp$ collision data delivered by the \ac{LHC} and taken by the \ac{ATLAS} detector between $2015$ and $2018$. The data used in this analysis are collected using three different triggers depending of the topology of the event, described in \autoref{tab:triggers}. They are applied in the order listed to ensure consistency inside each event topology

\begin{table}[htb]
\small
\begin{center}
\begin{tabular}{|l|c|}
\hline
Topology       & Trigger \\
\hline\hline
\texttt{if} $\geq e,~\pt > 160~\gev$      & \texttt{HLT\_g140\_loose}    \\
\texttt{else if} $\geq 2e,~\pt > 60~\gev$  & \texttt{HLT\_2g50\_loose}\footnote{The L1 seed for this trigger changed in 2017, and an otherwise identical trigger was used, \texttt{HLT\_2g50\_loose\_L12EM20VH}.} \\
\texttt{else if} $\geq 1 \mu,~\pt > 60~\gev,~|\eta| < 1.07$                            & \texttt{HLT\_mu60\_0eta105\_msonly}   \\
\hline
\end{tabular}
\caption{Summary of trigger selection based on event topology. This serves to have a consistent data stream inside each event topology. The first line is requested, if the event topology is not met, the second line is requested. If the event has the specified topology, but fails the trigger selection, the event is not selected.}
\label{tab:triggers}
\end{center}
\end{table}

The \ac{LRT} and special recontruction used in this analysis are computationally intensive and introduce many more fake tracks and physics objects, so it is only run over a subset (roughly $ 10\% $) of the full \texttt{physics\_Main} dataset. Events are filtered based on fired triggers and loose requirements on reconstruction-level objects. This analysis uses filters based on the triggers described in \autoref{tab:triggers} combined with selections on reconstruction-level muons, \ac{MS}-only muons, electrons, and photons (to account for loss of efficiency reconstructing electrons with standard tracking) listed in \autoref{tab:filters}.

\begin{table}[htb]
\small
\begin{center}
\begin{tabular}{|l|cccc|cccc|}
\hline
Filter                            & \multicolumn{4}{c|}{Object 1}       & \multicolumn{4}{c|}{Object 2} \\
                                  & Object & \pt\ [GeV] & $|\eta|$ & \absdz [mm] & Object &\pt\ [GeV] & $|\eta|$ & \absdz [mm] \\
\hline\hline
Single muon                       & $\mu$ & $>60$                   & $<1.07$                 & > 2 & & & & \\%\multirow{4}{*}{-} \\
\hline
\multirow{3}{*}{Single photon}    & $\gamma$& \multirow{3}{*}{$>160$} & \multirow{3}{*}{$<2.5$} & \multirow{3}{*}{-} & $\gamma$ & > 10 & < 2.5 & - \\
                                  & $\gamma$& & & &                                                                     $e$     & > 10 & < 2.5 & > 2 \\
                                  & $\gamma$& & & &                                                                     $\mu$   & > 10 & < 2.5 & > 2 \\
\hline
Single electron                   & $e$ & $>160$                  & $<2.5$                  & $>2$ & & & & \\ %\multirow{4}{*}{-} \\
\hline
Di-Photon                         & $\gamma$ & $> 60$                  & $<2.5$                  & -    & $\gamma$ & > 60 & < 2.5 & - \\
Di-Electron                       & $e$      & $> 60$                  & $<2.5$                  & > 2  & $e$      & > 60 & < 2.5 & > 2 \\
Di-Electron/Photon                & $e$      & $> 60$                  & $<2.5$                  & > 2  & $\gamma$ & > 60 & < 2.5 & - \\

\hline
\end{tabular}
\caption{Offline selection applied to select single $e/\gamma$ or muon event objects for \texttt{DRAW\_RPVLL} processing. For muons, the \absdz > 2 mm requirement is only enforced on combinded muons.}
\label{tab:filters}
\end{center}
\end{table}

If an event is selected by a filter, it is saved in its \texttt{RAW}, or detector-level, format in a \texttt{DRAW\_RPVLL} dataset. This \texttt{DRAW\_RPVLL} dataset then undergoes the full reconstruction chain, this time including \ac{LRT} during track reconstruction, and is saved in the same \texttt{xAOD} format as the standard datasets, now called \texttt{DAOD\_RPVLL}. These datasets are further processed into \texttt{DAOD\_SUSY15} derivations, which reduce the size of the dataset by making additional selections on the physics objects which now have the special reconstruction. Finally, the \texttt{DAOD\_SUSY15} derivations are processed by analysis-specific code into flat n-tuples that can be easily used for studies and background estimates. 

\section{Monte Carlo}

Monte Carlo generation is the process by which a physical model is simulated. It allows analyzers to understand how a specific physical process would appear in the \ac{ATLAS} detector. For this analysis, Monte Carlo samples are generated for various masses and lifetimes of theoretical sleptons, so that event selection can be optimized for the widest sensitivity. This is done by breaking up the process into many steps. 


%PDF ref: https://arxiv.org/abs/1207.1303
%MADgraph: https://arxiv.org/abs/1405.0301
%pythia: https://cds.cern.ch/record/1966419
First, the physical process at the collision is modeled. The center of mass energy must be determined, as discussed in \autoref{chap:LHC}, this is not a known quality and is modeled by \ac{PDF}s. There are many different choices of \ac{PDF}, in this analysis \texttt{NNPDF23LO}. Then, the \emph{generator} models the production of the given physical process by calculating its Matrix Elements. Generally, thought not in this analysis, the generator step models the full process from hard scatter to final state particles including all of their kinematic properties. For this analysis, we use \texttt{MadGraph5\_aMC@NLO} to produce $pp \rightarrow \tilde{\ell}\tilde{\ell}$ events with up to two additional radiated partons, using a perturbative \ac{QCD} calculation at leading order accuracy. Then these partons are hadronized and showered into \emph{jets} using \texttt{Pythia8.230} using the \texttt{A14} tuning of the parton showering, hadronization, and modeling of the underlying event. Finally, simulated pileup collisions are overlaid onto the event to mimic the actual conditions at the \ac{LHC}. At the end of this stage, particles are referred to as \emph{truth-level}.

Next, the full event is propagated through a simulation of the detector created in \texttt{GEANT4}. Each particle created in the previous steps passes through the detector and all of its magnetic fields and support structures in order to as accurately as possible simulate particle trajectories, detector signatures, and other interactions with material. Until this point the sleptons have been treated as stable and travel through and interact with the detector. Their decay into a lepton and gravitino is a simple two-body decay, and is performed at this stage. At the end of this stage, particles are referred to as \emph{detector-level}.

Finally, the trajectories of the particles is \emph{digitized}, to emulate the readout of the actual detector, and reconstruction is run on the simulated detector signals in the same way it is run for real data. Monte Carlo is not required to undergo the same filtering as data, but is processed with \ac{LRT} and made into \texttt{DAOD\_SUSY15} and analysis-specific n-tuples. At the end of this stage, particles are referred to as \emph{reconstruction-level}. 

The importance of Monte Carlo simulations to an analyzer's ability to search for new physics and measure the \ac{SM} cannot be overstated and an enormous amount of effort is made to make each step as accurate and precise as possible. However, many corrections must be made to these samples after they are generated (specifically for the pileup modeling), and even so discrepancies between data and \ac{MC} are an important uncertainty in this analysis and many others.

\todo{table of analysis lifetimes and masses, maybe some truth level kinematics.}
In this analysis 

In addition to the signal \ac{MC} samples, several background \ac{MC} samples are used. However, these have limited use in this analysis since the major backgrounds are not well modeled by \ac{MC}. These include \ttbar sample, simulating the production and semi-leptonic decay of two top quarks $pp\rightarrow \ttbar$; a \bbmm sample, simulating the production and leptonic decay of two bottom quarks $pp\rightarrow \ttbar \rightarrow \mu\mu$; and a sample of photons. 






\cleardoublepage 

%----------------------------------------------------------------------------------------
\part{Long Lived Particle Context}

\chapter{Long Lived Particles}

\section{Motivation}
\section{Basics}

\section{Experimental Context}
\label{chap:context}


\cleardoublepage
%----------------------------------------------------------------------------------------
\part{Search for Displaced Leptons}
\chapter{Event Selection}


\section{Datasets}
\subsection{Recorded Data Streams}
\subsection{Trigger Strategy}
\subsection{Monte Carlo Samples}

\section{Electron Selection}
\subsection{Quality Requiremens}
\subsection{Efficiency}

\section{Muon Selection}
\subsection{Quality Requirements}
\subsection{Efficiency}

\section{Final Event Selection}

\chapter{Backgrounds}

In this analysis, backgrounds are estimated per signal region. In SR-$ee$ and SR-$e\mu$, algorithmic fakes are the dominant source of background. In SR-$\mu\mu$, the background contribution from algorithmic fake muons and muons from heavy flavor decays is negligible ($\mathcal{O}(10^{-4})$), and the dominant background is from cosmic muons coincident with \ac{LHC} collisions. While the signal lepton selection and event selection described in the previous chapter very efficiently remove these backgrounds, background estimates are calculated to estimate any residual contribution. 

\section{Background to SR-$\mu\mu$}

\subsection{Cosmic Muon Identification}
\label{sec:cosmics}
Muons from cosmic rays constantly pass through the earth and thus the \ac{ATLAS} detector, particularly through the service shaft above the detector where there is no layer of earth above the detector\footnote{The cosmic muon flux at sea level is about 1 $\mu$ per cm$^2$ per minute at sea level. ATLAS is approximately 46 m long and 50 m wide and a single data-taking run lasts 8 hours. That's $10^{10}$ cosmic muons per run! Of course, most of \ac{ATLAS} is more than 50 m underground, which absorbs most of the cosmic muons. A muon from a cosmic ray must also be exactly coincident with a bunch crossing to be triggered and well reconstructed. \todo{Do this better with the flux under ground}}. If a cosmic ray muon were coincident with a bunch crossing, the event could be triggered, reconstructed, and enter the dataset used for this analysis. The cosmic ray muon could pass through the entire detector at any distance from the \ac{PV}, interacting with the \ac{ID} and \ac{MS}, and be reconstructed as two muons with high \absdz, passing all quality variables (because the signature comes from a real muon) exactly mimicking the signature of SR-$\mu\mu$. 

A very efficient cosmic tag was defined for this analysis using a reoptimization of the strategy used by \cite{ATLAS-CONF-2019-006}.  This method first defines a spatial cosmic identification, and then conservatively tags all muons which would be impossible to identify using this method due to detector coverage. The cosmic muon, \mcos, passes through the entire detector and gets reconstructed as two muons, one with $\phi > 0$ and the other with $\phi < 0$. They would have 

\begin{equation}
\Delta \phi_{\text{cos}} = \phi_{\mu_{0}} - \phi_{\mu_{1}} = \pi 
\end{equation}
and 
\begin{equation}
\Sigma \eta = \eta_{\mu_{0}} + \eta_{\mu_{1}} = 0
\end{equation} 

The combination of these variables form a useful variable to describe events with 2 cosmic muons.
\begin{equation}
\Delta R_{\text{cos}} = \sqrt{ (\phi_{\mu_{0}} - \phi_{\mu_{1}})^{2} + (\eta_{\mu_{0}} + \eta_{\mu_{1}})^2}
\end{equation}

The muon reconstruction algorithm described in \autoref{sec:muonreco} uses the momentum direction measured in the \ac{MS} in its extrapolation from \ac{MS} track to \ac{ID} track. In 90\% of cases, a only the  $\phi < 0$ muon on the bottom of the detector, \mb is reconstructed and identified, while the detector signature for \mt, the muon with $\phi > 0$, exists but the fully reconstructed muon does not. Thus it is advantageous to tag a cosmic muon as one which is back to back with activity in the \ac{MS}, not as two muons back to back, illustrated in \autoref{fig:tag_sketch}. A cosmic veto is defined based on the \dphicos and \sigeta between a muon and a \ac{MS} segment. 

\begin{figure}[!ht]
\centering
\includegraphics[width=.8\textwidth]{figures/cosmics/tag_sketch.png}
\caption{A sketch of a cosmic passing through the ATLAS detector, illustrating why the tag is designed the way that it is. This image is slightly adapted from Ref.~\cite{ATLAS-CONF-2019-006}}
\label{fig:tag_sketch}
\end{figure}

Since the \ac{MS} is so far from the \ac{PV}, the \z of individual MS segments is not measured, so they point to the origin. This creates a mismatch between the $\eta$ of the segment and the $\eta$ of the reconstructed muon in the determination of the cosmic tag. This is geometrically corrected for by re-calculating the $\Delta\eta$ between the segment and the muon during the cosmic veto. A schematic of this correction is shown in \autoref{fig:cos_eta_recalculation} and its effect on cosmic and signal muons shown in \autoref{fig:cos_eta_phi}. This narrows the distribution of cosmic muons by an order of magnitude in \sigeta. This distribution is isotropic for signal muons, so this definition allows for high cosmic muon rejection with minimal signal rejection. 

\begin{figure}[!ht]
\centering
\includegraphics[height=4cm]{figures/cosmics/eta_correction_1.png}
\includegraphics[height=4cm]{figures/cosmics/eta_correction_2.png}
\includegraphics[height=5cm]{figures/cosmics/eta_correction_3.png}
\caption{This series of figures shows the problem of the $\eta$ recalculation. The MS segment should be measured as back to back in $\eta$ and $\phi$ with the muon, because it is really one high $p_{T}$ object moving through the whole detector. However, because the MS segments are reconstructed assuming they come from the origin, they will not actually be measured as back-to-back with the muon. We calculate the $\eta$ that would be measured by a segment back to back with the muon, and compare it to the $\eta$ of all other segments in the event.}
\label{fig:cos_eta_recalculation}
\end{figure}

\begin{figure}[!ht]
\centering
\includegraphics[width=.48\textwidth]{figures/cosmics/v4_widetag_2_sumEta_dPhi_min.pdf}
\includegraphics[width=.48\textwidth]{figures/cosmics/v4_widetag_2_sumEta_dPhi_min_corr.pdf}
\includegraphics[width=.48\textwidth]{figures/cosmics/300_slep_2_sumEta_dPhi_min.pdf}
\includegraphics[width=.48\textwidth]{figures/cosmics/300_slep_2_sumEta_dPhi_min_corr.pdf}
\caption{The $\Sigma\eta - \Delta\phi$ distribution is shown before (left) and after (right) the $\eta$ recalculation. The top row shows the distributions in VR-$\mu$ and bottom row 300 GeV slepton signal samples in our full range of lifetimes. Using this cosmic tag, we can cut very tightly on cosmic muons without losing signal efficiency.}
\label{fig:cos_eta_phi}
\end{figure}


The distribution the $\dphicos-\sigeta$ distribution is much narrower in \sigeta than in \dphicos. This is because the \ac{MS} measures $\eta$ with an extremely high precision in the \acp{MDT} ($\mathcal{O}(10~\um)$), since it is the bending direction of the toroid and thus gives the momentum measurement, while the $\phi$ is measured by the \acp{RPC} with an order of magnitude less precision (($\mathcal{O}(10~\text{mm})$)). A high precision $\phi$ measurement of the combined muon comes from the \ac{ID}, but the cosmic tag must contend with these resolution issues to find a muon segment.

Additionally, there are gaps in the \ac{MS} to allow for detector access, so a muon is conservatively tagged as a cosmic if it is back to back with this gap in detector coverage, as it could not be tagged as a cosmic using the geometric algorithm. A map of the material of the \ac{MS} is used to veto cosmic muons using this \emph{detector coverage veto}. \autoref{fig:cos_material_veto} shows the impact of each step on the muon distribution in VR-$\mu$, which is dominated by muons from cosmic rays.


\begin{figure}[!ht]
\centering
\includegraphics[width=.48\textwidth]{figures/cosmics/v4_widetag_2_eta_phi_baseline.pdf}
\includegraphics[width=.48\textwidth]{figures/cosmics/v4_widetag_2_eta_phi_costag.pdf}
\includegraphics[width=.48\textwidth]{figures/cosmics/v4_widetag_2_eta_phi_costag_mv.pdf}
\caption{The $\eta-\phi$ distribution of muons in VR-$\mu$ are shown after baseline cuts (top left), the cosmic veto (top right), then the detector coverage veto (bottom).}
\label{fig:cos_material_veto}
\end{figure}

For this analysis, two different definitions of a ``cosmic muon'' are used. The \emph{nominal tag} is used to veto events in the \ac{SR}. If any muon is tagged by the nominal cosmic tag, the entire event is vetoed. A second \emph{narrow tag} is used during the validation of the estimate of background from cosmic muons. Both tags are required to pass the detector coverage veto. \autoref{tab:costag_values} describes the values used for the various cosmic tags used and \autoref{fig:cos_tag} shows the cuts used in the \dphicos-\sigeta distribution. 

\begin{figure}[!ht]
\centering
\includegraphics[width=.48\textwidth]{figures/cosmics/cosmic_tag.png}
\caption{The $\Sigma\eta_{\mu,\textrm{seg}} - \Delta\phi_{\mu,\textrm{seg}}$ distribution of signal quality muons in VR-$\mu$. The red line and shadow shows the bounds of the full cosmic tag, while the blue line and shadow shows the boundary of the narrow cosmic tag. Everything inside the respective boxes is tagged as a cosmic. The intermediate tag defines the region included in the full tag, but not included in the narrow tag.}
\label{fig:cos_tag}
\end{figure}

\begin{table}
\centering
\begin{tabular}{lcccc}
Tag & $\Delta \phi (\textrm{muon, segment})$ & $\Sigma \eta (\textrm{muon, segment})$ & Pass Coverage Acceptance & Rejection Efficiency\\
\hline
Full   & $<0.25$   & $ <0.018 $   & True & 99.5\% \\
Narrow & $<0.02$   & $ <0.013$    & True & 59\% \\
\hline
\end{tabular}
\caption{Cuts applied in full and narrow cosmic tags. The narrow is contained in the full tag. An intermediate tag is defined as the region between the full and narrow tags, that is, tagged by the full but not the narrow tag.}
\label{tab:costag_values}
\end{table}

The cosmic tagging efficiency is determined by fitting distributions in VR-$\mu$. A template of the \tavg of cosmic tagged muons and prompt (collision) muons is taken from data \todo{insert figure}. The measurement of \tavg does not change with displacement, as the timing resolution is the dominant effect over the ns-level timing offsets from long lived slepton decays. These are used to determine the fraction of cosmic muons in VR-$\mu$ before and after the cosmic tagged muons are removed from VR-$\mu$, which can then be used to evaluate the fraction of muons vetoed by the various cosmic vetoes. The nominal tag removes 99.5\% of cosmic muons and the narrow tag removes 59\% of cosmic muons. The nominal tag removes 8\% of collision muons, primarily from the detector coverage veto. \autoref{fig:cos_eff} show the cosmic tagging efficiency in VR-$\mu$ compared to signal \ac{MC}. 

\begin{figure}[!ht]
  \centering
  \includegraphics[width=0.38\textwidth]{figures/cosmics/t0_avg_template_comp.pdf}
  \includegraphics[width=0.38\textwidth]{figures/cosmics/t0_avg_prepost_comp.pdf}
  \caption{\tavg of cosmic and prompt muons used as templates to the fit (left) and VR-$\mu$ dataset before and after removal of cosmic tagged muons on which the fit was performed.}
  \label{fig:d0_t0avg}
\end{figure}


\begin{figure}[!ht]
  \centering
  \begin{subfigure}[b]{0.4\textwidth}
 	\includegraphics[width=\textwidth]{figures/cosmics/wider_tag_ratio_pt.pdf}
  	\caption{VR-$\mu$ data with respect to \pt}
  \end{subfigure}
  \begin{subfigure}[b]{0.4\textwidth}
 	\includegraphics[width=\textwidth]{figures/cosmics/mc_300_ratio_pt.pdf}
  	\caption{Signal \ac{MC} with respect to \pt}
  \end{subfigure}

  \begin{subfigure}[b]{0.4\textwidth}
  	\includegraphics[width=\textwidth]{figures/cosmics/wider_tag_ratio_d0.pdf}
  	\caption{VR-$\mu$ data with respect to \dz}
  \end{subfigure}
  \begin{subfigure}[b]{0.4\textwidth}
  	\includegraphics[width=\textwidth]{figures/cosmics/mc_300_ratio_d0.pdf}
  	\caption{Signal \ac{MC} with respect to \dz}
  \end{subfigure}

  \begin{subfigure}[b]{0.4\textwidth}
  	\includegraphics[width=\textwidth]{figures/cosmics/wider_tag_ratio_z0.pdf}
  	\caption{VR-$\mu$ data with respect to \z}
  \end{subfigure}
  \begin{subfigure}[b]{0.4\textwidth}
  	\includegraphics[width=\textwidth]{figures/cosmics/mc_300_ratio_z0.pdf}
  	\caption{Signal \ac{MC} with respect to \z}
  \end{subfigure}
\end{figure}

\begin{figure}[!ht]
  \ContinuedFloat
  \centering
  \begin{subfigure}[b]{0.4\textwidth}
  	\includegraphics[width=\textwidth]{figures/cosmics/wider_tag_ratio_phi.pdf}
  	\caption{VR-$\mu$ data with respect to $\phi$}
  \end{subfigure}
  \begin{subfigure}[b]{0.4\textwidth}
 	\includegraphics[width=\textwidth]{figures/cosmics/mc_300_ratio_phi.pdf}
  	\caption{Signal \ac{MC} with respect to $\phi$}
  \end{subfigure}

 \begin{subfigure}[b]{0.4\textwidth}
  	\includegraphics[width=\textwidth]{figures/cosmics/wider_tag_ratio_eta.pdf}
  	\caption{VR-$\mu$ data with respect to $\eta$}
  \end{subfigure}
  \begin{subfigure}[b]{0.4\textwidth}
  	\includegraphics[width=\textwidth]{figures/cosmics/mc_300_ratio_eta.pdf}
  	\caption{Signal \ac{MC} with respect to $\eta$}
  \end{subfigure}
    \caption{Cosmic tagging efficiency with respect to kinematic variables in VR-$\mu$ data and signal \ac{MC}}
  \label{fig:cos_eff}
\end{figure}



%\begin{sidewaysfigure}[h]
%  \centering
%  \begin{subfigure}[b]{\textwidth}
%  	\centering
%  	\includegraphics[width=.18\textwidth]{figures/cosmics/wider_tag_ratio_pt.pdf}
%  	\includegraphics[width=.18\textwidth]{figures/cosmics/wider_tag_ratio_d0.pdf}
% 	\includegraphics[width=.18\textwidth]{figures/cosmics/wider_tag_ratio_z0.pdf}
% 	\includegraphics[width=.18\textwidth]{figures/cosmics/wider_tag_ratio_phi.pdf}
% 	\includegraphics[width=.18\textwidth]{figures/cosmics/wider_tag_ratio_eta.pdf}
% 	\caption{Cosmic tagging efficiency in VR-$\mu$ data}
%  \end{subfigure}

%  \begin{subfigure}[b]{\textwidth}
%  	\centering
%  	\includegraphics[width=.18\textwidth]{figures/cosmics/mc_300_ratio_pt.pdf}
%  	\includegraphics[width=.18\textwidth]{figures/cosmics/mc_300_ratio_d0.pdf}
%  	\includegraphics[width=.18\textwidth]{figures/cosmics/mc_300_ratio_z0.pdf}
%  	\includegraphics[width=.18\textwidth]{figures/cosmics/mc_300_ratio_phi.pdf}
%  	\includegraphics[width=.18\textwidth]{figures/cosmics/mc_300_ratio_eta.pdf}
%  	\caption{Cosmic tagging efficiency in signal \ac{MC}.}
%  \end{subfigure}
%    \caption{Cosmic tagging efficiency with respect to \pt, \dz, \z, $\phi$, $\eta$ in VR-$\mu$ data %(top) and signal \ac{MC}. The green shows the distribution of all muons, while the black line shows the subset of all muons that are cosmic tagged. A ratio of the two is shown below each plot.}
%  \label{fig:cos_eff}
%\end{sidewaysfigure}





\subsection{Properties of Events with Cosmic Tagged Muons}
Because all all events with a cosmic tagged muon are vetoed, a \ac{CR} with at least one cosmic tagged muon, CR-$M_{\textrm{full}}$, is defined to study cosmic events. 90\% of events in CR-$M_{\textrm{full}}$ have only one muon, and that muon is cosmic tagged. 90\% of those events find the cosmic muon on the bottom of the detector, \mb. The other 10\% of events have two reconstructed muons and they are both cosmic tagged. One event was seen that had four cosmic tagged muons, two on either side of the detector, indicating a cosmic shower. In CR-$M_{\textrm{full}}$ events, some jets are seen from pileup collisions, but additional leptons are coincident with cosmic muons. Interesting \ac{LHC} collisions are rare and so are cosmic muons, so the odds of having the two coincident in the same bunch crossing is minuscule. Generally, the cosmic muon is the most notable feature of the event and is the reason the event was triggered.

Events with two muons reconstructed from one cosmic muon have a very distinct signature. They have exactly correlated $z_{0}$, exactly anti-correlated $d_{0}$, equal and opposite $\phi$ measurement, and their $\eta$ measurements sum to 0. This can be seen in \autoref{fig:2cos}.

\question{talk about the brems rabbit hole? maybe no}

\begin{figure}[!ht]
\centering
\includegraphics[width=.48\textwidth]{figures/cosmics/v4_widetag_2_2cos_d0_d0.pdf}
\includegraphics[width=.48\textwidth]{figures/cosmics/v4_widetag_2_2cos_z0_z0.pdf}
\includegraphics[width=.48\textwidth]{figures/cosmics/v4_widetag_2_2cos_eta_eta.pdf}
\includegraphics[width=.48\textwidth]{figures/cosmics/v4_widetag_2_2cos_phi_phi.pdf}
\caption{Relationship between two cosmic tagged muons in an event. Their \dz (top left), \z (top right), $\eta$ (bottom left), and $\phi$ (bottom right) values all indicate that the two muons originate from the same cosmic. }
\label{fig:2cos}
\end{figure}


The timing distributions as measured by the MDT segments of muons in 1 and 2 cosmic tagged events can be seen in \autoref{fig:cos_timing}. In the 1 $\mu$ case, \mb has a timing distribution centered around 0, indicating that \mb is responsible for the trigger decision in these events. However, in 1-$\mu$ events in which only \mt passes baseline selections, its timing distribution is shifted negative (early w.r.t the collision), mirroring the distribution for \mt in 2-$\mu$ events. This indicates that, even in cases in which only \mt is reconstructed, the trigger decision is made based on \mb. 

\begin{figure}[!ht]
\centering
\includegraphics[width=.48\textwidth]{figures/cosmics/t0_plusphi.pdf}
\includegraphics[width=.48\textwidth]{figures/cosmics/t0_negphi.pdf}
\includegraphics[width=.48\textwidth]{figures/cosmics/t0_2cos_plusminus.pdf}
\caption{Comparison of timing distributions of positive (top left) and negative (top right) $\phi$ muons in 1 and 2 cosmic events, as well as the timing distribution of top and bottom muons in 2 cosmic events (bottom). Here, ``2 cosmic'' implies 2 muons reconstructed from 1 cosmic muon. \tavg is calculated by taking the average of the $t_{0}$ measured by all segments associated to the muon. Note that the peak at \tavg = 0 indicates a failure of the fit used to measure \tavg, these are handled separately and individual segments with \tavg = 0 do not enter the \tavg calculation.}
\label{fig:cos_timing}
\end{figure}

It takes roughly the same time for a cosmic muon to cross the width of the detector as is the bunch spacing ($\tilde 25$ ns). This means that in order to have sufficient detector information to reconstruct two muons, the two \ac{MS} signatures must be at the edges of the detector readout window. This makes one or both muons likely to have detector information read out and associated to the wrong bunch crossing. Due to the early timing of \mt, this is more likely for \mt, but occurs for both \mt and \mb. 

Each of the combined muons will pass signal selections, because high quality information exists from the \ac{ID}, but if enough \ac{MS} information is missing, the \ac{MS} segments will not be found back to back with the opposite combined muon, causing one or both of the muons to evade the cosmic tag. Ultimately, it is this mismeasurement that leads to the background in SR-$\mu\mu$. 



\subsection{Background Estimate}

Events with this measurement scheme were seen in a preliminary definition of the cosmic tag, which used the same \sigeta cut, but the \dphicos cut was reduced to $\dphicos = 0.18$. If a muon is reconstructed without direct $\phi$ measurements from \ac{RPC} or \ac{TGC} hits, its $\phi$ measurement is taken as the center of the \ac{MDT}, which has resolution of $\Delta \phi = 0.2$ (the final nominal tag value of $\dphicos = 0.25$ was expanded to include these cases). With this preliminary version of the cosmic tag, about 40 events were observed with one cosmic tagged muon, and another untagged muon. In these events, \mt was missing direct \ac{MS} $\phi$ measurements, so its segments were not back-to-back with \mb and \mb was not tagged as a cosmic. \mb however, was well measured, so \mt was cosmic tagged. An event can enter the signal region if both \mb and \mt are sufficiently mis-measured that neither can be tagged. This is sketched in \autoref{fig:cosmic_mismeasure}. A muon's cosmic tag is dependent on the opposite muon's quality, so the muon's cosmic tag status and its quality are assumed to be uncorrelated in order to make an estimate of the background.

This estimate and validation makes use of the two cosmic tags described in \autoref{tab:costag_values}, as well as an \emph{intermediate tag} which defines a muon that is not tagged by the narrow tag but is tagged by the nominal tag. Because a cosmic muon is defined using detector information on the opposite side of the detector, muons tagged using the narrow tag must have a higher quality measurement on the opposite side of the detector in order to pass the tighter cuts than those tagged with only the intermediate or full tags. Cosmic tagged muons with $\dphicos-\sigeta(\mu, \text{seg})$ values closer to the edge of the nominal cosmic tag window are more similar to those that would enter SR-$\mu\mu$ as background (which are outside of the cosmic tag window). 



\begin{figure}[!ht]
\centering
\includegraphics[width=.4\textwidth]{figures/cosmics/1badcos.png}
\includegraphics[width=.45\textwidth]{figures/cosmics/2badcos.png}
\caption{Sketches illustrating how a a 2 $\mu$ cosmic event could evade a cosmic tag. In this diagram, the thick lines represent ID tracks and MS segments, and the dashed line the CB muon measurement. A muon is tagged as cosmic if it is back to back with an MS segment. If an MS segment does not have a direct $\phi$ measurement from an RPC hit, its $\phi$ measurement is taken as the center of the MDT, which has an uncertainty of 0.2 (though the muon can be mismeasured in other ways as well, for example lacking MDT hits or resulting from a bad combination of ID track and MS track). The sketch on the left shows a 2 $\mu$ event where the red muon is cosmic tagged but the blue is not. The segments attached to the red muon are not measured well, so when we look for the red segments back to back with the blue muon (the blue circles), the segments are not in the right place and the blue muon is not cosmic tagged. However, the blue muon is well measured, so when we look back to back with the red  muon, we find the segments of the blue muon. Thus, the better quality muon is not tagged, while the poorer quality muon is. The right shows a scenario where both muons have this mismeasurement, and so neither is cosmic tagged. This is what contributes to the background in SR-$\mu\mu$.}
\label{fig:cosmic_mismeasure}
\end{figure}


To estimate the number of events entering SR-$\mu\mu$ from poorly measured 2 cosmic muon events, a scaling factor from good quality to bad quality muons, \rgood, is defined and applied to events which have one good quality and one bad quality muon, CR-$\mu\mu$-topbad. ``Good'' quality defines a muon that passes signal cuts on \nprecision, \nphi, and \chiCB, while a ``bad'' quality muon fails at least one of these cuts. For the nominal version of the estimate, it is assumed that there is one $\phi > 0$ muon (\mt) and one $\phi < 0$ muon (\mb), and \mt is the muon is scaled from poor to good quality. This is illustrated in \autoref{fig:cosmic_est}.

\rgood for the SR-$\mu\mu$ estimate is defined using muons tagged with the full cosmic tag. A validation estimate is also defined, using the narrow and intermediate cosmic tags. The signal and validation regions used in this estimate are sketched in \autoref{fig:cosmic_SRVR} and the numbers of events in each region listed in \autoref{tab:estimate_counts}. Aside from Region 2 (CR-$\mu\mu$-topbad), all other regions in the SR and VR estimates are subsets of CR-$M_{\textrm{full}}$. These regions were chosen to maintain orthogonality, while minimizing signal contamination and maximizing statistical power. In all cases, statistical errors on \rgood are computed using the Wilson interval which is recommended for values close to zero with small statistics \cite{ROOTAsymmErrors}. 

\begin{figure}[!ht]
\centering
\includegraphics[width=.8\textwidth]{figures/cosmics/SR-VR-sketch.png}
\caption{A visual representation of the CRs and VRs used to estimate and validate the background contribution to SR-$\mu\mu$ due to cosmic muons. Regions 3 and 4 are used to define \rgood. These regions were chosen to maintain orthogonality, while minimizing signal contamination and maximizing statistical power.}
\label{fig:cosmic_SRVR}
\end{figure}

\begin{table}
\centering
\begin{tabular}{ccc}
Region & $\phi < 0$ muon & $\phi > 0$ muon \\
\hline
1 & signal & signal\\ 
2 & signal & fails at least one MS quality cut\\ 
3 & signal & cosmic tagged \\ 
4 & signal & cosmic tagged and fails at least one quality cut\\ 
5 & narrow cosmic tagged & narrow cosmic tagged\\ 
6 & narrow cosmic tagged & narrow cosmic tagged and fails at least one quality cut\\ 
7 & narrow cosmic tagged & full, but not narrow, tagged \\ 
8 & narrow cosmic tagged & full, but not narrow, tagged and fails at least one quality cut\\  
\hline
\end{tabular}
\caption{A description of the regions used for the cosmic estimate and validation. Each column describes the way in which the muon deviates from a signal muon, meaning a muon is signal in all respects except for the parameter(s) listed in the table.}
\label{tab:cosmic_SRVR}
\end{table}


\begin{table}
\centering
\begin{tabular}{cccccccccc}
Region & 1    & 2 & 3 & 4 & 5 & 6 & 7 & 8\\
\hline
Event Yield & --  & 2 & 1 & 18 & 1088 & 1000 & 1947 & 2465 \\
\hline
\end{tabular}
\caption{Numbers of events in each region used for the cosmic estimate. Regions 2-4 are used to estimate SR-$\mu\mu$ (Region 1) and Region 6-8 estimate the number of events in Region 5, show in \autoref{fig:cosmic_SRVR}.}
\label{tab:estimate_counts}
\end{table}

Results of the estimate and validation are shown in \autoref{tab:estimate_results}. The non-closure in the VR can be explained by the difference in \absdz distributions in Region 5 and Region 6, shown in \autoref{fig:cos-nonclosure-plots}. The excess of low \absdz, high \chiCB muons in the weighted Region 6 compared to Region 5 implies that more muons have been created due to a poor combination of prompt track with MS track. The estimate was performed in the VR with \rgood  defined as a function of \absdz and the nonclosure was reduced from 37.7\% to 11.7\%. It is not possible to perform this estimate as a function of \absdz in the SR due to low statistics in Region 2 and Region 3. The estimate is performed without the \absdz binning and the nonclosure from the unbinned estimate in the VR is taken as an uncertainty (37\%). In the SR, seen in \autoref{fig:SR-rgood}, there are no overlapping \absdz bins in CR-$\mu\mu$-topbad and \rgood, however, adjacent bins are filled, so the extrapolation over \absdz is not large and uncertainity from this extrapolation is contained in the nonclosure systematic of the unbinned estimate. \rgood and \absdz distributions for each estimate is shown in \autoref{fig:SR-rgood} and \autoref{fig:VR-rgood} for the SR and VR, respectively.

\begin{table}
\centering{}
\begin{tabular}{ccccccc}
Region & central value & up error & down error & actual value & \% diff from actual \\
\hline
VR        & 789.9  & 34.85 & 34.4 & 1088 & 37.7\%  \\
SR        & 0.11 & 0.20  & 0.10  & --   & --   \\
\hline
\end{tabular}
\caption{Results of the background estimation strategy in the two validation regions and the signal region}
\label{tab:estimate_results}
\end{table}

\begin{figure}[!ht]
\centering
\includegraphics[width=.48\textwidth]{figures/cosmics/ratio_d0.pdf}
\includegraphics[width=.48\textwidth]{figures/cosmics/d0_ratio_binned.pdf}
\caption{Comparison of \absdz in Region 5 (orange) compared with Region 6 weighted by unbinned \rgood = 0.78 (left) and by \rgood defined as a function of \absdz (right). The nonclosure improves from 37.7\% to 11.7\% by binning \rgood in \absdz. The error shown is statistical only. There is no trend in other variables (\pt, $\eta, \phi$, \z, \tavg).}
\label{fig:cos-nonclosure-plots}
\end{figure}

\begin{figure}[!ht]
\centering
\includegraphics[width=.48\textwidth]{figures/cosmics/d0_VR_v4_rgood.pdf}
\includegraphics[width=.48\textwidth]{figures/cosmics/d0_VR_v4_2mu.pdf}
\caption{\absdz distribution of \rgood (left) and \mt in Region 6 to perform the VR estimate. To calculate the background estimate, the distributions are multiplied bin by bin and summed. The error shown is statistical only. This method is ultimately not used, in favor of an unbinned version, due to statistical limitations in the regions used for the SR estimate}
\label{fig:VR-rgood}
\end{figure}

\begin{figure}[!ht]
\centering
\includegraphics[width=.48\textwidth]{figures/cosmics/d0_SR_v4_rgood.pdf}
\includegraphics[width=.48\textwidth]{figures/cosmics/d0_SR_v4_2mu.pdf}
\caption{\absdz distribution of \rgood (left) and \mt in Region 2 to perform the SR estimate. To calculate the background estimate, the distributions are multiplied bin by bin and summed. The error shown is statistical only. It can be seen here that there are no overlapping bins between the two plots, and so this method is ultimately not used, in favor of an unbinned version.}
\label{fig:SR-rgood}
\end{figure}

\subsection{\label{sec:cos_syst}Systematic Uncertainties}

\paragraph{Muon Orientation}

The estimate is performed using the quality of \mt since muons in the upper hemisphere are expected to be more temporally marginal and mismeasured. However, the strategy can be formed with the quality of \mb. Because poor quality \mb are more rare than poor quality \mt, it is not possible to compare this strategy in the SR estimate, which is already statistically limited. However, the comparison can be made in the VR. As shown in \autoref{tab:syst-orientation}, this change produces an estimate that is consistent with the nominal estimate within statistical uncertainties. Nonetheless, a 13\% systematic uncertainty is applied from the difference between the two predictions.

The estimate described above relies on the assumption that if two muons are reconstructed from a cosmic muon, one will be on the top of the detector and the other on the bottom. 0 events were observed with one good muon and one bad muon on the same side of the detector (sign$(\phi_{0}$) = sign$(\phi_{1}$)) in both the one cosmic tag regions as well as the 0 cosmic tag regions, so contributions from same side muons are considered negligible.

\begin{table}
\centering
\small
\begin{tabular}{ccccccc}
central value  & up error & down error & actual value & \% diff from actual & \% diff from nominal\\
\hline
689.0  & 115.1 & 108.8 & 1088 & 62.3 \% & 13\% \\
\hline
\end{tabular}
\caption{Estimate in VR with bottom muon as test muon instead of top muon. CR-$\mu\mu$-topbad becomes VR-$\mu\mu$-bottombad and \rgood is defined using the quality of \mb instead of \mt.}
\label{tab:syst-orientation}
\end{table}


\paragraph{Quality parameter dependence}

Here, a ``bad'' quality muon means one that fails any one of the \nprecision or the \nphi or the \chiCB cut. To evaluate the dependence one each variable, the estimate can be performed using only one of these to define a ``bad'' quality muon, requiring the others to pass the signal requirements. However, \chiCB is dependent on the MS track hit requirements and has a small contribution to the estimate regions on its own. Only inverting only \chiCB leaves 11 events in each Region 6 and Region 8 and 0 in Region 2 and Region 4. Thus, we account for the \chiCB contribution by remaining agnostic to it in performing the estimate with the other two quality variables.  Allowing for the failure of the \chiCB cut increases the \nprecision-only and \nphi-only estimates by 2\% and 1\%, respectively. \autoref{tab:estimate_variables_VR} and \autoref{tab:estimate_variables_SR} show the results of this procedure in the VR and SR, respectively. We take the largest difference from the nominal estimate in the VR (16.5\%) to account for the quality variable dependence. 

\begin{table}
\centering
\small
\begin{tabular}{cccccccc}
Variable & central value  & up error & down error &  actual value & \% diff from actual & \% diff from nominal\\
\hline
\nphi          & 783.6 & 69.2 & 68.1 & 1088 & 38.8\% & 0.9\% \\
\nprecision    & 920.6 & 51.8 & 50.9 & 1088 & 18.2\% & 16.5\% \\

\hline
\end{tabular}
\caption{Dependence of the background estimate in the VR on each of the variables used. In each estimate, only the given quality variable is used to define a ``bad'' muon and the other must always pass. In both cases, the \chiCB is allowed to pass or fail the signal cut.}
\label{tab:estimate_variables_VR}
\end{table}

\begin{table}
\centering
\begin{tabular}{cccccc}
Variable & central value  & up error & down error & \% diff from nominal\\
\hline
\nphi          & 0 & -- & -- & 100\% \\
\nprecision    & .33  & 0.75 & 0.40 & 200\% \\
\hline
\end{tabular}
\caption{Dependence of the background estimate in the SR on each of the variables used. In each estimate, only the given quality variable is used to define a ``bad'' muon and the other must always pass. In both cases, the \chiCB is allowed to pass or fail the signal cut.}
\label{tab:estimate_variables_SR}
\end{table}


\subsection{Summary}

This estimate is dominated by statistical uncertainties in the SR estimation regions. All combined $0.11^{+0.20}_{- 0.11}$ ($^{+0.198}_{-0.104}$ stat. and 0.047 syst) events are expected in SR-$\mu\mu$ due to cosmic muons.

\section{Background to SR-$ee$}

\subsection{Fakes and Heavy Flavor Decays}

The primary background to SR-$ee$ is algorithmic fakes from the misassociation of a fake \ac{LRT} track with a real energy deposit in the \ac{EM} calorimeter (such as a photon); there is a secondary contribution from electrons from \ac{HF} decays. \ac{MC} samples of \ttbar along with photon dominated sample (described in \autoref{sec:mc}) were used to study the relative contributions. The \ttbar provides a good sample of \ac{HF} decays (though not the dominant source of \ac{HF} decays at the \ac{LHC}), yet after all of the signal requirements, the remaining high \absdz electron was the result of a photon combined with an \ac{ID} track. That this sample has many more b-jet decays than photons, yet the photons contribute a larger background even in this narrow region of phase space, indicates that algorithmic fakes will be the larger contributer of background events to SR-$ee$. Additionally, in direct studies in background \ac{MC}, such as \bbmm (at truth level, the electron and muon kinematics are the same), $Z\rightarrow \tau\tau$ (where the Z bosons decay to electrons), and \ttbar, all show exponentially falling distributions with no two lepton events that pass the \pt and \absdz cuts in the \ac{SR}, as designed. 

Further studies were performed in data, with one baseline electron passing the filter requirements and another anti-isolated lepton with no \absdz cut made. A lepton is \emph{anti-isolated} if it fails the isolation requirement made on signal leptons, meaning that there is substantial energy surrounding the lepton in the calorimeter and/or the tracker, likely inside of a jet. It was shown that while the \dpt cut is designed to remove algorithmic fakes, it is a very effective remover of anti-isolated electrons as well. Clusters associated to electrons reconstructed inside of jets are likely to have additional energy added to their clusters, increasing the cluster \pt and decreasing the \dpt. Thus, it is not possible to disentangle heavy flavor electrons from fake electrons, and the fake electrons make the dominant contribution to the background to SR-$ee$.


\subsection{Background Estimate}
Fake electrons are the dominant background to SR-$ee$ and since they are a failure of an algorithm, the two fake electrons should be uncorrelated. This assumption is used to estimate the background using an \emph{ABCD method}. The ABCD method relies divides events into four regions, shown in \autoref{fig:abcd}. The number of events in regions B, C, and D are combined to estimate the number of events in region A, the signal region:

\begin{equation}
N_A = \frac{N_B}{N_D}\times N_C
\end{equation}


\begin{figure}[!ht]
\centering
\includegraphics[width=.5\textwidth]{figures/otherbackgrounds/abcd.png}
\caption{The regions used for ABCD estimation.}
\label{fig:abcd}
\end{figure}


In the nominal estimate, region A is the signal region, with two electrons passing all signal cuts, region D has both electrons failing at least one signal cut, and regions C and D have only one electron passing all signal cuts while the other fails at least one (regions B, C, and D compose CR-$ee$-fake). For this estimate, a ``failing'' electron fails either \dpt, \chiID, or \nmiss, while a ``passing'' electron is a signal electron, passing all three cuts. The number of events in each region and the estimated number of events in SR-$ee$ is shown in \autoref{tab:abcd_ee}.

\subsection{Validation and Systematic Uncertainties}
The definitions of the ABCD regions can be changed to perform validations of the estimate and quantify systematic uncertainties. There are two ways to do this: first, the estimate can be done in slightly different ways, and the difference from the nominal estimate taken as an uncertainity; second, an estimate can be done in a validation region, where one or more signal cuts are inverted, and the estimated number of events can be compared to the actual number of events in region A. In the second scenario, one looks for \emph{closure}, that the estimate correctly estimates the number of events (within statistical uncertainties), giving evidence that the method and its assumption are sound. If this is not the case, the \emph{nonclosure}, the extent to which the estimate disagrees with the correct number of events, is evaluated and an uncertainty can be taken to cover this.

\begin{table}[htb]
\small
\begin{center}
\begin{tabular}{lcc}
Region     & Nominal            & Only \dpt   \\
\hline
D Observed & 9068 				& 1440 			\\
C Observed & 77   				& 28   		\\
B Observed & 54   				& 19   		\\
A Estimate & $0.459 \pm 0.082$	& $0.37 \pm 0.11$ 	\\
\hline
\end{tabular}
\caption{Results of the ABCD estimate for the nominal SR-$ee$ estimate, the number of events in each region are shown as well as the estimate for A. The uncertainties are statistical only and using Poisson statistics.}
\label{tab:abcd_ee}
\end{center}
\end{table}

Validations are performed by estimating the number of events in SR-$ee$ in different ways. For example, only the \dpt, the most effective fake discriminator, is used as the ``passing''/``failing'' variable. The results of this estimate are also shown in \autoref{tab:abcd_ee}, the results are consistent within statistical uncertainties but the nominal estimate is more precise.

The second kind of validations are performed in two regions: one that enhances the fake contribution, and a second that enhances the \ac{HF} contribution. First, in a fake enhanced region, VR-$ee$-fake, where the \dpt cut is inverted and the estimate is performed with \chiID and \nmiss together as the ``passing''/``failing'' variables. This changes the estimate to predicting the number of fake electrons (failing \dpt) from regions with electrons that are more fake (failing \dpt and one or both of \chiID and \nmiss). In this region, 1440 events are observed, and 1356 $\pm$ 49 are predicted. Even though this result is correct within the statistical uncertainties, the difference between the central values (6.2\%) is taken as a nonclosure and applied as a systematic uncertainty on the \ac{SR} estimate.

Next, a validation is done in a \ac{HF} enhanced region, VR-$ee$-fake-hf, which is identical to CR-$ee$-fake with the additional requirement that at least one electron is \emph{anti-isolated} (fails the isolation cut). This additional requirement reduces the statistics in the region quite a bit, so the electron cuts are loosened to $\pt>50\gev$, $\absdz > 2$ mm, and instead of the usual \dpt cut at -0.5, they must satisfy $\dpt>-0.9$. Electrons in dense environments are more likely to have extra energy added to their clusters or have the wrong track associated, so the \dpt cut must be loosened to probe the subdominant \ac{HF} contribution to the SR-$ee$ background. In this region, $23.5\pm1.9$ events were predicted, and 26 were seen. Again, the results are consistent within uncertainties, with only a 11\% difference in central values. This difference is taken as a systematic uncertainty.

\subsection{Summary}

This estimate is dominated by statistical uncertainties with additional, conservative, systematic uncertainties taken, giving a final estimate of $0.46 \pm 0.10$ (0.082 stat. and 0.058 syst).

\section{Background to SR-$e\mu$}

\subsection{Fake Background}
The background to SR-$e\mu$ is very similar to the background in SR-$ee$ and is estimated in a similar way. By tagging a lepton that fails either isolation or quality requirements and studying the properties of the other, probe lepton, the main contributing background (either fake or \ac{HF}) is determined. Of the probe leptons, 100 pairs failed some signal requirements, but none only failed isolation, indicating that as in the case of SR-$ee$, algorithmic fakes are the dominant background to SR-$e\mu$. 

\subsection{Background Estimate}
As in SR-$ee$, the two fake leptons in the event should be uncorrelated and so an ABCD method is used to estimate the background. A ``failing'' electron (as in SR-$ee$) is one that fails any one of the \dpt, \chiID, or \nmiss requirements, a ``failing'' muon fails any one of the  \chiID, \chiCB, \nmiss, \nprecision, or \nphi requirements, and in both cases ``passing'' indicates a signal electron or muon.

However, when performing this estimate, the B region (electron passes, muon fails) has only 1 event, and the C region (muon passes, electron fails) has 0 events. This result cannot be used to calculate a background estimate, but it can be used to place an upper bound by setting the number of events in the C region to 1 event: the total number of events in A must be less than $0.012 \pm 0.017$, where the uncertainty is statistical only. In order to increase the statistical power, different combinations of ``passing'' and ``failing'' leptons were required to only pass the baseline kinematic cuts of $\pt > 50 \GeV$ and $\absdz > 2$ mm. Allowing both passing and failing leptons to only meet the signal requirements allowed 1 event in the C region, enabling a background estimate of $0.007^{+0.018}_{-0.009}$. Statistical uncertainties are quoted using the Wilson interval for the C/D ratio summed in quadrature with the Poisson uncertainty on the number of events in the B region. This is taken as the nominal upper bound and the full result of this loosening can be seen in \autoref{tab:abcd_loose_em}.

\begin{table}[htb]
\small
\begin{center}
\begin{tabular}{lccc}
Estimate Region     & signal \pt, \absdz cuts on all $\ell$ & signal \pt, \absdz cuts on passing $\ell$ & no signal \pt, \absdz cuts \\
\hline
D Observed 								& 81 					& 139 					& 138 		\\
C Observed 								& 0   					& 0   					& 1  		\\
B Observed 								& 1   					& 1   					& 1   		\\
A Estimate 								& $< 0.012 \pm 0.017$ 	& $< 0.007 \pm 0.010$ 	& $0.007^{+0.018}_{-0.009}$ \\
%TH NOTE: Numbers are updated with v5.1 ntuples
%Which must pass signal \dz and \pt? 	& fail and pass  		& pass only 			& neither \\
\hline
\end{tabular}
\caption{Results of the ABCD method in the $e\mu$ channel in which the \dz and \pt requirements are selectively loosened from 65 to 50 \gev, and 3 to 2 mm. The first column shows the results without any loosening, the second shows the results with the loosening applied only to the failing leptons, and the final column shows the results with the loosening applied in all regions. Uncertainties are statistical only. For the upper bound results, the value is obtained by setting the C region to 1 event, and the uncertainties are calculated using Poisson statistics. In the final case, the full calculation can be done.}
\label{tab:abcd_loose_em}
\end{center}
\end{table}

\subsection{Validation and Systematic Uncertainties}

As in the case of SR-$ee$, validations are performed enhancing the fake or \ac{HF} contributions. VR-$e\mu$-fake again inverts the most powerful fake discriminators, \dpt for electrons and \chiCB for muons with the loosened \pt and \absdz cuts. In this region, 2 events are observed in the A region compared to $1.9^{+1.8}_{-1.0}$. While these agree within the very large uncertainties, a 7.8\% nonclosure systematic uncertainty is taken from the difference in central values. 

Then, VR-$e\mu$-fake-hf is defined requiring at least one anti-isolated lepton with loosened \pt and \absdz cuts. To increase statistics, the \dpt cut is again loosened to -0.9 and the cuts on \nprecision and \nphi are removed. Here, one event is observed in the A region, and $0.38^{+0.37}_{-0.32}$ events are predicted. To attempt this estimate another way with more statistical power, the estimate is performed again in this region, this time remaining agnostic to whether or not the lepton is isolated. This results in an estimate of $2.6^{+2.0}_{-1.4}$, while 5 events are observed. These numbers are again consistent within their substantial statistical uncertainties, but a conservative 92\% nonclosure uncertainty is taken to account for the difference between the central values.

\subsection{Summary}
This region is extremely statistically limited such that a full background estimate is not possible, and so an upper limit is set. Less than 
$0.007^{+0.019}_{-0.011}$ ($^{+0.018}_{-0.009}$ syst. and 0.006 stat.) background events are expected in SR-$e\mu$.
 
\section{Negligible Backgrounds}
Since this analysis is built on the assumption that there should be 0 background events in all signal regions, it is extremely important to ensure that all possible backgrounds are quantified and accounted for. Three different backgrounds were studied and determined to be negligible, meaning that they contribute $\mathcal{O}10^{-4}$ events to the relevant \ac{SR}. 

\subsection{Material Interactions}

Material interactions define events with dilepton pairs that come from interaction with the material of the \ac{ATLAS} detector. In other searches for long lived particles in \ac{ATLAS} where displaced vertices (\acp{DV}) are required, \ac{DV}s are reconstructed by extrapolating their tracks to their intersection, and it can be determined if verticies originate from material. In this analysis, there is no vertex so such a veto cannot be used. In order to study material interactions, events with leptons associated to displaced vertices from material were studied. It was found that in the cases of both electrons and muons, events with an \ac{DV}, where one track is associated to a lepton, the \dR between the lepton and the other track in the \ac{DV} was less than 0.2. After making the requirement that $\dR_{\ell, \text{track}} < 0.2$, no \acp{DV} with signal leptons were found. In the case of the background to one of the \acp{SR}, the second track in the \ac{DV} would be associated to a lepton. After this cut is made, this background is negligible to all 3 \ac{SR}.

\subsection{Fake Muons}
Fake muons were estimated using an ABCD method similar to SR-$ee$ and SR-$e\mu$. Muons were selected using the baseline requirements, plus asking that be isolated, and not cosmic tagged. All remaining cuts used for defining signal muons were used to define ``passing'' muons. Using this strategy, only 6 events were seen in the D region, and 0 in either B or C. In the SR-$e\mu $fake estimate, the ratio of passing to failing muons is less than 1\%. Given this, the probability for two fake muons to pass our selections should be less than 0.01\%, making this background negligible ($<0.0006$ events) and we consider it to be negligible relative to cosmic muons.

\subsection{Heavy Flavor Muons}

Heavy flavor electrons and muons cannot be disentangled from algorithmic fakes in SR-$ee$ and SR-$e\mu$ and are estimated n conjunction with fakes. Fakes are negligible in SR-$\mu\mu$, so a fake estimate must be performed separately. \ttbar \ac{MC} sees very few single muons from decays of b-jets that have \pt and \absdz cuts that enter the \ac{SR}, so this background is expected to be small. An ABCD method cannot be used for this estimate because it cannot be assumed that the probability to find an isolated muon is not uncorrelated in a 2 muon event.

First, data was checked for anti-isolated muon events in CR-$\mu\mu$-hf, a region with two signal muons but allowing for one of the muons to be anti-isolated. No events were observed. Loosening the kinematic cuts to $\pt > 50 \GeV$ and $\absdz > 2$ mm also found zero events. Finally, since the muon trigger and thus muon \texttt{DRAW\_RPVLL} filter only requires one muon, it was possible to study events with one baseline muon and one muon with $\absdz > 0.5$ mm. One event was observed in this region. Then, extrapolation factors into the \ac{SR} were defined from the same dataset. The extrapolation factor  from the extremely loosened region into SR-$\mu\mu$ for both the \absdz and \pt of both muons was calculated to be $0.00013 \pm 0.00013$. This means that $1.3 \times 10^{-4}$ are expected in SR-$\mu\mu$ from muons from \ac{HF}, a negligible contribution.


\section{Summary}

In all, $<1$ events are expected in each \ac{SR}. \autoref{tab:bkg-summary} summarizes the background estimates and their uncertainties.

\begin{table}
\centering
\begin{tabular}{lccc}
Region 			 & SR-$ee$ 			& SR-$\mu\mu$ 				& SR-$e\mu$ \\
\hline
Total background & $0.46 \pm 0.10$ 	& $0.11 ^{+0.20}_{-0.11}$	& $0.007^{+0.019}_{-0.011}$\\
\hline
Fakes 			 & $0.46 \pm 0.10$ 	& - 						& $0.007^{+0.019}_{-0.011}$\\
Cosmics 		 & - 				& $0.11 ^{+0.20}_{-0.11}$ 	& - \\
\hline
\end{tabular}
%%%%
\caption{Summary table of the background estimate and uncertainty in each \ac{SR}.}
\label{tab:bkg-summary}
\end{table}






\chapter{Signal Systematics}
\label{chap:systematics}

In order to use this analysis to make a statement about a potential \ac{BSM} model, the extent to which the signal \ac{MC} correctly simulates the real physical environment must be evaluated. Differences between data and \ac{MC} are studied in order to avoid underestimating or overestimating the expected number of \ac{BSM} events that could have been seen in the data.

Where possible, the \ac{MC} is corrected to better represent the data using \emph{scale factors}, and in other cases systematic uncertainties are applied to the final result. Uncertainties are also evaluated on the scale factors. A list of all of the systematic uncertainties applied in the interpretation are listed in \autoref{tab:siguncertainties}. The value listed in the table describes how much varying the efficiency of a given parameter changes the final signal yield. 

The dominant source of systematic uncertainty on the signal \ac{MC} in this analysis comes from the efficiency for selecting displaced leptons, for which there is no data to compare to \ac{MC}. As a result, this is evaluated in several steps: first the trigger, reconstruction, and selection efficiencies are compared for prompt leptons in the same physics process in data and \ac{MC}; then the tracking efficiency is compared between signal muons and cosmic muons, which compares different physical phenomena that result in \pt, high \absdz tracks; finally, the lepton reconstruction efficiency is studied with respect to displacement is studied in \ac{MC} only and a conservative additional uncertainty to account for any missed effects in data is measured. Other event-level systematic uncertainties are applied to account for mismodeling of pileup and theory assumptions made during \ac{MC} generation. There are many standard systematic uncertainties derived by \ac{ATLAS}, for example the jet energy measurements or the sagitta measurements of muons, that do not have a large impact on this analysis with its nonstandard physics objects.

\begin{table}[htb]
\small
\begin{center}
\begin{tabular}{lcc}
Uncertainty Source & Uncertainty [\%] (\selec/\smu)  &  Uncertainty [\%] (\stau)  \\
\hline
Statistical				& 2-46    & 2-100                 \\
Cross Section 			& 2-5     & 2-5                 \\
Tracking				& 2-15    & 11-14                 \\
Muon Trigger			& 1-4     & 4            \\
Muon Selection	        & 3-15    & 20-37          \\
Electron Selection      & 0.5-2   & 1-2    \\
Electron Trigger	    & 0       & 0       \\
Lepton Displacement  	& 1-18    & 2-26         \\
Pileup Modeling     	& 7       & 7            \\
Other theory         	& 0-5     & 0-5              \\
\texttt{DRAW\_RPVLL} Filter Efficiency		& 1.5     & 1.5          \\
Luminosity				& 2       & 2          \\
\hline
\end{tabular}
\caption{Table describing statistical and systematic uncertainties impacting \selec, \smu and \stau efficiencies. Systematics in this table are defined as the difference varying each parameter makes in the final signal yield.} 
\label{tab:siguncertainties}
\end{center}
\end{table}


\section{Displaced Lepton Reconstruction}

\subsection{Prompt Lepton Reconstruction}
Z bosons can decay into two electrons or two muons. This process is well modeled in \ac{MC}, and easy to identify in data as the invariant mass of the two leptons should equal the Z boson mass (within resolution effects). The invariant mass of the leptons provides a way to identify a lepton as a real lepton that should pass all of the trigger, reconstruction, identification, and selection algorithms. A \emph{tag-and-probe} analysis \emph{tags} events by finding two lepton candidates with invariant mass near the Z boson peak, and then the \emph{probe} leptons are used to measure the selection efficiency by asking that one of the leptons pass a given requirement. This method is used to evaluate trigger, reconstruction, and selection efficiencies of prompt electrons and muons.

The difference between data and \ac{MC} can be thoroughly studied, so scale factors are applied to correct the leptons in MC. A scale factor is defined as:

\begin{equation}
\text{scale factor} = \frac{\text{efficiency in data}}{\text{efficiency in MC}}
\end{equation}
Then the statistical uncertainty on this value is evaluated and applied as an additional uncertainty on the signal. In general, \ac{MC} estimates higher efficiencies than are seen in data. In particular, the \ac{MC} assumes a perfectly aligned detector with all subsystems working perfectly, which is not true in practice, so variables that correlate multiple subdetectors, like electron \dpt or muon \chiCB, contribute to a difference between data and \ac{MC}. A 90\% scale factor means that \ac{MC} overestimates the selection efficiency by 10\%.   

\ac{ATLAS} centrally defines electron and muon scale factors for prompt electrons and muons, but due to the special triggers and selection criteria used in this analysis, custom scale factors are required. For electrons, this analysis uses photon triggers as well as a bug-fixed electron reconstruction, as well as a custom identification and non-standard selection criteria, so all scale factors must be derived specifically for this analysis. For muons, a nonstandard trigger is used, but the reconstruction is standard (except for the tracking, evaluated separately) and the only change to the identification criteria is the removal of the cut on pixel hits, which does not impact prompt muons, so central scale factors are used, with additional selection and special trigger scale factors derived for this analysis.


\subsubsection{Electrons}

For electrons, $Z\rightarrow ee$ events are used to evaluate trigger scale factors, and a single scale factor for reconstruction, identification, and selection as all electrons coming from the Z are real electrons and should pass the trigger as well as all selection criteria.

For both \texttt{HLT\_2g50\_loose} and \texttt{HLT\_g140\_loose} triggers, the efficiency is defined as the number of electrons passing the trigger divided by the number of electrons passing the offline identification criteria. This is done for single electrons, and in the case of the 2 electron trigger, the results are summed in quadrature. It was found that above the trigger threshold, the trigger efficiency in both data and \ac{MC} is very close to 100\%, so this scale factor and its associated uncertainty is considered negligible.

The reconstruction, identification, and selection scale factors are evaluated together, by measuring the efficiency for a reconstructed electron candidate to pass the final signal selection. Electron candidates are cluster-track combinations that will get eventually identified as a photon, converted photon, or electron (or none of these). The track reconstruction is studied separately and the cluster reconstruction efficiency at the signal \pt is nearly 100\%. The scale factors for electron selection are defined as a function of $E_{T}$ and $\eta$ and around 98\% except for in the region between the barrel and endcap (around $|\eta|$ = 1.5) where it drops to 90\%. The statistical uncertainty varies from 1-3\%. 


\subsubsection{Muons}
$Z\rightarrow \mu\mu$ data and \ac{MC} are used to define additional scale factors for the \texttt{HLT\_mu60\_0eta105\_msonly} trigger. The standard trigger scale factors correct for many features of the \ac{MS} and additional corrections are derived and applied on top of them for this analysis. Events are required to pass a \ac{MET} trigger to ensure an unbiased data sample and have two muons within 10 GeV of the mass of the Z boson. The trigger efficiency is then defined as the number of muons passing the \texttt{HLT\_mu60\_0eta105\_msonly} trigger divided by the number of baseline muons. Similarly, selection scale factors are defined by requiring that one muon pass all signal selections (except the \absdz cut), then the efficiency for the second baseline muon to pass the same signal selection cuts is evaluated.

The scale factors are larger for muons than for electrons because many of the structural features of the \ac{MS} are not well modeled in \ac{MC}. The statistical errors are also around 3\%.


\subsection{Tracking}

This analysis relies on \ac{LRT} in order to reconstruct leptons with high \pt and high \absdz. 
Tracks from cosmic muons have high \pt and high \absdz and can be used to measure the tracking efficiency in data. Cosmic muons are tagged as muons with \ac{MS} activity on the other side of the detector, so the cosmic muon must have also passed through the \ac{ID} leaving a track behind. Provided we make some kinematic selections, the existence of the track back-to-back with the cosmic should be solely dependent on the \ac{LRT} efficiency.

A tag-and-probe analysis is performed here as well, by tagging a cosmic muon and looking for tracks back-to-back with it in a narrow $\Delta R_{\text{cos}}$ cone. Then compare this to a tag-and-probe analysis in signal \ac{MC} by looking for a track in a narrow $\Delta R$ cone nearby a truth muon. 

There are several important kinematic selections that must be made to ensure the collection of tracks are similar and to correct for the different kinematic distributions between cosmics and data (see \autoref{fig:cos_eff}). 

First, to ensure all the hits in the track will be read out with the event, $\phi > 0$ muons must have negative \tavg (early w.r.t collision), and $\phi < 0$ muons must have positive \tavg (late w.r.t collision). Making this cut shows a flat reconstruction efficiency w.r.t. \tavg. All signal muons have very central timing, with \ac{ID} signatures created before \ac{MS} signatures, so this problem does not apply. The impact of the timing cut can be seen in \autoref{fig:cos_sys_t0}. 

Second, the cosmic ray muon passes through the detector in an approximately straight line through the \ac{ID}, so the \dz is simply the distance the cosmic muon was from the PV. This is not the case for signal muons, whose \dz does not measure a point the muon has gone through, but an extrapolation backwards to the PV. This means that signal tracks with the same \dz can have very different properties, such as number of hits on track. To correct for this, we require the $R_{\textrm{decay}}$ and \dz to fall between the same two silicon layers. A sketch of this difference can be seen in \autoref{fig:lrt_sig_sketch}. 

Finally, cosmic muons have a much wider \z range than signal muons, which induces an $\eta$ dependence. So we require cosmic muons to have $\absz < 120$ mm, to harmonize with signal muons. Additionally, both signal and cosmic muons must have $|\eta| < 1.05$ in order to be triggered. A full list of cuts made on tag muons and probe tracks can be found in \autoref{tab:lrt-mu-cuts} and \autoref{tab:lrt-track-cuts}.

The remaining difference between cosmic and signal muons is the correlation between \pt and \absdz in signal. Thus, the cosmic muon \pt distribution is reweighted in each \absdz bin to match the signal distribution. Then, the ratio of the efficiencies as a function of \absdz is determined per lepton. The maximum difference, 8\%, is taken as the systematic uncertainty per lepton, then summed in quadrature for the two leptons in the event resulting an 11\% event-level systematic. Tracking efficiency is assumed to be symmetric around the detector and that after GSF tracking, electron tracking and muon tracking have equivalent efficiency, motivated by \autoref{fig:trk_el_mu}. 



\begin{figure}[htbp]
\centering
\includegraphics[width=.3\textwidth]{figures/LRT_systs/cosmics_phil0_eff_t0.pdf}
\includegraphics[width=.3\textwidth]{figures/LRT_systs/cosmics_phig0_eff_t0.pdf}
\includegraphics[width=.3\textwidth]{figures/LRT_systs/cosmics_eff_t0.pdf}
\caption{\ac{LRT} efficiency measured with various $\phi$ and \tavg cuts. The right shows $\phi < 0$ muons with no \tavg cut, the center shows $\phi > 0$ muons with no \tavg cut, and the left shows the final selections, with $\phi > 0$ requried to have \tavg < 0 and $\phi < 0$ \tavg > 0. This gives a consistent readout configuration and an approximately flat efficiency w.r.t \tavg. In particular, there $\phi < 0$ muons with negative timing that result in an artificially low efficiency, likely due to incomplete readout.}
\label{fig:cos_sys_t0}
\end{figure}


\begin{figure}[htbp]
\centering
\includegraphics[width=.6\textwidth]{figures/LRT_systs/cos_sig_LRT.png}
\caption{An illustration of the difference in \dz measurements between cosmic (left) and signal muons. The \dz of a cosmic muon is always measured just before its first hit, whereas for a signal muon, the first hit can come far after the \dz. The blue x's represent \ac{ID} hits and the red lines represent muon tracks.}
\label{fig:lrt_sig_sketch}
\end{figure}


\begin{figure}[htbp]
\centering
\includegraphics[width=.48\textwidth]{figures/LRT_systs/stlrt_compare_elmu_d0.pdf}
\includegraphics[width=.48\textwidth]{figures/LRT_systs/stlrt_compare_elmu_pt.pdf}
\includegraphics[width=.48\textwidth]{figures/LRT_systs/stlrt_compare_elmu_eta.pdf}
\includegraphics[width=.48\textwidth]{figures/LRT_systs/stlrt_compare_elmu_phi.pdf}
\caption{Tracking efficiency for electrons and muons in signal MC (all lifetimes of 300 GeV \selec or \smu). These plots justify the assumpution that tracking efficiency is the same between electrons and muons and symmetric about the \ac{ID} volume.}
\label{fig:trk_el_mu}
\end{figure}

\begin{table}
\centering
\begin{tabular}{c}
Cut on cosmic muon \\
\hline
$\pt > 50 \GeV$ \\
$\absdz > 3$mm \\
$|\eta| < 1.05$ \\
$\absz < 120$ mm \\
$\tavg > 0$ if $\phi_{\mu} < 0$  \\
OR $\tavg > 0$ if $\phi_{\mu} > 0$ \\
\hline
\end{tabular}
\quad
\begin{tabular}{c}
Cut on truth muon\\
\hline
$\pt > 50 \GeV$ \\
$\absdz > 3$mm \\
$|\eta| < 1.05$ \\
parent is a \smu \\
\dz and $R_{\textrm{decay}}$ between\\
the same silicon layers\\
\hline
\end{tabular}
\caption{Cuts on tag muons. Cosmic muons (right) and truth signal muons (left).}
\label{tab:lrt-mu-cuts}
\end{table}

\begin{table}
\centering
\begin{tabular}{c}
Cut on cosmic muon \\
\hline
$\pt > 30 \GeV$ \\
$\Delta R_{\textrm{cos}}= \sqrt{ (\Delta \phi - \pi)^{2} + (\Sigma \eta)^{2}} < 0.3$ \\
$|d_{0, \textrm{track}} - d_{0, \mu}| < 20$ \\
$|z_{0, \textrm{track}} - z_{0, \mu}| < 20$ \\
\hline
\end{tabular}
\quad
\quad
\begin{tabular}{c}
Cut on truth muon\\
\hline
$\pt > 30 \GeV$ \\
$\Delta R = \sqrt{ (\Delta \phi)^{2} + (\Delta \eta)^{2}} < 0.05$ \\
 \\
 \\
\hline
\end{tabular}
\caption{Cuts on probe \ac{ID} tracks. In data (right) and signal (left).}
\label{tab:lrt-track-cuts}
\end{table}



\begin{figure}[htbp]
\centering
\includegraphics[width=.48\textwidth]{figures/LRT_systs/compare_pt_z0120_Rgd0_timing_idcuts_2dweight.pdf}
\includegraphics[width=.48\textwidth]{figures/LRT_systs/compare_d0_z0120_Rgd0_timing_idcuts_2dweight.pdf}
\includegraphics[width=.48\textwidth]{figures/LRT_systs/compare_eta_z0120_Rgd0_timing_idcuts_2dweight.pdf}
\includegraphics[width=.48\textwidth]{figures/LRT_systs/compare_phi_z0120_Rgd0_timing_idcuts_2dweight.pdf}
\caption{A comparison of tracking efficiency in cosmics data and signal MC (all masses and lifetimes can be used to due the eventual \pt and \dz binning) with respect to \pt (top left), \dz (top right), $\eta$ (bottom left), and $\phi$ (bottom right). The \ac{MC} efficiency is shown in purple, the data efficiency in black, and the ratio between the efficienceis is shown in gray.}
\label{fig:lrt_eff_comp}
\end{figure}




\subsection{Lepton Displacement}

The effect of displacement on lepton reconstruction cannot be directly compared between data and \ac{MC}. An additional uncertainty is defined in \ac{MC} to account for reconstruction effects potentially not captured by the previous two systematic uncertainties. It is taken as the deviation in the reconstruction efficiency as a function of \absdz relative to prompt muons.
This uncertainty is meant to capture any discrepancy in \ac{MS} track or electron shower shape parameters at high \absdz, as well as changes in how they are combined with the \ac{ID} track. This uncertainty is extremely conservative as there is no way to determine if the same fluctations are seen in data.


To determine the systematic uncertainty due to displaced reconstruction, the baseline or signal reconstruction/selection efficiency is divided by the tracking efficiency in order to separate any inefficiences from \ac{LRT}. Then, the ratio of each high \dz bin relative the prompt bin (0--3 mm) is taken, results of this are shown in Fig.~\ref{fig:disp_systs}. The uncertainty is assigned to each lepton and they are summed in quadrature to get an event level systematic. 


This uncertainty for muons is quite small, while it is much more substantial for electrons. Electrons are identified using a likelihood, which is trained with \dz as a discriminating variable. \absdz was removed from the cuts performed offline, but the LH was not retrained, and so there as a larger relic \dz dependence and the displacement uncertainty is much larger.


\begin{figure}[htbp]
\centering
\includegraphics[width=.48\textwidth]{figures/disp_systs/signal_effcompare_d0_el_300.pdf}
\includegraphics[width=.48\textwidth]{figures/disp_systs/signal_effwrttrack_d0_el_300.pdf}
\caption{Electron selection efficiencies with respect to truth (left) and with respect to the tracking efficiency (right). Made from a 300 GeV \slep signal samples with lifetimes between 0.01 ns-1 ns. The denominator of the efficiency is truth electrons from \selec with $\pt > 65 \GeV$ and $|\eta| < 2.5$, and the numerator is truth matched and signal (or baseline) quality tracks or leptons.}
\label{fig:disp_el}
\end{figure}

\begin{figure}[htbp]
\centering
\includegraphics[width=.48\textwidth]{figures/disp_systs/signal_effcompare_d0_mu_300.pdf}
\includegraphics[width=.48\textwidth]{figures/disp_systs/signal_effwrttrack_d0_mu_300.pdf}
\caption{Muon selection efficiencies with respect to truth (left) and with respect to the tracking efficiency (right). Made from 300 GeV \slep signal samples with lifetimes between 0.0 1ns-1 ns. The denominator of the efficiency is truth muons from \smu with $\pt > 65 \GeV$ and $|\eta| < 2.5$, and the numerator is truth matched and signal (or baseline) quality tracks or leptons.}
\label{fig:disp_mu}
\end{figure}

\begin{figure}[htbp]
\centering
\includegraphics[width=.48\textwidth]{figures/disp_systs/signal_sf_d0_mu_300.pdf}
\includegraphics[width=.48\textwidth]{figures/disp_systs/signal_sf_d0_el_300.pdf}
\caption{Fractional systematic uncertainties defined for muons (left) and electrons. The value of each bin is defined as 1 minus the ratio of the selection efficiency with respect to tracking efficiency of the given bin to the same value of the prompt (0-3 mm) bin. These are defined in 300 GeV \slep signal samples with lifetimes between 0.01 ns-1 ns. It was confirmed that the trends are consistent across various \slep masses.}
\label{fig:disp_systs}
\end{figure}


%\begin{figure}[htbp]
%\centering
%\includegraphics[width=.48\textwidth]{figures/disp_systs/signal_effcompare_R_el_300.pdf}
%\includegraphics[width=.48\textwidth]{figures/disp_systs/signal_effcompare_pt_el_300.pdf}
%\includegraphics[width=.48\textwidth]{figures/disp_systs/signal_effcompare_eta_el_300.pdf}
%\includegraphics[width=.48\textwidth]{figures/disp_systs/signal_effcompare_phi_el_300.pdf}
%\caption{Electron selection efficiencies vs $R_{\textrm{decay}}$ (top left), \pt (top right), $\eta$ (bottom left), and $\phi$ (bottom right). Plots are made from 300 GeV slepton signal samples with lifetimes between 0.01-1ns. The denominator of the efficiency is truth muons from sleptons with \pt > 65 GeV and $|\eta|$ < 2.5, and the numerator is truth matched and signal (or baseline) quality tracks or leptons.}
%\label{fig:effs_el}
%\end{figure}

%\begin{figure}[htbp]
%\centering
%\includegraphics[width=.48\textwidth]{figures/disp_systs/signal_effcompare_R_mu_300.pdf}
%\includegraphics[width=.48\textwidth]{figures/disp_systs/signal_effcompare_pt_mu_300.pdf}
%\includegraphics[width=.48\textwidth]{figures/disp_systs/signal_effcompare_eta_mu_300.pdf}
%\includegraphics[width=.48\textwidth]{figures/disp_systs/signal_effcompare_phi_mu_300.pdf}
%\caption{Electron selection efficiencies vs $R_{\textrm{decay}}$ (top left), \pt (top right), $\eta$ (bottom left), and $\phi$ (bottom right). Plots are made from 300 GeV slepton signal samples with lifetimes between 0.01-1ns. The denominator of the efficiency is truth muons from sleptons with \pt > 65 GeV and $|\eta|$ < 2.5, and the numerator is truth matched and signal (or baseline) quality tracks or leptons.}
%\label{fig:effs_mu}
%\end{figure}



%Small trends in \dz can be seen in some shower shape variables, particularly in ERatio and RHad, though not enough to explain the full discrepancy from the prompt efficiency. In muons, trends can be seen in the \ac{MS} track $\chi^{2}$ as well as the Eloss, and in electrons trends can be seen in $f_{1}$ and ERatio. We checked for, but did not find, trends in \dz with respect to other \ac{MS} track or shower shape variables. 


%\begin{figure}[htbp]
%\centering
%\includegraphics[width=.48\textwidth]{figures/disp_systs/m_signal_MStrack_chi2_profile.pdf}
%\includegraphics[width=.48\textwidth]{figures/disp_systs/m_signal_MStrack_ELoss_profile.pdf}
%\caption{Muon quality variables with trends with respect to \absdz, \ac{MS} track $\chi^{2}$ (left) and Eloss. Taken from a 300 GeV signal sample with lifetimes between 0.01ns-1ns.}
%\label{fig:profs_mu}
%\end{figure}

%\begin{figure}[htbp]
%\centering
%\includegraphics[width=.48\textwidth]{figures/disp_systs/e_signal_f1_profile.pdf}
%\includegraphics[width=.48\textwidth]{figures/disp_systs/e_signal_Eratio_profile.pdf}
%\caption{Electron shower shape quality variables with trends with respect to \absdz, $f_{1}$ (left) and ERatio. Taken from a 300 GeV signal sample with lifetimes between 0.01ns-1ns.}
%\label{fig:profs_el}
%\end{figure}


\subsection{Other Sources of Uncertainty}

\subsubsection{Pileup Modeling}
When \ac{MC} is generated, particularly when it is generated during the course of the run, the actual pileup distribution of the events from the \ac{LHC} is not known. This is corrected through a process called \emph{pileup reweighting}, where a more realistic pileup profile is added to \ac{MC} events. The change in number of signal events when the pileup profile is varied is taken as a systematic uncertainty, 2\%.

\subsubsection{Theoretical Uncertainties}

Additional uncertainties are taken for the renormalization and factorization scales that are used to generate the physical processes in \ac{MC}. These impact both the cross section measurement and the final lepton kinematics. Both scales are varied, impact on the final results quantified, and the range of variation is taken as an uncertainty, in this analysis about 5\%. 

\subsubsection{Luminosity Measurement}
\ac{ATLAS} measures luminosity using dedicated detectors and calibrations (discussed in \autoref{sec:lumi}. The uncertainty on this measurement contributes a 2\% uncertainty to the analysis, as it impacts the normalization of the signal yield from \ac{MC}.






\cleardoublepage
\part{Results}
\chapter{Results}

\section{Signal Yield}

0 events are observed in each signal region, in full agreement with the background estimate of fewer than 1 event per \ac{SR}. The result of each cut made in each \ac{SR} can be seen in \autoref{tab:data_cutflow_srmm}, \autoref{tab:data_cutflow_sree}, and \autoref{tab:data_cutflow_sremu} for SR-$\mu\mu$, SR-$ee$, and SR-$e\mu$, respectively. Also in agreement with the expectation, most powerful background discriminators are the \dpt cut for electrons and cosmic veto for muons. 

This analysis was designed to be as general as possible, so the results can be applied, with some care, to any new physics signature that results in 2 leptons with high \absdz and high \pt.  

\begin{table}[htb]
\begin{center}
%\resizebox{\columnwidth}{!}{
\begin{tabular}{l  c } 
Cut & data yield\\
\hline
Pass trigger and at least 2 baseline leptons & 65484.000 $\pm$ 255.898\\
2 leading leptons are electrons & 19419.000 $\pm$ 139.352\\ 
\pt $> 65$ GeV & 14061.000 $\pm$ 118.579\\
\absdz $> 3$ mm & 11589.000 $\pm$ 107.652\\
both electrons pass isolation & 9220.000 $\pm$ 96.021\\
\dpt $\geq$ -0.5 & 7.000 $\pm$ 2.646\\
\chiID $< $ 2 &  2.000 $\pm$ 1.414\\
\nmiss $\leq 1$ & 0.000 $\pm$ 0.000\\
\dRll $> 0.2$ &  0.000 $\pm$ 0.000\\ 
\hline
\end{tabular}
\caption{Cutflow for SR-$ee$ for Run 2 data.}
\label{tab:data_cutflow_sree}
\end{center}
\end{table}

\begin{table}[htb]
\begin{center}
\begin{tabular}{l  c } 
Cut & data yield\\
\hline
Pass trigger and at least 2 baseline leptons & 65484.000 $\pm$ 255.898\\
2 leading leptons are muons & 45845.000 $\pm$ 214.114 \\ 
$\pt> 65$ GeV & 35607.000 $\pm$ 188.698\\
\absdz $ > $ 3 mm & 35326.000 $\pm$ 187.952\\
both muons pass isolation & 35204.000 $\pm$ 187.627\\
muons pass cosmic veto & 2.000 $\pm$ 1.414\\
\tavg$ <$ 30 & 2.000 $\pm$ 1.414\\
\chiID $ < $ 2 & 2.000 $\pm$ 1.414\\
\nmiss $\leq 1$ & 2.000 $\pm$ 1.414\\
\nprecision $\geq 3$ & 0.000 $\pm$ 0.000 \\
 \chiCB $ < $ 3& 0.000 $\pm$ 0.000  \\
\nphi $> 0$ &0.000 $\pm$ 0.000 \\
\dRll $ > 0.2$ & 0.000 $\pm$ 0.000\\ 
\hline
\end{tabular}
\caption{Cutflow for SR-$\mu\mu$ for Run 2 data.}
\label{tab:data_cutflow_srmm}
\end{center}
\end{table}

\begin{table}[htb]
\small
\begin{center}
\begin{tabular}{l  c} 
Cut & data yield\\
\hline
Pass trigger and at least 2 baseline leptons & 65484.000 $\pm$ 255.898\\
2 leading leptons are a muon and an electron & 191.000 $\pm$ 13.820 \\ 
$\pt > 65$ GeV  & 128.000 $\pm$ 11.314\\
\absdz$ > $ 3 mm & 107.000 $\pm$ 10.344\\
both leptons pass isolation & 98.000 $\pm$ 9.899\\
\hline
muon pass cosmic veto & 86.000 $\pm$ 9.274\\
muon \tavg$ <$ 30 & 82.000 $\pm$ 9.055\\
muon \chiID$ < $ 2 & 72.000 $\pm$ 8.485 \\
muon \nmiss $\leq 1$ & 68.000 $\pm$ 8.246\\
muon \nprecision $\geq 3$ & 6.000 $\pm$ 2.449 \\
muon \chiCB $ < $ 3 &  0.000 $\pm$ 0.000\\
muon \nphi $> 0$ & 0.000 $\pm$ 0.000\\
\hline
electron \dpt $ \geq$ -0.5  &0.000 $\pm$ 0.000 \\
electron \chiID $ < $ 2 & 0.000 $\pm$ 0.000 \\
electron \nmiss $\leq 1$ &0.000 $\pm$ 0.000 \\
\hline
\dRll $ > 0.2$ &  0.000 $\pm$ 0.000 \\ 
\hline
\end{tabular}
\caption{Cutflow for SR-$e\mu$ for Run 2 data.}
\label{tab:data_cutflow_sremu}
\end{center}
\end{table}

\section{Interpretation}

While these results are designed to be applicable to many \ac{BSM} physics models, we use them to set bounds on the possible parameter space of sleptons in \ac{GMSB} \ac{SUSY} models, discussed in \todo{theory ref}, now further restricted due to the lack of events in any \ac{SR} in this analysis. These limits are computed using HistFitter \cite{histfitter}, an \ac{ATLAS} framework that combines the observed number of events with uncertainties on background predictions and MC modeling and calculates both model dependent and model independent limits using the CL$_{\text{S}}$ technique \cite{CLs-1}. 

Model dependent limits are set in terms of the mass and lifetime of sleptons. Four different limits are set using the results of this analysis: each selectron, smuon, or stau \ac{NSLP}, or the mass degenerate case with all three as co-NSLPs. For the selectron and smuon cases, the results from SR-$ee$ and SR-$\mu\mu$, respectively, are used. While for the stau and co-NSLP cases, all three SRs are combined.

\todo{plots when finished}

This search provides almost an order of magnitude of exclusion over the previous result from OPAL and 90 \GeV. 


Additionally, model independent limits are set on generic new physics processes. These limits are based on the visible cross-section of new physics ($\langle \epsilon \sigma^{95}_{\text{obs}}$) and the observed ($S^{95}_{\text{obs}}$) and expected ($S^{95}_{\text{exp}}$) number of signal events. 

\todo{table when finished.}
\todo{what are toys}



\section{Future Prospects}

\chapter{Interpretation}

Since no events are observed, limits are set on the parameter space of \ac{GMSB} \ac{SUSY} mdoels. Limits are computed using HistFitter \cite{histfitter}, an \ac{ATLAS} framework that combines the observed number of events with uncertainties on background predictions and signal systematics and calculates both model dependent and model independent limits using the CL$_{\text{s}}$ technique \cite{CLs-1}. CL$_{\text{s}}$ is a useful tool for particle physics analyses. Somewhere between a purely frequentist and purlely basian method of deriving a confidence interval, the CL$_{\text{s}}$ describes the confidence in a signal-only hypothesis. The limit curve is drawn where CL$_{\text{s}}(m_{\tilde{\ell}}) \leq 5\%$, meaning that the probability of having falsely excluded a \slep of a given mass is less than or equal to 5\%. The smaller the value of CL$_{\text{s}}$, the lower the probability that a \slep with a given mass exists.  

\section{Slepton Limits}
Limits are set on the possible masses and lifetimes of long-lived \slep. Four different limits are set using the results of this analysis: each \selec, \smu, or \stau \ac{NLSP}, or the mass degenerate case with all three as co-\ac{NLSP}s. Results for all four scenarios combined are shown in \autoref{fig:excl_limit}, and the individual limit for each of the four scenarios along with their observed CL$_{\text{s}}$ value is shown in \autoref{fig:plot_cls}.

For the \selec \ac{NLSP} and \smu \ac{NLSP} cases, the results from SR-$ee$ and SR-$\mu\mu$, respectively, are used. While for the \stau and co-\ac{NLSP} cases, all three SRs are combined. For a lifetime of 0.1 ns, \selec NLSP, \smu \ac{NLSP}, \stau \ac{NLSP}, and co-NLSP scenarios are excluded for \slep masses up to 720~\GeV, 680~\GeV, 340~\GeV, and 820~\GeV, respectively. Co-NSLP events are also excluded up to 10 ns for masses below 330~\GeV. The previous from OPAL~\cite{opal} is surpassed by nearly an order of magnitude for \selec-, \smu-, and co-NSLP scenarios.


 \begin{figure}[!ht]
    \centering
        \includegraphics[width=0.6\textwidth]{figures/limits/displaced_lepton_summary.pdf}
    \caption{
    Expected (dashed) and observed (solid) exclusion contours for $\tilde{e}$ \ac{NLSP}, $\tilde{\mu}$ \ac{NLSP}, $\tilde{\tau}$ \ac{NLSP}, and co-NLSP production as a function of the lifetime at 95\% CL.
    }
    \label{fig:excl_limit}
\end{figure}

\begin{figure}[!ht]
    \centering
        \includegraphics[width=0.45\textwidth]{figures/limits/SlepSlep_SRee_clsobs_ps-eps-converted-to.pdf}
        \includegraphics[width=0.45\textwidth]{figures/limits/SlepSlep_SRmm_clsobs_ps-eps-converted-to.pdf}
        \includegraphics[width=0.45\textwidth]{figures/limits/Stau_clsobs_ps-eps-converted-to.pdf}
        \includegraphics[width=0.45\textwidth]{figures/limits/Comb_clsobs_ps-eps-converted-to.pdf}
    \caption{Individual exclusion curves for four different \ac{NLSP} scenarios: \selec (top left), \smu (top right), \stau (bottom left), co-\ac{NLSP} (bottom right). Overlaid numbers represent the observed CLs values.}
    \label{fig:plot_cls}
\end{figure}

\subsection{Comparison to LEP}

The previous results include both right- and left-handed slepton production and cannot be directly compared to the results from \ac{LEP}. However, limits limited to the assumptions of the \ac{LEP} serach are shown in \todo{figure}

\todo{RH and LH when available}

\FloatBarrier

\section{Model-Indpendent Limits}
Additionally, model independent limits are set on generic new physics processes in the \acp{SR}. These limits are based on the visible cross-section of new physics ($\langle \epsilon \sigma^{95}_{\text{obs}} \rangle$) and the observed ($S^{95}_{\text{obs}}$) and expected ($S^{95}_{\text{exp}}$) number of signal events that would be measured. Visible \ac{BSM} cross sections above 0.02 fb are excluded in each \ac{SR} These values are shown in \autoref{table.results.exclxsec.pval.upperlimit.SR}

\begin{table}
\centering
\begin{tabular*}{\textwidth}{@{\extracolsep{\fill}}lccccccc}
\noalign{\smallskip}\noalign{\smallskip}
{ Signal channel} & $\langle\epsilon{ \sigma}\rangle_\text{obs}^{95}$[fb]  &  $S_\text{ obs}^{95}$  & $S_\text{ exp}^{95}$  & $p(s=0)$ 
\\
\noalign{\smallskip}\hline\noalign{\smallskip}
%%
 SR-$ee$   & $0.02$ &  $3.0$ & $ { 3.1 }^{ +1.1 }_{ -0.1 }$ &  $ 0.47$ \\%                                                                                 
 SR-$\mu\mu$   &$0.02$ &  $3.0$ & $ { 3.0 }^{ +0.1 }_{ -0.0 }$ & $ 0.72$ \\%                                                                               
SR-$e\mu$   &$0.02$ &  $2.7$ & $ { 3.0 }^{ +0.0 }_{ -0.0 }$ & $ 0.10$ \\%
\noalign{\smallskip}\hline\noalign{\smallskip}
\end{tabular*}
\caption[Breakdown of upper limits.]{
Left to right: 95\% CL upper limits on the visible cross section
($\langle\epsilon\sigma\rangle_\text{ obs}^{95}$) and on the number of
signal events ($S_\text{ obs}^{95}$ ).  The third column
($S_\text{ exp}^{95}$) shows the 95\% CL upper limit on the number of
signal events, given the expected number of background events. The last column shows the p-value of the no-signal hypothesis.
\label{table.results.exclxsec.pval.upperlimit.SR}}
\end{table}
\chapter{Conclusions and Future Improvements}
\label{chap:conclusions}

The lifetime is an underexplored region of phase space for many natural \ac{BSM} theories and this thesis describes a search that covers a large gap in signature space at the \ac{LHC}. In $139~ \ifb$ of $\sqrt{s} = 13~\TeV$ \ac{LHC} data collected by the \ac{ATLAS} detector, no signs of \ac{BSM} physics are seen. Three orthogonal signal regions are defined and less than 1 background event is predicted in each. 0 events are seen and limits are set on the possible mass and lifetime of \slep in \ac{GMSB} \ac{SUSY} models. For a lifetime of 0.1 ns, selectron NLSP, smuon NSLP, stau NSLP, and co-NLSP scenarios are excluded for slepton masses up to 720~\GeV, 720~\GeV, 370~\GeV, and 830~\GeV, respectively, exceeding the LEP co-NSLP limit~\cite{LEP-comb} by almost an order of magnitude. Co-NSLP events are also excluded up to 10 ns for masses below 330~\GeV.

A minimal, though substantial, set of optimizations were done to obtain this result, but future analyses using the same dataset could improve upon this result in several ways. First, further optimization could be done of the electron reconstruction algorithm in order to boost efficiency at high \absdz as well as reduce the systematic uncertainty due to the variation in the lepton displacement selection efficiency. Also, additional signal regions could be defined using a \ac{MET} trigger to increase sensitivity. This signal model has real \ac{MET} due to the gravitinos (and neutrinos in the \stau decays). More importantly, muons are not included in the \ac{MET} calculation at the trigger stage, so an \ac{MET} trigger is an effective displaced muon trigger. A bolder improvement would be to reconstruct all recorded events with \ac{LRT}, instead of the filtered 10\% that is currently used. This would enable more creativity in signal region design and more centralized displaced lepton identification algorithms. 

Additional optimizations should be performed to prepare for an analysis in Run 3. In particular, this search suffered from trigger limitations. A simple improvement to the data collection scheme would be to introduce an EM-only + MS-only trigger. This would likely have a low enough fake rate that it could be added in relatively easily and \pt and $\eta$ requirements on the single EM-only and MS-only signatures could be relaxed. A more exciting improvement would be to implement \ac{LRT} in the trigger and design trigger-level displaced lepton identification algorithms. This would likely have a low enough rate to reduce the \pt requirements substantially. 

This result should also be interpreted in conjunction with other searches, including prompt searches and those for disappearing tracks and stable massive particles, to probe the full possible lifetime space of sleptons and make a more lifetime-inclusive statement about \ac{GMSB} \ac{SUSY} at the \ac{LHC}. Furthermore, the minimal event-level requirements make this result model-independent and applicable to any \ac{BSM} decay resulting in displaced leptons.



\cleardoublepage 
\singlespacing
\printbibliography[heading=bibintoc,title={References}]

% Figures and tables, if you decide to leave them to the end
%\input{figure}
%\input{table}

\end{document}

